\documentclass[12pt]{article}
\usepackage{fullpage}
\usepackage{amssymb,amsmath}

\newtheorem{theorem}{Theorem}

 \newcommand{\Header}[1]{\begin{center} {\Large\bf #1} \end{center}}
 \newcommand{\header}[1]{\begin{center} {\large\bf #1} \end{center}}
\setlength{\parindent}{0.0in}
\setlength{\parskip}{1ex}


%\newif\ifnotes\notestrue
\newif\ifnotes\notesfalse


\usepackage{amsmath,amssymb,amsthm,amsfonts,latexsym,bbm,xspace,graphicx,float,mathtools,epigraph}
\usepackage[backref,colorlinks,citecolor=blue,bookmarks=true]{hyperref}
\usepackage{enumitem,manyfoot,fullpage}
\usepackage{subfig,tikz,framed}
\usepackage{endnotes}
\usepackage{braket}


\usepackage{fullpage}
\usepackage{hyperref}
\usepackage{pdfsync}
\usepackage{microtype}
\usepackage{color}
\usepackage{cleveref}

\newtheorem*{namedtheorem}{\theoremname}
\newcommand{\theoremname}{testing}
\newenvironment{named}[1]{ \renewcommand{\theoremname}{#1} \begin{namedtheorem}} {\end{namedtheorem}}
\newtheorem{lemma}[theorem]{Lemma}
\newtheorem{claim}[theorem]{Claim}
\newtheorem{proposition}[theorem]{Proposition}
\newtheorem{fact}[theorem]{Fact}
\newtheorem{corollary}[theorem]{Corollary}

\theoremstyle{definition}
\newtheorem{definition}[theorem]{Definition}
\newtheorem{remark}[theorem]{Remark}
\newtheorem{observation}[theorem]{Observation}
\newtheorem{notation}[theorem]{Notation}
\newtheorem{example}[theorem]{Example}
\newtheorem{examples}[theorem]{Examples}
\newtheorem{exercise}{Exercise}


\newenvironment{quotenote}{
\begin{quote}
  \footnotesize
\noindent{\bf Note:}}
{\end{quote}
}


% probability and other mathops
\renewcommand{\Pr}{\mathop{\bf Pr\/}}
\newcommand{\E}{\mathop{\bf E\/}}
\newcommand{\Ex}{\mathop{\bf E\/}}
\newcommand{\Var}{\mathop{\bf Var\/}}
\newcommand{\Cov}{\mathop{\bf Cov\/}}
\newcommand{\stddev}{\mathop{\bf stddev\/}}
\newcommand{\littlesum}{\mathop{{\textstyle \sum}}}
\newcommand{\apx}{\mathop{\approx}}

\newcommand{\epr}{\textsc{EPR}}

\newcommand{\Zt}{\ensuremath{\Z_t}}
\newcommand{\Zp}{\ensuremath{\Z_p}}
\newcommand{\Zq}{\ensuremath{\Z_q}}
\newcommand{\ZN}{\ensuremath{\Z_N}}
\newcommand{\Zps}{\ensuremath{\Z_p^*}}
\newcommand{\ZNs}{\ensuremath{\Z_N^*}}
\newcommand{\JN}{\ensuremath{\J_N}}
\newcommand{\QR}{\ensuremath{\mathbb{QR}}}
\newcommand{\QRN}{\ensuremath{\QR_{N}}}
\newcommand{\QRp}{\ensuremath{\QR_{p}}}

% mathrm terms
\newcommand{\poly}{\mathrm{poly}}
\newcommand{\negl}{\mathrm{negl}}
\newcommand{\Tr}{\mathrm{Tr}}
\newcommand{\polylog}{\mathrm{polylog}}
\newcommand{\size}{\mathrm{size}}
\newcommand{\avg}{\mathop{\mathrm{avg}}}
\newcommand{\sgn}{\mathrm{sgn}}
\newcommand{\dist}{\mathrm{dist}}
\newcommand{\spn}{\mathrm{span}}
\newcommand{\supp}{\mathrm{supp}}
\newcommand{\Val}{\mathrm{Val}}
\newcommand{\Opt}{\mathrm{Opt}}
\newcommand{\LPOpt}{\mathrm{LPOpt}}
\newcommand{\SDPOpt}{\mathrm{SDPOpt}}
\newcommand{\vol}{\mathrm{vol}}
\newcommand{\Id}{\mathbb{I}}

\newcommand{\Ext}{\mathrm{Ext}}
\newcommand{\IP}{\mathrm{IP}}

% number systems
\newcommand{\R}{\mathbbm R}
\newcommand{\C}{\mathbbm C}
\newcommand{\N}{\mathbbm N}
\newcommand{\Z}{\mathbbm Z}
\newcommand{\F}{\mathbbm F}
\newcommand{\Q}{\mathbbm Q}

\newcommand{\mH}{\mathcal{H}}

% complexity classes
\newcommand{\PTIME}{\mathsf{P}}
\newcommand{\NP}{\mathsf{NP}} \newcommand{\np}{\NP}

% short forms
\newcommand{\eps}{\varepsilon}
\newcommand{\lam}{\lambda}
\newcommand{\vphi}{\varphi}
\newcommand{\la}{\langle}
\newcommand{\ra}{\rangle}
\newcommand{\wt}[1]{\widetilde{#1}}
\newcommand{\wh}[1]{\widehat{#1}}
\newcommand{\ul}[1]{\underline{#1}}
\newcommand{\ol}[1]{\overline{#1}}
\newcommand{\ot}{\otimes}
\newcommand{\Ra}{\Rightarrow}
\newcommand{\half}{\tfrac{1}{2}}
\newcommand{\grad}{\nabla}
\newcommand{\sse}{\subseteq}


% calligraphic letters
\newcommand{\calA}{\mathcal{A}}
\newcommand{\calB}{\mathcal{B}}
\newcommand{\calC}{\mathcal{C}}
\newcommand{\calD}{\mathcal{D}}
\newcommand{\calE}{\mathcal{E}}
\newcommand{\calF}{\mathcal{F}}
\newcommand{\calG}{\mathcal{G}}
\newcommand{\calH}{\mathcal{G}}
\newcommand{\calI}{\mathcal{I}}
\newcommand{\calJ}{\mathcal{J}}
\newcommand{\calK}{\mathcal{K}}
\newcommand{\calL}{\mathcal{L}}
\newcommand{\calM}{\mathcal{M}}
\newcommand{\calN}{\mathcal{N}}
\newcommand{\calO}{\mathcal{O}}
\newcommand{\calP}{\mathcal{P}}
\newcommand{\calQ}{\mathcal{Q}}
\newcommand{\calR}{\mathcal{R}}
\newcommand{\calS}{\mathcal{S}}
\newcommand{\calT}{\mathcal{T}}
\newcommand{\calU}{\mathcal{U}}
\newcommand{\calV}{\mathcal{V}}
\newcommand{\calW}{\mathcal{W}}
\newcommand{\calX}{\mathcal{X}}
\newcommand{\calY}{\mathcal{Y}}
\newcommand{\calZ}{\mathcal{Z}}

%\newcommand{\ketbra}[2]{|#1\rangle\langle#2|}
\newcommand{\hmin}{H_{\rm min}}
\newcommand{\Hmin}{H_{\rm min}}

\newcommand{\myfig}[4]{\begin{figure}[H] \begin{center} \includegraphics[width=#1\textwidth]{#2} \caption{#3} \label{#4} \end{center} \end{figure}} 

\newcommand{\bit}{\ensuremath{\{0,1\}}}

%%% CRYPTO-RELATED NOTATION

% length of a string
\newcommand{\len}[1]{\lvert{#1}\rvert}
\newcommand{\lenfit}[1]{\left\lvert{#1}\right\rvert}
% length of some vector, element
\newcommand{\length}[1]{\lVert{#1}\rVert}
\newcommand{\lengthfit}[1]{\left\lVert{#1}\right\rVert}


% types of indistinguishability
\newcommand{\compind}{\ensuremath{\stackrel{c}{\approx}}}
\newcommand{\statind}{\ensuremath{\stackrel{s}{\approx}}}
\newcommand{\perfind}{\ensuremath{\equiv}}

% font for general-purpose algorithms
\newcommand{\algo}[1]{\ensuremath{\mathsf{#1}}}
% font for general-purpose computational problems
\newcommand{\problem}[1]{\ensuremath{\mathsf{#1}}}
% font for complexity classes
\newcommand{\class}[1]{\ensuremath{\mathsf{#1}}}


% KEYS AND RELATED

\newcommand{\key}[1]{\ensuremath{#1}}

\newcommand{\pk}{\key{pk}}
\newcommand{\vk}{\key{vk}}
\newcommand{\sk}{\key{sk}}
\newcommand{\mpk}{\key{mpk}}
\newcommand{\msk}{\key{msk}}
\newcommand{\fk}{\key{fk}}
\newcommand{\id}{id}
\newcommand{\keyspace}{\ensuremath{\mathcal{K}}}
\newcommand{\msgspace}{\ensuremath{\mathcal{M}}}
\newcommand{\ctspace}{\ensuremath{\mathcal{C}}}
\newcommand{\tagspace}{\ensuremath{\mathcal{T}}}
\newcommand{\idspace}{\ensuremath{\mathcal{ID}}}

\newcommand{\concat}{\ensuremath{\|}}

% GAMES

% advantage
\newcommand{\advan}{\ensuremath{\mathbf{Adv}}}

% different attack models
\newcommand{\attack}[1]{\ensuremath{\text{#1}}}

\newcommand{\atk}{\attack{atk}} % dummy attack
\newcommand{\indcpa}{\attack{ind-cpa}}
\newcommand{\indcca}{\attack{ind-cca}}
\newcommand{\anocpa}{\attack{ano-cpa}} % anonymous
\newcommand{\anocca}{\attack{ano-cca}}
\newcommand{\euacma}{\attack{eu-acma}} % forgery: adaptive chosen-message
\newcommand{\euscma}{\attack{eu-scma}} % forgery: static chosen-message
\newcommand{\suacma}{\attack{su-acma}} % strongly unforgeable

% ADVERSARIES
\newcommand{\attacker}[1]{\ensuremath{\mathcal{#1}}}

\newcommand{\Adv}{\attacker{A}}
\newcommand{\AdvA}{\attacker{A}}
\newcommand{\AdvB}{\attacker{B}}
\newcommand{\Dist}{\attacker{D}}
\newcommand{\Sim}{\attacker{S}}
\newcommand{\Ora}{\attacker{O}}
\newcommand{\Inv}{\attacker{I}}
\newcommand{\For}{\attacker{F}}

% CRYPTO SCHEMES

\newcommand{\scheme}[1]{\ensuremath{\text{#1}}}

% pseudorandom stuff
\newcommand{\prg}{\algo{PRG}}
\newcommand{\prf}{\algo{PRF}}
\newcommand{\prp}{\algo{PRP}}

% symmetric-key cryptosystem
\newcommand{\skc}{\scheme{SKC}}
\newcommand{\skcgen}{\algo{Gen}}
\newcommand{\skcenc}{\algo{Enc}}
\newcommand{\skcdec}{\algo{Dec}}

% public-key cryptosystem
\newcommand{\pkc}{\scheme{PKC}}
\newcommand{\pkcgen}{\algo{Gen}}
\newcommand{\pkcenc}{\algo{Enc}} % can also use \kemenc and \kemdec
\newcommand{\pkcdec}{\algo{Dec}}

% digital signatures
\newcommand{\sig}{\scheme{SIG}}
\newcommand{\siggen}{\algo{Gen}}
\newcommand{\sigsign}{\algo{Sign}}
\newcommand{\sigver}{\algo{Ver}}

% message authentication code
\newcommand{\mac}{\scheme{MAC}}
\newcommand{\macgen}{\algo{Gen}}
\newcommand{\mactag}{\algo{Tag}}
\newcommand{\macver}{\algo{Ver}}

% key-encapsulation mechanism
\newcommand{\kem}{\scheme{KEM}}
\newcommand{\kemgen}{\algo{Gen}}
\newcommand{\kemenc}{\algo{Encaps}}
\newcommand{\kemdec}{\algo{Decaps}}

% identity-based encryption
\newcommand{\ibe}{\scheme{IBE}}
\newcommand{\ibesetup}{\algo{Setup}}
\newcommand{\ibeext}{\algo{Ext}}
\newcommand{\ibeenc}{\algo{Enc}}
\newcommand{\ibedec}{\algo{Dec}}

% hierarchical IBE (as key encapsulation)
\newcommand{\hibe}{\scheme{HIBE}}
\newcommand{\hibesetup}{\algo{Setup}}
\newcommand{\hibeext}{\algo{Extract}}
\newcommand{\hibeenc}{\algo{Encaps}}
\newcommand{\hibedec}{\algo{Decaps}}

% binary tree encryption (as key encapsulation)
\newcommand{\bte}{\scheme{BTE}}
\newcommand{\btesetup}{\algo{Setup}}
\newcommand{\bteext}{\algo{Extract}}
\newcommand{\bteenc}{\algo{Encaps}}
\newcommand{\btedec}{\algo{Decaps}}

% trapdoor functions
\newcommand{\tdf}{\scheme{TDF}}
\newcommand{\tdfgen}{\algo{Gen}}
\newcommand{\tdfeval}{\algo{Eval}}
\newcommand{\tdfinv}{\algo{Invert}}
\newcommand{\tdfver}{\algo{Ver}}

%%% PROTOCOLS

\newcommand{\out}{\text{out}}
\newcommand{\view}{\text{view}}


\newcommand{\proj}[1]{\ket{#1}\!\!\bra{#1}}
\newcommand{\ketbra}[2]{\ket{#1}\!\bra{#2}}
\renewcommand{\braket}[2]{\langle #1 \vert  #2\rangle}


\newcommand{\inp}[2]{\langle{#1}|{#2}\rangle} 


%%%%%%%%%%%% document-writing macros %%%%%%%%%%%%

\ifnotes
\usepackage{color}
\definecolor{mygrey}{gray}{0.50}
\newcommand{\notename}[2]{{\textcolor{mygrey}{\footnotesize{\bf (#1:} {#2}{\bf ) }}}}
\newcommand{\noteswarning}{{\begin{center} {\Large WARNING: NOTES ON}\endnote{Warning: notes on}\end{center}}}

\else

\newcommand{\notename}[2]{{}}
\newcommand{\noteswarning}{{}}
\newcommand{\notesendofpaper}{}

\fi

\newcommand{\tnote}[1]{{\notename{Thomas}{#1}}}
\newcommand{\note}[1]{{\notename{Note}{#1}}}
%\newcommand{\note}[1]{}


\bibliographystyle{alpha}



\begin{document}
\header{COM-440, Introduction to Quantum Cryptography, Fall 2025}
{\bf Final} \hfill {\bf due: October 23rd, 2025}


\medskip

\hrule

\medskip 

Ground rules: 

{\bf Please
  format your solutions so that each problem begins on a new page, and
  so that your name appears at the top of each page.}

\medskip
\hrule
\medskip

The total points value of the problems is 25 points. Your midterm grade will be the \emph{minimum} of your total points number and \emph{20}. This means that we do not expect you to solve all problems. Instead, we encourage you to spend the first 5-10 minutes looking at all problems and deciding which ones to attempt. Your goal is to collect as close to 20 points as possible in total, not necessarily to solve all questions. 
 
\medskip
{\bf Problems:}

\begin{enumerate}
		



\item \textbf{Optimal qubit strategies in the CHSH game.}\\
\emph{This problem is a little harder. It will be graded, but it is ok if you are not able to complete it. You should still read it, as the result of the problem is used in the following problem.}\\
The goal of this problem is to evaluate the maximum success probability that can be achieved in the CHSH game by players sharing a two-qubit entangled state of the form 
\begin{equation}\label{eq:psitheta}
\ket{\psi_\theta}_{AB} = \cos(\theta) \ket{0}_A\ket{0}_A + \sin(\theta)\ket{1}_A\ket{1}_B,
\end{equation}
where $\theta \in [0,\pi/4]$ (other values of $\theta$ can be reduced to this case by simple change of basis or phase flip). 
 Having fixed the state, what are the optimal measurements for the players, and what is their success probability? 

We will assume the players each makes a basis measurement on their qubit. Recall that an observable 
 $O$ is a $2\times 2$ matrix with complex entries such that $O$ is Hermitian ($O^\dagger = O$) and squares to identity ($O^2 = \Id$). For any single-qubit basis measurement $\{\ket{u_0},\ket{u_1}\}$, there is an associated observable is $O = \ket{u_0}\bra{u_0}-\ket{u_1}\bra{u_1}$. Conversely, any observable that is not $\pm\Id$ has two non-degenerate eigenvalues $+1$ and $-1$, so we can uniquely identify it with a basis. 

To reduce the number of cases to consider we first make a few symmetry observations. 
\begin{enumerate}
\item[(a)] Let $O$ be a single-qubit observable such that $O$ is non-degenerate $(O\neq \pm \Id)$. Show that there exists real numbers $\alpha,\beta,\gamma$ such that $\alpha^2 + \beta^2 + \gamma^2 = 1$ and $O = \alpha X + \beta Y + \gamma Z$, with $X,Y,Z$ the standard Pauli matrices. 
\item[(b)] Let $\mathcal{B} = A_0\otimes B_0 + A_1\otimes B_0 + A_0\otimes B_1 - A_1 \otimes B_1$. Show that the success probability of the strategy in the CHSH game is $p_s = \frac{1}{2} + \frac{1}{4} \bra{\psi_\theta}\mathcal{B} \ket{\psi_\theta}$.
\item[(c)] Argue that for the purposes of computing the maximum success probability in the CHSH game of players using state $\ket{\psi_\theta}_{AB}$ as in~\eqref{eq:psitheta} we may without loss of generality restrict our attention to observables of the form $A_x = \cos(\alpha_x) X + \sin(\alpha_x)Y$ and $B_y =  \cos(\beta_x) X + \sin(\beta_y)Y$ for some angles $\alpha_x,\beta_y \in [0,2\pi)$. \emph{[Hint: do a rotation on the Bloch sphere.]}
\end{enumerate}
Based on the symmetry argument from the previous questions we have reduced our problem to understanding the maximum value that $\bra{\psi_\theta} \mathcal{B} \ket{\psi_\theta}$ can take, when $\ket{\psi_\theta}$ is as in~\eqref{eq:psitheta} and $\mathcal{B}$ is defined from observables $A_x,B_y$ as in (b). To understand this maximum value we compute the spectral decomposition of $\mathcal{B}$. 
\begin{enumerate}
\item[(d)] Show that $(Z\otimes \Id) \mathcal{B} (Z\otimes \Id) =  (\Id\otimes Z) \mathcal{B} (\Id\otimes Z) = -\mathcal{B}$. \emph{[Hint: use the special form of $A_x$ and $B_y$ you obtained from question (b).]}
\item[(e)] Show that $\mathcal{B}$ has a basis of eigenvectors of the form $\ket{\phi_{ab}} = e^{i\theta_{ab}} \ket{ab} + \ket{\overline{a}\overline{b}}$, where $a,b\in\{0,1\}$ and $\overline{a}=1-a$, $\overline{b}=1-b$. Note that up to local rotations this is the Bell basis. 
\item[(f)] Write $\mathcal{B}^2$ as a $4\times 4$ matrix depending on the angles $\alpha_x,\beta_y$, and show that $\Tr(\mathcal{B}^2)\leq 16$.
\item[(g)] Show that the largest success probability achievable in the CHSH game using $\ket{\psi_\theta}_{AB}$ is at most  $\frac{1}{2} + \frac{1}{4}\sqrt{1+\sin(2\theta)}$. \emph{[Hint: Decompose $\ket{\psi_\theta}$ in the eigenbasis of $\mathcal{B}$. Use (f) and the symmetries from (d) to bound the bound the success probability via the expression found in (b).]} 
\item[(h)] Give a strategy for the players which achieves this value, i.e. specify the players'  observables. 
\end{enumerate}

\item \textbf{Trading success probability for randomness in the CHSH game.}\\
The goal of this problem is to show that, if players succeed with higher and higher probability in the CHSH game then Alice's outputs in the game must contain more and more randomness. 
\begin{enumerate}
\item[(a)] Suppose that Alice and Bob play the CHSH game using a two-qubit entangled state $\ket{\psi_\theta}_{AB}$ as in~\eqref{eq:psitheta}. Let $p_\theta(a|x)$ be the probability that, in this strategy, Alice returns answer $a\in\{0,1\}$ to question $x\in\{0,1\}$. Show that $\max_{a,x} p_\theta(a|x) \leq \cos (2\theta)$. 
\item[(b)] Let $p_s = \frac{1}{2}+\frac{1}{4}I$ be the players' success probability in CHSH, where $I\in [-2,2]$. Using (g) from the previous problem, deduce from (a) that 
$$\forall a,x\in\{0,1\},\qquad p_\theta(a|x) \leq \frac{1}{2}\Big(1+\sqrt{2-\frac{I^2}{4}}\Big).$$
\item[(c)] Suppose now the players use any single-qubit strategy (not necessarily using $\ket{\psi_\theta}$). Prove a lower bound on the conditional min-entropy $\Hmin(A|X=x)$, for any $x\in \{0,1\}$, that is generated in Alice's outputs, as a function of the players' success probability in the CHSH game. 
\item[(d)] Show that the bound from (b) is tight: for any $\theta\in[0,\pi/4]$ find a strategy for the players using $\ket{\psi_\theta}$ such that $\max_{a,x} p_\theta(a|x) = \frac{1}{2}(1+\sqrt{2-I^2/4})$.
\end{enumerate}

\end{enumerate}
\end{document}






 
















