\chapter{}

\begin{exercises}

\item {\bf BB'84 fails in the Device-Independent setting}\\
Consider the purified variant of the BB'84 protocol. Suppose that Eve prepares the state $\rho_{ABE}$ in the following form:
\begin{equation*}
 \rho_{ABE} = \frac12\sum_{x, z=0}^1 \ket{x z}\bra{x z}_A \otimes \ket{x z}\bra{x z}_B \otimes \ket{x z}\bra{x z}_E.
\end{equation*}
where $\ket{x z}$ is short-hand notation for $\ket{x}\otimes \ket{z}$.
Now suppose Alice and Bob's measurement devices, instead of measuring a single qubit in the standard or Hadamard bases, as they think the device does, in fact performs the following:
\begin{itemize}
\item When the device is told to measure in the standard basis, it measures the first qubit of the two-qubit system associated with the device in the standard basis;
\item When the device is told to measure in the Hadamard basis, it measures the \emph{second} qubit of the two-qubit system associated with the device in the \emph{standard} basis.
\end{itemize}
\begin{enumerate}
\item Alice and Bob put blind faith in their hardware and attempt to implement BB'84. They want to check that their state is an EPR pair, so Alice asks her box to measure in the standard basis. The box returns a measurement outcome of $0$. What is the post-measurement state?
%\solopen{The post-measurement state can be found by applying the projection onto the $\ket 0$ state in Alice's system, the result is:
%$$\sum_{z=0}^1\proj{0 z}_A\otimes \proj{0 z}_B\otimes \proj{0 z}_E$$}
\item After Alice's measurement, Bob asks his box to measure in the Hadamard basis. What measurement outcome will Bob receive?
%\solopen{Bob's reduced state is $\frac12\proj{0,0} + \frac12\proj{0,1}$. The box measures the second qubit in the standard basis and therefore produces a uniformly random bit.}
\item Suppose that instead Bob asks his box to measure in the standard basis. What measurement outcome will Bob receive?
%\solopen{Bob's reduced state is $\frac12\proj{0,0} + \frac12\proj{0,1}$. The box measures the first qubit in the standard basis and therefore always returns $0$.}
\item After Bob asked his box to measure in the standard basis, Eve measures her first qubit in the standard basis. What measurement outcome does she receive?
%\solopen{Eve obtains a $0$ as well.}
\item As per the BB'84 protocol, Alice and Bob look at all the rounds on which they made the same measurement as each other. They pick a random subset of those rounds and test whether they received the same output on all the rounds. With what probability do they pass the test?
%\solopen{We saw in the previous questions that they pass the test on rounds where they both ask their box to measure in the standard basis. By symmetry, the same holds for rounds where they both ask their box to measure in the Hadamard basis. Therefore, they pass the test with probability 1.}
\item Let $T'$ be set of rounds on which Alice and Bob made the same measurement but didn't perform a test. Let $\set{\theta_j}_{j\in T'}$ be the measurements they made and $\set{x_j}_{j\in T'}$ be the results they received. The $\theta_j$ have been communicated over the public channel. Eve wishes to learn the $x_j$. Which measurements should she make?
%\solopen{Measure qubit $\theta_j$ in the standard basis at all rounds $j\in T'$. We saw in the previous problems that on each two-qubit device, Eve will get the same outcome as Alice and Bob if they all choose the same qubit to measure.}
\item Let $X$ be the classical key generated by Alice and Bob. What is $\Hmin(X\mid E)$, where $E$ is Eve's system?
%\solopen{Eve can learn $X$ entirely, so $\Hmin(X\mid E) = 0$.}
\end{enumerate}

\item {\bf Commuting observables are compatible}\\
In order to analyze the upcoming protocols we need to use the following fact:
If $A,B$ are commuting observables, then the product of the results of measuring $A$ and then $B$ is the same as the result of measuring $AB$.

In this exercise, you'll verify a special case of the previous fact in a way that should illuminate the proof. Consider $X\otimes X$ and $Z\otimes Z$. Since they commute, they have a simultaneous eigenbasis. It happens to consist of the Bell states, which are
\[
	\ket{\Phi^+} = \frac1{\sqrt2}\left(\ket{00} + \ket{11}\right)
	; \quad \ket{\Phi^-} = \frac1{\sqrt2}\left(\ket{00} - \ket{11}\right)
\]
\[
	\ket{\Psi^+} = \frac1{\sqrt2}\left(\ket{01} + \ket{10}\right)
	; \quad \ket{\Psi^-} = \frac1{\sqrt2}\left(\ket{01} - \ket{10}\right)
\]
\begin{enumerate}
\item Suppose we measure the two-qubit state $\ket\phi$ using the observable $X\otimes X$ and receive the outcome $-1$. The post-measurement state belongs to which $2$-dimensional eigenspace?
%\solopen{$X\otimes X$ has eigenvalues $+1$ and $-1$ with equal multiplicity, so each one should induce an eigenspace of dimension $2$ in the $4$-dimensional space of two qubits.
%Among states in the Bell basis, $\ket{\Phi^-},\ket{\Psi^-}$ are the $-1$ eigenstates of $X\otimes X$. }
\item Next, we measure the observable $Z\otimes Z$ and receive the outcome $1$. What is the post-measurement state $\ket{\phi'}$?
%\solopen{$\ket{\Phi^-}$ is the $+1$ eigenstate of $Z\otimes Z$ which lives in the $-1$ eigenspace of $X\otimes X$.}
\item Suppose that instead we performed the measurement \mbox{$-Y\otimes Y = (X\otimes X)(Z\otimes Z)$} directly, and the post-measurement state had nonzero overlap with $\ket{\phi'}$. What measurement outcome would we have received? (In other words, what is the eigenvalue of the $-(Y\otimes Y)$-eigenspace in which $\ket{\phi'}$ lies?). Compare your answer to the product of the answers in the previous problems.
%\solopen{Recall from the last problem that $\ket{\phi'} = \ket{\Phi^-}$.
%	Again another way to think about the question is to ask what measurement outcome we receive when we measure the observable $-(Y\otimes Y)$ in the state $\ket{\Phi^-}$?
%	\[
%	\bra{\Phi^-}-(Y\otimes Y)\ket{\Phi^-} = -1
%	\]}
\end{enumerate}

\item {\bf Another Pseudo-Telepathy Game}\\
Alice and Bob tell their friend Eve that they have a magic $3\times 3$ square of numbers with the following properties:
\begin{itemize}
    \item Every entry is either $1$ or $-1$
    \item The product of each column is $1$
    \item The product of each row is $-1$
\end{itemize}
\begin{enumerate}
\item What is the product of all of the entries of Alice and Bob's square?
%\solopen{The product of all of the entries is not well-defined. This implies that Alice and Bob are lying: no such square exists.
%}
\end{enumerate}
As you may now see, Eve is not convinced by Alice and Bob's claim. Therefore, she asks them to play the following \emph{magic square game}. First, Eve randomly generates two numbers $i,j\in \set{0,1,2}$. She gives $i$ to Alice and $j$ to Bob. Alice and Bob each produce a triple of $\pm 1$ numbers $(a_0,a_1,a_2), (b_0,b_1,b_2)$. They win if $a_0a_1a_2 = 1$, $b_0b_1b_2 = -1$, and $a_j = b_i$. In other words, they win if Alice produces the $i^{\text{th}}$ column of the magic square and Bob produces the $j^{\text{th}}$ row of the magic square.
\begin{enumerate}
\item[2.] Suppose that Alice and Bob use a deterministic strategy in this game. What is the highest success probability they can achieve?
%\solopen{If they win on every question, then they must hold an impossible square. It is possible for them to win on $8$ of $9$ questions as follows:
	%Alice holds a square with every column product equal to $1$. Bob holds a square with every row product equal to $-1$. Alice and Bob's squares disagree in only one location.
%	Their strategy is for Alice to give the $i^{\text{th}}$ column of her square and Bob to give the $j^{\text{th}}$ column of his.
%	For example, we might have
%	\[
%	A= \mathree
%	11{-1}
%	11{-1}
%	111
%	; \quad B = \mathree
%	11{-1}
%	11{-1}
%	11{-1}
%	\]
%	So that Alice and Bob lose only when $i = j = 2$.}
\end{enumerate}
Consider the following $3\times 3$ square of observables:
\begin{equation*}
	\mathree
    {-\id\otimes Z} {X\otimes \id} {X\otimes Z}
	{-Z\otimes \id} {\id\otimes X} {Z\otimes X}
	{ Z\otimes Z} {X\otimes X} {Y\otimes Y}
\end{equation*}
Use the Pauli commutation relations, recalled here, to answer the following two questions.
\begin{equation*}
	X^2 = Y^2 = Z^2 = \id;\quad XYZ = -i\id; \quad XY = - YX, YZ = - ZY, ZX = -XZ.
\end{equation*}
\begin{enumerate}
\item[3.] Which tensor products of Pauli operators commute with $Y\otimes Y$?
%\solopen{An operator commutes with $Y\otimes Y$ if both of its tensor factors anticommute with $Y$. The rest of the operators anticommute with $Y\otimes Y$, since one factor commutes with $Y$ and the other anticommutes. So $X\otimes X$, $X\otimes Z$, $Z\otimes X$, $Z\otimes Z$}
\item[4.] Which tensor products of Pauli operators commute with $\id\otimes X$?
%\solopen{Since everything commutes with $\id$, an operator commutes with $\id\otimes X$ if and only if its second tensor factor commutes with $X$.}
\item[5.] How would you describe the commutation pattern of the square?
%\solopen{Each observable commutes with the ones in the same row or column as it. We already observed this for $\id\otimes X$ and $Y\otimes Y$. It is not hard to do the same computation for the rest of the obersvables.}
\item[6.] The product of the first row of the magic square is $({\id\otimes Z})({ Z\otimes \id})({Z\otimes Z})$. What are the eigenvalues of this operator? Convince yourself that the product of each row is the same as the product of the first row.
%\solopen{The operator is $(Z\otimes Z)^2 =\id\otimes \id$.}
\item[7.] The product of the third column of the magic square is $({X\otimes X})({Y\otimes Y})({Z\otimes Z})$. What are the eigenvalues of this operator? Convince yourself that the product of each column is the same as the product of the third column.
%\solopen{The second operator is $(XYZ)\otimes (XYZ) = (-i)^2\id\otimes \id = - \id\otimes \id$.}
\end{enumerate}
This square of operators gives rise to a quantum strategy for Alice and Bob in the magic square game. Alice and Bob share two EPR states, with each of them holding one qubit of each pair. In other words, their overall state is $\ket\psi = \ket{\Phi_+}_{12}\otimes\ket{\Phi_+}_{34}$. When Eve distributes $(i,j)$, Alice picks the $i^\text{th}$ column of the square, measures the three operators, and returns the three measurements in order. Similarly, Bob measures the three operators in the $j^\text{th}$ row of the square and returns their measurements in order. For example, if $(i,j) = (1,2)$ then Alice will put $(a_0,a_1,a_2)$ equal to the measurement results of
\mbox{$({-\id_1\otimes Z_3}, {-Z_1 \otimes \id_3}, Z_1\otimes  Z_3)$}. Bob will put $(b_0,b_1,b_2)$ equal to the measurement results of
\mbox{$({-Z_2\otimes \id_4},{\id_2 \otimes X_4},Z_2\otimes X_4)$}.
\begin{enumerate}
\item[8.] What does the previous question, along with the previous problem "Commuting Observables are Compatible", tell us about Alice and Bob's output in this game?
%\solopen{$a_1a_2a_3 = 1; b_1b_2b_3 = -1$. The product of the bits Alice outputs is equal to the measurement outcome of the operators that Alice measures. That operator is $\id$, so its outcome is $+1$. Similarly, Bob always outputs a product of $-1$. }
\end{enumerate}

\item {\bf A nonlocal game}\\
In this problem we study an example of a game such that repeating the game twice does not decrease the maximum winning probability. As we will see, this demonstrates an example of a ``coherent attack'' where two independent repetitions can be ``attacked'' better than a single one. 

We begin by describing the game. The referee starts by generating a pair $(s,t) \in \set{(0,0),(0,1),(1,0)}$ uniformly at random. She gives $s$ to Alice and $t$ to Bob. Alice and Bob generate output bits $a,b\in\set{0,1}$, respectively. They win if $a\lor s \neq b \lor t$. (If $a$ and $b$ are bits, then $a\lor b = 0$ if $a$ and $b$ are both $0$ and $a\lor b = 1$ if $a=1$ or $b=1$.)
As a warm-up, consider the strategy in which $a = s$ and $b = t$.
\begin{enumerate}
\item What is the win probability? Which inputs cause Alice and Bob to lose?
%\solopen{They win with probability $\frac 23$; they lose only if $(s,t) = (0,0)$. $a\lor s = a = s$ and $b\lor t = b = t$, so they win if $s \neq t$.}
\end{enumerate}
In the two-parallel-repeated version $G^{(2)}$ of the game we just described, the referee picks two strings $(s_0,t_0),(s_1,t_1)$ from $\set{(0,0),(0,1),(1,0)}$ independently and uniformly at random. She gives $(s_0,s_1)$ to Alice, $(t_0,t_1)$ to Bob, and demands outputs $(a_0,a_1),(b_0,b_1)$ from Alice and Bob. They win if $a_0\lor s_0 \neq b_0 \lor t_0$ and $a_1\lor s_1 \neq b_1 \lor t_1$.
\begin{enumerate}
\item[2.] Which of the following is a valid strategy giving Alice and Bob the highest win probability?
\begin{statements}
    \item $a_0 = s_0, a_1 = t_0; b_0 = s_1, b_1 = t_1$
    \item $a_0 = s_0, a_1 = s_1; b_0 = t_0, b_1 = t_1$
    \item $a_0 = s_0 = a_1; b_0 = t_0 =b_1$
    \item $a_0 = s_1, a_1 = s_0; b_0 = t_1, b_1 = t_0$
\end{statements}
%\solopen{$a_0 = s_1, a_1 = s_0; b_0 = t_1, b_1 = t_0$. They win if $s_1 \lor s_0 \neq t_0 \lor t_1$. This fails if all four values are $0$ (this happens with probability $\frac19$) or if one value on each side is $1$ (this happens with probability $\frac29$). One can check that there are more than three failure cases for the other strategies. Furthermore, the first strategy is not valid because Alice uses one of Bob's input bits and vice versa.}
\end{enumerate}
Suppose Alice and Bob have a valid strategy for the two-parallel-repeated game which wins with probability $\omega_c$.
\begin{enumerate}
\item[3.] Which of these protocols is a valid strategy in the single-repeated game guaranteeing that Alice and Bob win with probability at least $\omega_c$?
\begin{statements}
    \item Alice and Bob receive their inputs $(s,t)$, then run their two-parallel-repeated strategy on the inputs $s_1 = s = s_0, t_1 = t = t_0$, and output $(a_0,b_0)$.
    \item Alice and Bob receive their inputs $(s,t)$, then communicate their bits to each other and run their two-parallel-repeated strategy on the inputs $s_1 = s = t_0, t_1 = t = s_0$. They output $(a_0,b_0)$.
    \item Alice and Bob agree on a shared string $(s_1,t_1)$ uniformly at random from $\set{(0,0),(0,1),(1,0)}$. When they receive their inputs $(s,t)$, they run their two-parallel-repeated strategy on the inputs $((s,t),(s_1,t_1))$, and output $(a_0,b_0)$.
    \item Alice and Bob independently generate random bits $s_1,t_1\in\set{0,1}$. When they receive their inputs $(s,t)$, they run their two-parallel-repeated strategy on the inputs $((s,t),(s_1,t_1))$, and output $(a_0,b_0)$.
\end{statements}
%\solopen{They simulate exactly the distribution of inputs in the two-parallel repeated game, so they'll produce winning outputs for the two-parallel-repeated game with probability $\omega_c$. These outputs are also winning for the single-repeated game that they're actually playing.
%}
\end{enumerate}
This proves that the optimal success probability in the one-shot game is an upper bound for the optimal success probability in the two-parallel game.

Now we will find an upper bound on the success probability of the single-repeated game, assuming that Alice and Bob may use shared entanglement in addition to classical resources.

The most general strategy that Alice and Bob can take is as follows. They each have two $\pm1$-eigenvalue-observables $A_0, A_1, B_0, B_1$. They share an entangled state $\ket\psi$. Alice measures her share of $\ket\psi$ using $A_s$, Bob measures his share using $B_t$, and they each output $0$ if they measured a $1$ and $1$ if they measured a $-1$.

In general, if $X$ is an observable, then $\bra\psi X \ket \psi$ is equal to the probability of measuring a $1$ minus the probability of measuring $-1$. In other words, the probability of measuring a $1$ is $p_1 = \frac12 + \frac12\bra\psi X \ket \psi$.
\begin{enumerate}
\item[4.] For which of the following $M$ is the probability that Alice and Bob win the game equal to $\frac 12 + \frac12 \bra\psi M \ket \psi$?
\emph{(Hint: consider the three possible inputs $(s,t)$ separately. What must Alice and Bob's measurements be in each case to guarantee victory?)}
\begin{statements}
    \item $M = -\frac13 A_0 \otimes B_0 + \frac13 A_0\otimes \Id + \frac 13\Id \otimes B_0$
    \item $M = -\frac13 A_0\otimes B_0 + \frac13 A_0\otimes B_1 + \frac 13A_1\otimes B_0$
    \item $M = \frac13 A_0\otimes B_0$
    \item $M = -\frac13 (A_0\otimes \Id + \Id\otimes B_0) + \frac13 (A_0\otimes \Id + A_0\otimes B_1) + \frac 13(\Id\otimes B_0 + A_1\otimes B_0)$
\end{statements}
%\solopen{If $(s,t) = (0,0)$, then Alice and Bob need to produce different measurement outcomes in order to win. That is, the measurement outcome of $A_0\otimes B_0$ must be $-1$. If $(s,t) = (0,1)$, then $b\lor t = 1$ regardless of what Bob says. They win iff Alice says $0$, i.e.\ the measurement outcome of $A_0$ must be $1$. Similarly, if $(s,t) = (1,0)$, they win iff the measurement outcome of $B_0$ is $1$.
%}
\end{enumerate}
This quantity $\bra \psi M \ket \psi$ is bounded above by the maximum eigenvalue of $M$. With a bit of arithmetic, we can find the eigenvalues of $M$ exactly, despite our ignorance about Alice and Bob's observables!
\begin{enumerate}
\item[5.] Which of the following equations is satisfied by $M$?
\begin{statements}
    \item $M^2 = \frac13\mathbb I - \frac23 M$
    \item $M^2 = \frac19\mathbb I - \frac79 M$
    \item $M^2 = \frac13\mathbb I - \frac13 M$
    \item $M^2 = \id - 2M$
\end{statements}
%\solopen{$A_0$, $B_0$ square to identity, so $A_0\otimes B_0$ also squares to identity. 
%	Squaring our expression from the last question gives
%	\[
%	M^2 = \frac19(
%	3\id\otimes \id - 2\id\otimes B_0 -2A_0\otimes \id +2A_0\otimes B_0
%	) = \frac13(\id - 2M)
%	\]}
\item[6.] The answer to the last question gives the characteristic polynomial of $M$ (indeed, it is the unique monic quadratic satisfied by $M$). Use it to solve for the largest eigenvalue $\lambda_\text{max}$ of $M$.
%\solopen{The characterstic polynomial $\lambda^2 + \frac23 \l - \frac13 = 0$ has roots $\l = \frac13, -1$. Therefore its largest eigenvalue is $\frac 13$.
%}
\item[7.] Now use the facts that $p_{\text{win}} \leq \frac 12 + \frac12 \bra\psi M \ket \psi$ and $\bra\psi M \ket \psi \leq \lambda_\text{max}$ to find the an upper bound on $p_\text{win}$.
%\solopen{The upper bound is $\frac12 + \frac12\frac13$. Notice that this matches the lower bound from earlier in this problem.
%}
\end{enumerate}

\end{exercises}
