\newcommand{\GC}{\ensuremath{\textsc{CLONE}}}


\chapter{}


\begin{exercises}

\item {\bf Composing quantum maps}\label{ex:map-composition}\\
 Show carefully that the composition of two quantum maps according to Definition~\ref{def:q-map} is still a quantum map, i.e.\ a sequence of (i), (ii), (iii) as in the definition repeated twice can be ``re-ordered'' so that all the ancilla preparation come first and all the tracing out comes last, without changing how the map operates on any quantum state. 

\item {\bf Measurement attacks}\\
Consider all cloning attacks that take the following form. The adversary decides on an arbitrary orthonormal basis $(\ket{u_0},\ket{u_1})$ for the single-qubit space $\Complex^2$. It then measures the challenger's state $\ket{\psi_\$}$ in the basis $(\ket{u_0},\ket{u_1})$ to obtain an outcome $b\in\{0,1\}$. Finally, the adversary returns the density matrix $\rho = \proj{u_b}\otimes \proj{u_b}$. 
\begin{enumerate}
\item Express the success probability of this attack as a function of the coefficients $\alpha,\beta$ of $\ket{u_0} = \alpha \ket{0} + \beta\ket{1}$. (Since $\ket{u_1}$ is orthogonal to $\ket{u_0}$, without loss of generality $\ket{u_1} = \beta\ket{0} - \alpha \ket{1}$.) 
\item Find the choice of $\alpha,\beta$ that maximizes the success probability (don't forget about complex numbers!). 
\item Did you find an attack that is better than the ones considered in Section~\ref{sec:pm-attack}? 
\end{enumerate}

\item {\bf A cloning map}\label{ex:2-qmamp}\\
In this exercise we verify that the map $T$ defined in Example~\ref{ex:opt-cl} is a valid quantum map. 
\begin{enumerate}
\item Start with a simple ``sanity check'': verify that each of the four matrices $\rho_0$, $\rho_1$, $\rho_+$ and $\rho_-$ is a valid density matrix. 
\end{enumerate}
However, this is not enough: for example, these conditions are satisfied by the ``optimal cloning map'' $\proj{\psi_\$}\mapsto \proj{\psi_\$}\otimes \proj{\psi_\$}$, but we know that there exists no such map! To see that $T$ is a well-defined map, we verify that it can be implemented by (i) adding two ancilla qubits in state $\ket{00}_{BC}$, (ii) a unitary transformation, and (iii) a tracing-out operation. 
\begin{enumerate}
\item[2.] Consider the following map $V$, where as usual $\ket{\epr}$ is an EPR pair.  
\begin{align*}
\ket{0}_A \ket{00}_{BC} \mapsto \frac{2}{\sqrt{6}}\ket{00}_{AB}\ket{0}_C + \frac{1}{\sqrt{3}} \ket{\epr}_{AB} \ket{1}_C \;,\\
\ket{1}_A \ket{00}_{BC} \mapsto  \frac{1}{\sqrt{3}} \ket{\epr}_{AB}\ket{0}_C + \frac{2}{\sqrt{6}}\ket{11}_{AB}\ket{1}_C \;.
\end{align*}
Check that  the two states on the right-hand side, call them $\ket{v_0}$ and $\ket{v_1}$, are orthonormal. Using Remark~\ref{rk:unitary-extend} this shows that it is possible to extend $V$ to a valid unitary operation on the entire $3$-qubit space $\Complex^8$. 
\item[3.] Show that the map $T$ is identical to the composition of adding two ancilla qubits in state $\ket{00}_{BC}$, applying $V$, and tracing out the third qubit. That is, for all states $\ket{\psi}$,
\[T(\proj{\psi}_A) = \Tr_C \big( V\big( \proj{\psi}_A \otimes \proj{00}_{BC} \big) V^\dagger \big) \;.\]
\end{enumerate}
This justifies that $T$ is a valid quantum map, because it can be written as a sequence of three valid operations.


\item {\bf An optimal attack}\label{ex:opt-wiesner}\\
Let 
\[ N_1 = \frac{1}{\sqrt{12}} \begin{pmatrix} 3 & 0 \\ 0 & 1 \\ 0 & 1 \\ 1 & 0 \end{pmatrix} \quad \text{and}\quad N_2 = \frac{1}{\sqrt{12}} \begin{pmatrix} 0 & 1 \\ 1 & 0 \\ 1 & 0 \\ 0 & 3\end{pmatrix}\;.\]
\begin{enumerate}
\item Show that $(N_1,N_2)$ are valid Kraus operators in the definition of a quantum channel $\mathcal{N}(\rho)=N_1\rho N_1^\dagger + N_2 \rho N_2^\dagger$ mapping one qubit to two qubits. 
\item Show that using $\mathcal{N}$, a quantum adversary succeeds in the game $\GC$ for Wiesner's quantum money scheme with probability $\frac{3}{4}$. 
\end{enumerate}
\end{exercises}
