% !TeX root = ../main_only_solutions.tex

\chapter{Quantum Cryptography Beyond Key Distribution}

\begin{exercises}


\item {\bf A Weak Coin Flipping Protocol}
\begin{enumerate}
\item We see from the description of steps 3 and 4 that when Alice and Bob are honest, they always both return the same outcome $c=a\oplus b$. Since $a$ and $b$ are both chosen uniformly at random, $c$ is also uniformly random. Therefore, the protocol is correct. 

\item Bob's reduced density matrix after step 1 is 
\begin{align*}
\rho_a &= \Tr_A(\ket{\psi_{a}}\bra{\psi_a}) = \frac12 \Tr_A \big((\ket{0} \ket{\psi_{a,0}} + \ket{1} \ket{\psi_{a,1}} )(\bra{0}\bra{\psi_{a,0}}+\bra{1}\bra{\psi_{a,1}})\big) \\
&= \frac12 (\ket{\psi_{a,0}}\bra{\psi_{a,0}} + \ket{\psi_{a,1}}\bra{\psi_{a,1}})\;.
\end{align*}
Simplifying the latter gives
\begin{equation*}
\rho_a = \cos^2(\frac{\alpha}{2}) \proj{0}  +\sin^2(\frac{\alpha}{2}) \proj{a+1}\;.
\end{equation*}

\item Recalling the interpretation of the fidelity as the square root of the probability that Alice can convince Bob that a state is another. Hence the probability that Alice wins given that Bob sent $b$ is precisely $F^2(\sigma_b, \ket{\psi_b}\bra{\psi_b})$, and this can be upper bounded (tracing out the qubit system) by $F^2(\sigma, \rho_b)$.

\item
\begin{align*}
	\Pr(\mbox{Alice wins}) &= \frac12 \big(\Pr(\mbox{Alice wins} | b = 0)+ \Pr(\mbox{Alice wins} | b = 1) \big) \\
	&\leq \frac12 \big(F^2(\sigma, \rho_0)+F^2(\sigma, \rho_1)\big)\;.
\end{align*}
Using the fact from the hint we get
\begin{equation*}
\Pr\big(\mbox{Alice wins}\big) \leq \frac12\big(1+F(\rho_0, \rho_1)\big)\;.
\end{equation*}
Finally, we can calculate $F(\rho_0, \rho_1) = \|\sqrt{\rho_0}\sqrt{\rho_1} \|_{tr} = \cos^2(\frac{\alpha}{2})$, which gives us the desired bound.

\item The state possessed by Alice after Bob has applied $U$ and returned his qutrit to Alice is
\begin{equation*}
\id\otimes U \ket{\psi_a} = \id\otimes U \frac{1}{\sqrt{2}}(\ket{0} \ket{\psi_{a,0}} + \ket{1} \ket{\psi_{a,1}} )\;,
\end{equation*}
which is
\begin{equation*}
\frac{1}{\sqrt{2}} \Big( \ket{0}\Big(\cos\frac{\alpha}{2} \ket{\xi_{0, \bar{a}}} + \sin\frac{\alpha}{2} \ket{\xi_{a+1, \bar{a}}}\Big) + \ket{1}\Big(\cos\frac{\alpha}{2} \ket{\xi_{0, \bar{a}}} - \sin\frac{\alpha}{2} \ket{\xi_{a+1, \bar{a}}}\Big)\Big)\;.
\end{equation*}
The probability of Bob winning, given that Alice has picked $a$, is the modulus squared of the overlap between the honest state $\ket{\psi_a}$ expected by Alice and the state after Bob's unitary, that is
\begin{align*}
\Big|\bra{\psi_a} \otimes \id  &\cdot \frac{1}{\sqrt{2}} \Big( \ket{0}\Big(\cos\frac{\alpha}{2} \ket{\xi_{0, \bar{a}}} + \sin\frac{\alpha}{2} \ket{\xi_{a+1, \bar{a}}}\Big)\\
& \qquad\qquad+ \ket{1}\Big(\cos\frac{\alpha}{2} \ket{\xi_{0, \bar{a}}} - \sin\frac{\alpha}{2} \ket{\xi_{a+1, \bar{a}}}\Big)\Big) \Big|^2 \\
&= \frac14 \Big|\bra{\psi_{a,0}}\otimes I \Big(\cos\frac{\alpha}{2} \ket{\xi_{0, \bar{a}}} + \sin\frac{\alpha}{2} \ket{\xi_{a+1, \bar{a}}}\Big) \\
&\qquad\qquad+ \bra{\psi_{a,1}} \otimes I \Big(\cos\frac{\alpha}{2} \ket{\xi_{0, \bar{a}}} - \sin\frac{\alpha}{2} \ket{\xi_{a+1, \bar{a}}}\Big) \Big|^2\;.
\end{align*}
Substituting the definitions of $\ket{\psi_{a,0}}$ and $\ket{\psi_{a,1}}$ gives, after simplification,
\begin{equation*}
\Pr(\mbox{Bob wins $|$ Alice sent $a$} ) =| \cos^2\frac{\alpha}{2} \bra{0}\otimes I \ket{\xi_{0,\bar{a}}} + \sin^2\frac{\alpha}{2} \bra{a+1} \otimes I \ket{\xi_{a+1,\bar{a}}}|^2\;.
\end{equation*}

\item Yes, it should be!

\item You are told that 
\[\Pr(\mbox{Bob wins} | \mbox{Alice picked $a$}) \leq \Big(\cos^2(\frac{\alpha}{2}) \|\ket{\xi_{0,\bar{a}}}\| + \sin^2(\frac{\alpha}{2})\Big)^2\;.\]
You can bound $\Pr(\mbox{Bob wins})$ by averaging the latter bound over $a \in \{0,1\}$. This is maximized when $\|\ket{\xi_{0,0}}\| = \|\ket{\xi_{0,1}}\| = \frac{1}{\sqrt{2}}$ (recall that $\|\ket{\xi_{0,0}}\|^2 + \|\ket{\xi_{0,1}}\|^2 = 1$). \\
Thus, $\Pr(\mbox{Bob wins})$ is bounded by $\Big( \frac{1}{\sqrt{2}}\cos^2(\frac{\alpha}{2}) +\sin^2(\frac{\alpha}{2})\Big)^2$.

\item The bias is minimized by choosing $\alpha$ that makes Alice and Bob's probabilities of dishonestly winning equal. That is, from the tight bounds found earlier, $\alpha$ such that
\begin{equation*}
\frac12(1+\cos^2\frac{\alpha}{2}) = \Big(\frac{1}{\sqrt{2}}\cos^2\frac{\alpha}{2} + \sin^2\frac{\alpha}{2}\Big)^2\;.
\end{equation*}
Solving for $\alpha$ makes the two sides equal to $0.739$, i.e. no player can win with probability greater than $0.739$. Thus the bias is $0.239$.
\end{enumerate}

\item {\bf A Simple Quantum Bit Commitment Protocol}
\begin{enumerate}
\item $\rho_b = \Tr_1(\ket{\psi_b}\bra{\psi_b}) = \alpha \proj{b} + (1-\alpha) \proj{2}$.

\item Recall that the optimal probability with which Bob can distinguish between two states $\rho$ and $\sigma$ is $\frac12 + \frac14 \|\rho-\sigma\|_{tr}$.
In our protocol, Bob's cheating probability is the optimal probability with which he can distinguish between Alice commiting to $0$ or to $1$, i.e. between $\rho_0$ and $\rho_1$. 
So 
\[ P_B^{*} = \frac12 + \frac14 \|\rho_0 - \rho_1\| = \frac12 + \frac{\alpha}{4} \| \ket{0}\bra{0}- \ket{1}\bra{1}\|_{tr} = \frac12 + \frac{\alpha}{2}\;.\]

\item The probability that Alice successfully opens bit $b$ equals the probability that she passes Bob's check for $b$ at the end of the protocol, i.e. that Bob measures his part of the joint state and gets the honest $\ket{\psi_b}$ as the outcome:
\begin{align*}
\Pr(\mbox{Alice successfully opens $b$}) &= \sum_{i} p_i |\braket{\psi_b}{\tilde{\psi_{i,b}}}|^2 \\
&= F^2(\sigma_b, \ket{\psi_b}\bra{\psi_b})\;.
\end{align*}
Now, tracing out the system $\mathcal{H}_s$, and using the fact that the fidelity is non-decreasing under taking partial trace, we have
\begin{align*}
\Pr(\mbox{Alice successfully opens $b$}) &\leq F^2\Big(Tr_{\mathcal{H}_s}(\sigma_b), \Tr_{\mathcal{H}_s}(\ket{\psi_b}\bra{\psi_b})\Big)\\
& = F^2(\sigma, \rho_b)\;.
\end{align*}
Hence
\begin{equation*}
P_A^* \leq \frac{1}{2} \big( F^2(\sigma, \rho_0) + F^2(\sigma, \rho_1) \big)\;.
\end{equation*}

\item From the previous problem we have that $P_A^* \leq \frac12 \Big(F^2(\sigma, \rho_0) + F^2(\sigma, \rho_1)\Big)$.\\
Applying the bound in the hint, we have $P_A^* \leq \frac12 \Big(1 + F(\rho_0, \rho_1)\Big)$. \\
Now, one can compute $F(\rho_0, \rho_1) = 1-\alpha$. So, substituting gives $P_A^* \leq 1-\frac{\alpha}{2}$.

\item When solving question 3, you found that 
\[\Pr(\mbox{Alice successfully opens b}) = F^2(\sigma_b, \proj{\psi_b})\;.\]
Now, when Alice dishonestly prepares the state $\ket{\psi_0} + \ket{\psi_1}$ normalized, the fidelity is between two pure states. Thus,
\begin{equation*}
\Pr(\mbox{Alice successfully opens b}) = \Big|(\bra{\psi_0}+\bra{\psi_1}) \ket{\psi_b}\Big|^2/\Big\|\ket{\psi_0}+\ket{\psi_1} \Big\|^2\;.
\end{equation*}
Now, one can easily compute $\|\ket{\psi_0}+\ket{\psi_1} \|^2 = 2(2-\alpha)$, and $|\bra{\psi_0}+\bra{\psi_1} \ket{\psi_b}|^2 = (2-\alpha)^2$.
Clearly, by symmetry 
\[\Pr(\mbox{Alice successfully opens 0}) = \Pr(\mbox{Alice successfully opens 1})\;.\] Hence,
\begin{equation}
P_A^* = \frac{(2-\alpha)^2}{2(2-\alpha)} = 1-\frac{\alpha}{2}\;,
\end{equation}
which achieves the upper bound.
With similar calculations, one can check that options II and III do not achieve the upper bound.

\item To minimize the cheating probability, one just needs to pick $\alpha$ such that $P_A^* = P_B^*$, i.e. $\frac12(1+\alpha) = 1-\frac{\alpha}{2}$.\\
Solving for $\alpha$ gives $\alpha = \frac12$, which implies $P_A^* = P_B^* = \frac34$. And the latter is the cheating probability.
\end{enumerate}


\item {\bf From Coin Flipping to Bit Commitment}
\begin{enumerate}
\item Because it decreases the trace distance between the two possible commitments that Alice can send to Bob during the commit phase. Increasing the amplitude of the term $\ket{22}$ decreases the trace distance between $\ket{\psi_0}$ and $\ket{\psi_1}$, making it harder for Bob to distinguish between the two.

\item At the end of the commit phase, depending on Alice's committed bit $b$, the joint state is
\begin{equation*}
\ket{\Omega_b^*} =  \sqrt{p'} \ket{L,b}_A \otimes \ket{ L,b, G'_L}_B + \sqrt{1-p'} \ket{W,2}_A \otimes \ket{ W,2, G'_W}_B\;.
\end{equation*}
Bob's reduced density matrix is
\begin{equation*}
\sigma_b^* = p' \ket{b,G'_L}\bra{b,G'_L} + (1-p')\ket{2,G'_W}\bra{2,G'_W}\;.
\end{equation*}
So,
\begin{align*}
P_B^* &= \Pr(\mbox{Bob guesses $b$})\\
& = \frac12 + \frac{D(\sigma_0,\sigma_1)}{2} = \frac12 + \frac{1+p'}{2} \leq \frac{1+p}{2}+\frac{\epsilon}{2}\;,
\end{align*}
since $p' \leq p+\epsilon$ because the weak coin flipping protocol has $\epsilon$-bias. (As usual, $D(\sigma_0,\sigma_1)$ is the trace-distance.)

\item Condition II is true because the weak coin flipping protocol has $\epsilon$-bias. Condition III is a normalization condition. Condition I doesn't necessarily hold because in a weak coin flipping protocol Alice must not be able to bias in her favour only her winning outcome, not the losing outcome ($\ket{\bar{0}}$ and $\ket{\bar{1}}$ correspond to $L$.)

\item Let $\mathcal{M}$ be the operation that performs a measurement with respect to an orthonormal basis containing $\ket{\bar{0}}, \ket{\bar{1}}$ and $\ket{\bar{2}}$. 
Then
\begin{align*}
	M(\rho_b) &= \rho_b\;, \\
	M(\sigma) &= \sum_{i=0}^2 \bra{\bar{i}}\sigma \ket{\bar{i}}\ket{\bar{i}}\bra{\bar{i}} +
	\begin{pmatrix}
		\mbox{terms of same form involving the}\\
		\mbox{other elements of the measurement basis}\\
	\end{pmatrix}
\end{align*}

So,
\begin{equation*}
F(\sigma, \rho_b) \leq F(\mathcal{M}(\sigma), \mathcal{M}(\rho_b)) = \Big\|\sqrt{\mathcal{M}(\rho_b)} \sqrt{\mathcal{M}(\sigma)} \Big\|_{tr}\;.
\end{equation*}
The latter is equal to (since $\mathcal{M}(\rho_b)$ and $\mathcal{M}(\sigma)$ are already diagonal):
\begin{equation*}
\Big\|\big(\sqrt{p} \ket{\bar{b}}\bra{\bar{b}} + \sqrt{1-p} \ket{2}\bra{2}\big)\big(\sum_{i=0}^2 \sqrt{r_i}\ket{\bar{i}}\bra{\bar{i}} + (\mbox{other terms that vanish in the product}) \big) \Big\|_{tr}\;.
\end{equation*}
And one can easily see that the latter equals $\sqrt{pr_b}+\sqrt{(1-p)r_2}$. 
Hence,
\begin{equation*}
F(\sigma, \rho_b) \leq \sqrt{pr_b}+\sqrt{(1-p)r_2}\;.
\end{equation*}

\item As the hint suggests, it is the case that $P_A^* \leq \frac{1}{2} \big( F^2(\sigma, \rho_0) + F^2(\sigma, \rho_1) \big)$. Plugging in the bound we just derived gives
\begin{equation*}
P_A^* \leq \frac12 \Big( \big(\sqrt{p r_0}+\sqrt{(1-p)r_2}\big)^2 + \big(\sqrt{p r_1}+\sqrt{(1-p)r_2}\big)^2 \Big)\;.
\end{equation*}

\item We are told that for $\epsilon < p(1-\frac{1}{2-p})$, the bound is maximal when $r_2$ is maximal, i.e. when $r_2 = 1-p+\epsilon$, given the constraint from Problem 3. By the symmetry in $r_0$ and $r_1$ we deduce that, at the maximum, $r_0 = r_1$.\\
Recalling the normalization constraint $r_0 + r_1 +r_2 \leq 1$, we deduce that the maximum for the bound is then achieved at $r_0 = r_1 =\frac{p-\epsilon}{2}$.\\
Hence the maximum value of the bound is $\Big(\sqrt{p (\frac{p-\epsilon}{2})} + \sqrt{(1-p)(1-p+\epsilon)}\Big)^2$.

\item Since the bounds we found for $P_A^*$ and $P_B^*$ in the previous problems are tight, we have
$P_B^* = \frac{1+p}{2} + \frac{\epsilon}{2}$ and $P_A^* = \Big(\sqrt{p (\frac{p-\epsilon}{2})} + \sqrt{(1-p)(1-p+\epsilon)}\Big)^2 $. And ignoring $O(\epsilon)$ terms, $P_A^* = \Big(1-\big(1-\frac{1}{\sqrt{2}} \big)p \Big)$ and $P_B^* = \frac{1+p}{2}$. \\
As usual, in order to minimize the overall cheating probability, we need to set $P_A^* = P_B^*$ and solve for $p$. So,
\begin{equation*}
\Big(1-\big(1-\frac{1}{\sqrt{2}} \big)p \Big) = \frac{1+p}{2}\;.
\end{equation*}
Solving for $p$ gives $P_A^* = P_B^* \approx 0.739$, which is the overall cheating probability of the protocol.
\end{enumerate}
\end{exercises}