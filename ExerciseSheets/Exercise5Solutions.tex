\documentclass[12pt]{article}
\usepackage{fullpage}
\usepackage{amssymb,amsmath}

\newtheorem{theorem}{Theorem}
\newcommand{\pguess}{P_{\rm guess}}

 \newcommand{\Header}[1]{\begin{center} {\Large\bf #1} \end{center}}
 \newcommand{\header}[1]{\begin{center} {\large\bf #1} \end{center}}
\setlength{\parindent}{0.0in}
\setlength{\parskip}{1ex}


%\newif\ifnotes\notestrue
\newif\ifnotes\notesfalse


\usepackage{amsmath,amssymb,amsthm,amsfonts,latexsym,bbm,xspace,graphicx,float,mathtools,epigraph}
\usepackage[backref,colorlinks,citecolor=blue,bookmarks=true]{hyperref}
\usepackage{enumitem,manyfoot,fullpage}
\usepackage{subfig,tikz,framed}
\usepackage{endnotes}
\usepackage{braket}


\usepackage{fullpage}
\usepackage{hyperref}
\usepackage{pdfsync}
\usepackage{microtype}
\usepackage{color}
\usepackage{cleveref}

\newtheorem*{namedtheorem}{\theoremname}
\newcommand{\theoremname}{testing}
\newenvironment{named}[1]{ \renewcommand{\theoremname}{#1} \begin{namedtheorem}} {\end{namedtheorem}}
\newtheorem{lemma}[theorem]{Lemma}
\newtheorem{claim}[theorem]{Claim}
\newtheorem{proposition}[theorem]{Proposition}
\newtheorem{fact}[theorem]{Fact}
\newtheorem{corollary}[theorem]{Corollary}

\theoremstyle{definition}
\newtheorem{definition}[theorem]{Definition}
\newtheorem{remark}[theorem]{Remark}
\newtheorem{observation}[theorem]{Observation}
\newtheorem{notation}[theorem]{Notation}
\newtheorem{example}[theorem]{Example}
\newtheorem{examples}[theorem]{Examples}
\newtheorem{exercise}{Exercise}


\newenvironment{quotenote}{
\begin{quote}
  \footnotesize
\noindent{\bf Note:}}
{\end{quote}
}


% probability and other mathops
\renewcommand{\Pr}{\mathop{\bf Pr\/}}
\newcommand{\E}{\mathop{\bf E\/}}
\newcommand{\Ex}{\mathop{\bf E\/}}
\newcommand{\Var}{\mathop{\bf Var\/}}
\newcommand{\Cov}{\mathop{\bf Cov\/}}
\newcommand{\stddev}{\mathop{\bf stddev\/}}
\newcommand{\littlesum}{\mathop{{\textstyle \sum}}}
\newcommand{\apx}{\mathop{\approx}}

\newcommand{\epr}{\textsc{EPR}}

\newcommand{\Zt}{\ensuremath{\Z_t}}
\newcommand{\Zp}{\ensuremath{\Z_p}}
\newcommand{\Zq}{\ensuremath{\Z_q}}
\newcommand{\ZN}{\ensuremath{\Z_N}}
\newcommand{\Zps}{\ensuremath{\Z_p^*}}
\newcommand{\ZNs}{\ensuremath{\Z_N^*}}
\newcommand{\JN}{\ensuremath{\J_N}}
\newcommand{\QR}{\ensuremath{\mathbb{QR}}}
\newcommand{\QRN}{\ensuremath{\QR_{N}}}
\newcommand{\QRp}{\ensuremath{\QR_{p}}}

% mathrm terms
\newcommand{\poly}{\mathrm{poly}}
\newcommand{\negl}{\mathrm{negl}}
\newcommand{\Tr}{\mathrm{Tr}}
\newcommand{\polylog}{\mathrm{polylog}}
\newcommand{\size}{\mathrm{size}}
\newcommand{\avg}{\mathop{\mathrm{avg}}}
\newcommand{\sgn}{\mathrm{sgn}}
\newcommand{\dist}{\mathrm{dist}}
\newcommand{\spn}{\mathrm{span}}
\newcommand{\supp}{\mathrm{supp}}
\newcommand{\Val}{\mathrm{Val}}
\newcommand{\Opt}{\mathrm{Opt}}
\newcommand{\LPOpt}{\mathrm{LPOpt}}
\newcommand{\SDPOpt}{\mathrm{SDPOpt}}
\newcommand{\vol}{\mathrm{vol}}
\newcommand{\Id}{\mathbb{I}}

% number systems
\newcommand{\R}{\mathbbm R}
\newcommand{\C}{\mathbbm C}
\newcommand{\N}{\mathbbm N}
\newcommand{\Z}{\mathbbm Z}
\newcommand{\F}{\mathbbm F}
\newcommand{\Q}{\mathbbm Q}

\newcommand{\mH}{\mathcal{H}}

% complexity classes
\newcommand{\PTIME}{\mathsf{P}}
\newcommand{\NP}{\mathsf{NP}} \newcommand{\np}{\NP}

% short forms
\newcommand{\eps}{\varepsilon}
\newcommand{\lam}{\lambda}
\newcommand{\vphi}{\varphi}
\newcommand{\la}{\langle}
\newcommand{\ra}{\rangle}
\newcommand{\wt}[1]{\widetilde{#1}}
\newcommand{\wh}[1]{\widehat{#1}}
\newcommand{\ul}[1]{\underline{#1}}
\newcommand{\ol}[1]{\overline{#1}}
\newcommand{\ot}{\otimes}
\newcommand{\Ra}{\Rightarrow}
\newcommand{\half}{\tfrac{1}{2}}
\newcommand{\grad}{\nabla}
\newcommand{\sse}{\subseteq}


% calligraphic letters
\newcommand{\calA}{\mathcal{A}}
\newcommand{\calB}{\mathcal{B}}
\newcommand{\calC}{\mathcal{C}}
\newcommand{\calD}{\mathcal{D}}
\newcommand{\calE}{\mathcal{E}}
\newcommand{\calF}{\mathcal{F}}
\newcommand{\calG}{\mathcal{G}}
\newcommand{\calH}{\mathcal{G}}
\newcommand{\calI}{\mathcal{I}}
\newcommand{\calJ}{\mathcal{J}}
\newcommand{\calK}{\mathcal{K}}
\newcommand{\calL}{\mathcal{L}}
\newcommand{\calM}{\mathcal{M}}
\newcommand{\calN}{\mathcal{N}}
\newcommand{\calO}{\mathcal{O}}
\newcommand{\calP}{\mathcal{P}}
\newcommand{\calQ}{\mathcal{Q}}
\newcommand{\calR}{\mathcal{R}}
\newcommand{\calS}{\mathcal{S}}
\newcommand{\calT}{\mathcal{T}}
\newcommand{\calU}{\mathcal{U}}
\newcommand{\calV}{\mathcal{V}}
\newcommand{\calW}{\mathcal{W}}
\newcommand{\calX}{\mathcal{X}}
\newcommand{\calY}{\mathcal{Y}}
\newcommand{\calZ}{\mathcal{Z}}


\newcommand{\myfig}[4]{\begin{figure}[H] \begin{center} \includegraphics[width=#1\textwidth]{#2} \caption{#3} \label{#4} \end{center} \end{figure}} 

\newcommand{\bit}{\ensuremath{\{0,1\}}}

%%% CRYPTO-RELATED NOTATION

% length of a string
\newcommand{\len}[1]{\lvert{#1}\rvert}
\newcommand{\lenfit}[1]{\left\lvert{#1}\right\rvert}
% length of some vector, element
\newcommand{\length}[1]{\lVert{#1}\rVert}
\newcommand{\lengthfit}[1]{\left\lVert{#1}\right\rVert}


% types of indistinguishability
\newcommand{\compind}{\ensuremath{\stackrel{c}{\approx}}}
\newcommand{\statind}{\ensuremath{\stackrel{s}{\approx}}}
\newcommand{\perfind}{\ensuremath{\equiv}}

% font for general-purpose algorithms
\newcommand{\algo}[1]{\ensuremath{\mathsf{#1}}}
% font for general-purpose computational problems
\newcommand{\problem}[1]{\ensuremath{\mathsf{#1}}}
% font for complexity classes
\newcommand{\class}[1]{\ensuremath{\mathsf{#1}}}


% KEYS AND RELATED

\newcommand{\key}[1]{\ensuremath{#1}}

\newcommand{\pk}{\key{pk}}
\newcommand{\vk}{\key{vk}}
\newcommand{\sk}{\key{sk}}
\newcommand{\mpk}{\key{mpk}}
\newcommand{\msk}{\key{msk}}
\newcommand{\fk}{\key{fk}}
\newcommand{\id}{id}
\newcommand{\keyspace}{\ensuremath{\mathcal{K}}}
\newcommand{\msgspace}{\ensuremath{\mathcal{M}}}
\newcommand{\ctspace}{\ensuremath{\mathcal{C}}}
\newcommand{\tagspace}{\ensuremath{\mathcal{T}}}
\newcommand{\idspace}{\ensuremath{\mathcal{ID}}}

\newcommand{\concat}{\ensuremath{\|}}

% GAMES

% advantage
\newcommand{\advan}{\ensuremath{\mathbf{Adv}}}

% different attack models
\newcommand{\attack}[1]{\ensuremath{\text{#1}}}

\newcommand{\atk}{\attack{atk}} % dummy attack
\newcommand{\indcpa}{\attack{ind-cpa}}
\newcommand{\indcca}{\attack{ind-cca}}
\newcommand{\anocpa}{\attack{ano-cpa}} % anonymous
\newcommand{\anocca}{\attack{ano-cca}}
\newcommand{\euacma}{\attack{eu-acma}} % forgery: adaptive chosen-message
\newcommand{\euscma}{\attack{eu-scma}} % forgery: static chosen-message
\newcommand{\suacma}{\attack{su-acma}} % strongly unforgeable

% ADVERSARIES
\newcommand{\attacker}[1]{\ensuremath{\mathcal{#1}}}

\newcommand{\Adv}{\attacker{A}}
\newcommand{\AdvA}{\attacker{A}}
\newcommand{\AdvB}{\attacker{B}}
\newcommand{\Dist}{\attacker{D}}
\newcommand{\Sim}{\attacker{S}}
\newcommand{\Ora}{\attacker{O}}
\newcommand{\Inv}{\attacker{I}}
\newcommand{\For}{\attacker{F}}

% CRYPTO SCHEMES

\newcommand{\scheme}[1]{\ensuremath{\text{#1}}}

% pseudorandom stuff
\newcommand{\prg}{\algo{PRG}}
\newcommand{\prf}{\algo{PRF}}
\newcommand{\prp}{\algo{PRP}}

% symmetric-key cryptosystem
\newcommand{\skc}{\scheme{SKC}}
\newcommand{\skcgen}{\algo{Gen}}
\newcommand{\skcenc}{\algo{Enc}}
\newcommand{\skcdec}{\algo{Dec}}

% public-key cryptosystem
\newcommand{\pkc}{\scheme{PKC}}
\newcommand{\pkcgen}{\algo{Gen}}
\newcommand{\pkcenc}{\algo{Enc}} % can also use \kemenc and \kemdec
\newcommand{\pkcdec}{\algo{Dec}}

% digital signatures
\newcommand{\sig}{\scheme{SIG}}
\newcommand{\siggen}{\algo{Gen}}
\newcommand{\sigsign}{\algo{Sign}}
\newcommand{\sigver}{\algo{Ver}}

% message authentication code
\newcommand{\mac}{\scheme{MAC}}
\newcommand{\macgen}{\algo{Gen}}
\newcommand{\mactag}{\algo{Tag}}
\newcommand{\macver}{\algo{Ver}}

% key-encapsulation mechanism
\newcommand{\kem}{\scheme{KEM}}
\newcommand{\kemgen}{\algo{Gen}}
\newcommand{\kemenc}{\algo{Encaps}}
\newcommand{\kemdec}{\algo{Decaps}}

% identity-based encryption
\newcommand{\ibe}{\scheme{IBE}}
\newcommand{\ibesetup}{\algo{Setup}}
\newcommand{\ibeext}{\algo{Ext}}
\newcommand{\ibeenc}{\algo{Enc}}
\newcommand{\ibedec}{\algo{Dec}}

% hierarchical IBE (as key encapsulation)
\newcommand{\hibe}{\scheme{HIBE}}
\newcommand{\hibesetup}{\algo{Setup}}
\newcommand{\hibeext}{\algo{Extract}}
\newcommand{\hibeenc}{\algo{Encaps}}
\newcommand{\hibedec}{\algo{Decaps}}

% binary tree encryption (as key encapsulation)
\newcommand{\bte}{\scheme{BTE}}
\newcommand{\btesetup}{\algo{Setup}}
\newcommand{\bteext}{\algo{Extract}}
\newcommand{\bteenc}{\algo{Encaps}}
\newcommand{\btedec}{\algo{Decaps}}

% trapdoor functions
\newcommand{\tdf}{\scheme{TDF}}
\newcommand{\tdfgen}{\algo{Gen}}
\newcommand{\tdfeval}{\algo{Eval}}
\newcommand{\tdfinv}{\algo{Invert}}
\newcommand{\tdfver}{\algo{Ver}}

%%% PROTOCOLS

\newcommand{\out}{\text{out}}
\newcommand{\view}{\text{view}}


\newcommand{\proj}[1]{\ket{#1}\!\bra{#1}}




%%%%%%%%%%%% document-writing macros %%%%%%%%%%%%

\ifnotes
\usepackage{color}
\definecolor{mygrey}{gray}{0.50}
\newcommand{\notename}[2]{{\textcolor{mygrey}{\footnotesize{\bf (#1:} {#2}{\bf ) }}}}
\newcommand{\noteswarning}{{\begin{center} {\Large WARNING: NOTES ON}\endnote{Warning: notes on}\end{center}}}

\else

\newcommand{\notename}[2]{{}}
\newcommand{\noteswarning}{{}}
\newcommand{\notesendofpaper}{}

\fi

\newcommand{\tnote}[1]{{\notename{Thomas}{#1}}}
\newcommand{\note}[1]{{\notename{Note}{#1}}}
%\newcommand{\note}[1]{}


\bibliographystyle{alpha}



\begin{document}
\header{COM-440, Introduction to Quantum Cryptography, Fall 2025}
\header{\bf Exercise Solution \# 5}


\begin{enumerate}

\item \textbf{Computing the min-entropy. }
\begin{enumerate}
\item By definition of the min-entropy, $\hmin(X \vert E) =  -\log \pguess(X \vert E)$, $\hmin(X) = -\log \pguess(X)$. So the desired inequality $\hmin(X|E) \geq \hmin(X) - \log |E|$ is equivalent to the inequality $-\log \pguess(X \vert E) \ge -\log (|E|\cdot\pguess(X))$.

Since $-\log x$ is monotonically decreasing, it suffices to prove that \[\pguess(X \vert E) \leq |E|\cdot\pguess(X)\enspace.\]
\item Suppose $A \ge 0$ and $B \ge 0$. Write the eigenvalue decomposition of $B$:
$B=\sum_i \lambda_i(B)\, \ketbra{u_i}{u_i}$. Using the linearity
and cyclicity of the trace,
\[
\begin{aligned}
\Tr(AB)
&= \Tr \Big(A \sum_i \lambda_i(B)\ketbra{u_i}{u_i}\Big)
 = \sum_i \lambda_i(B) \Tr\big(A \ketbra{u_i}{u_i}\big) \\
&\le \lambda_{\max}(B) \sum_i \Tr\big(A \ketbra{u_i}{u_i}\big)
 = \lambda_{\max}(B) \Tr \Big(A \sum_i \ket{u_i}\!\bra{u_i}\Big) \\
&= \lambda_{\max}(B) \cdot \Tr(A\cdot \Id) = \lambda_{\max}(B) \cdot \Tr(A),
\end{aligned}
\]
where the inequality uses $\Tr\big(A \ketbra{u_i}{u_i}\big)=
\bra{u_i}A\ket{u_i}\ge 0$ since $A\ge 0$.
\item Let $\{M_x\}$ be a POVM and $\rho^{E}_x$ be a quantum state on $E$.
Then $M_x \ge 0$ and $\rho^{E}_x \ge 0$ with $\Tr(\rho^{E}_x)\le 1$, which implies $\lambda_{\max}(\rho^{E}_x)\le 1$.
Applying part (b) to $A=M_x$, $B=\rho^{E}_x$ gives
\[
\Tr \big(M_x \rho^{E}_x\big) \;\le\; \lambda_{\max}(\rho^{E}_x) \cdot \Tr(M_x) \;\le\; \Tr(M_x)\enspace.
\]
\item Let $\rho_{XE} = \sum_x p_x \ketbra{x}{x}_X \otimes \rho_x^E$. Then by definition of the guessing probability,
\[\pguess(X \vert E) = \max_{\{M_x\} \text{ is a POVM}} \left(p_x \Tr(M_x \rho_x^E)\right)\enspace.\]

Applying part (c) to the above equation gives
\begin{align*}
	\pguess(X \vert E) \le&  \max_{\{M_x\} \text{ is a POVM}} \left(p_x \Tr(M_x)\right)\\
	\le& \max_x p_x \max_{\{M_x\} \text{ is a POVM}}\left(\sum_{x}\Tr(M_x)\right)\\
	=& \pguess(X) \max_{\{M_x\} \text{ is a POVM}}\Tr(\sum_x M_x)\\
	=& \pguess(X) \max_{\{M_x\} \text{ is a POVM}}\Tr(\Id_{E})\\
	=& \pguess(X) \cdot |E|\enspace.
\end{align*}

By part(a), this implies the desired inequality $\hmin(X|E) \geq \hmin(X) - \log |E|$.
\end{enumerate}

\item \textbf{A dual formulation for the conditional min-entropy. }
\begin{enumerate}
\item $Z$ is block-diagonalized. Therefore, $Z \ge 0$ is equivalent to $\forall\, x, N_x \ge 0$, which holds because $\{N_x\}$ is a valid POVM. 

Moreover, by definition of the POVM, we have
\begin{align*}
\Phi(Z) = \sum_{x\in\mathcal{X}} \,\big(\bra{x} \otimes \Id_E\big) \,Z\, \big(\ket{x}\otimes\Id_E\big) = \sum_{x\in\mathcal{X}} N_x = \Id_E	
\end{align*}
\item 
\begin{align*}
\Tr(Z\rho_{XE})
&= \Tr \Big( \sum_{x\in \mathcal{X}} \ketbra{x}{x} \otimes N_x \sum_{x'\in\mathcal{X}} \proj{x'}\otimes \rho_{x'}^E \Big) \\
&= \sum_{x,x'\in \mathcal{X}} \Tr \big( \ketbra{x}{x} \ketbra{x'}{x'} \big)\Tr \big(N_x \rho^E_{x'}\big) \\
&= \sum_{x\in \mathcal{X}} \Tr \big(N_x \rho^E_{x}\big).
\end{align*}
\item Since $Z \ge 0$, it is easy to see that $N_x = (\bra{x} \otimes \Id_E) Z (\ket{x}\otimes\Id_E) \ge 0$ for all $x$. Besides, it is easy to verify that$\sum_{x \in \mathcal{X}}N_x = \sum_{x \in \mathcal{X}} (\bra{x} \otimes \Id_E) Z (\ket{x}\otimes\Id_E) = \Tr_{X}(Z) = \Id_E$. Therefore, $\{N_x\}$ is a valid POVM.
\item By the previous results, $\{M_x\}$ can be grouped together into the diagonal of a single matrix $Z$, and the constraint that $\{M_x\}$ is a POVM can be equivalently written as $Z \ge 0$ and $\Phi(Z) = \Id_E$. Besides, the objective function $\pguess(X|E) = \sup_{\{M_x\}}\,\sum_{x\in\mathcal{X}} \,\Tr\big(M_x\rho_x\big)$ can also be written in terms of $Z$ as shown in part (b). Therefore the primal problem that gives $\pguess(X|E)$ is
\[
\begin{aligned}
\pguess(X|E)
&= \sup_Z \Tr(Z \rho_{XE}) \\
&\text{s.t. } \Phi(Z) = \Id_E, \\
&\phantom{\text{s.t. }} Z \geq 0.
\end{aligned}
\]

In the language of Problem 1 in Exercise 3, we are using $A = \rho_{XE}$ and $B = \Id_E$, and the map $$\Phi(Z) = \sum_{x\in\mathcal{X}} \,\big(\bra{x} \otimes \Id_E\big) \,Z\, \big(\ket{x}\otimes\Id_E\big).$$
\item By Exercise 3, for any matrix $Y$ defined over system $E$, $$\Phi^*(Y) = \sum_{x\in\mathcal{X}} \, \big(\ket{x}\otimes\Id_E\big) Y \big(\bra{x} \otimes \Id_E\big) = \left(\sum_{x\in\mathcal{X}} \, \ketbra{x}{x}\right)\otimes Y  = \Id_X \otimes Y.$$
\item By part (d) and part (e), the dual problem is
\begin{align*}
\pguess(X|E) &=\inf_{Y}\ \Tr(Y)\\
\quad &\text{s.t.}\quad
\Id_X \otimes Y \ge \rho_{XE}, \\
&\quad \quad \; Y=Y^\dagger.
\end{align*}
\item We will show that $\Id_X \otimes Y \ge \rho_{XE}$ is equivalent to $\forall\, x \in \mathcal{X}, Y \ge \rho_x$.

$\Leftarrow$: if $Y \ge \rho_x$, then $\ketbra{x}{x} \otimes Y \ge \ketbra{x}{x} \otimes \rho_x$. We can do a summation over $x$ and we will get $\Id_X \otimes Y \ge \rho_{XE}$.

$\Rightarrow$: by definition, we have that for all $x \in \mathcal{X}$, $\bra{x}\Id_X \otimes Y \ket{x} \ge \bra{x}\rho_{XE}\ket{x}$, which is $Y \ge \rho_x$ for each $x$.

Therefore, we can replace the constraint in the dual problem in part (f) with $\sigma \ge \rho_x$ for all $x \in \mathcal{X}$, which concludes the proof of part (g).
\item Such $\sigma$ is a feasible solution to the problem in part (g). Therefore, by the definition of infimum, we have $\pguess(X|E)\leq \Tr(\sigma)$.
\item Set the feasible solution sets for $\tau$ and $\rho$ as follows:
\[
\begin{aligned}
P_1&=\big\{\{M_{x_1}\}\,:\ \{M_{x_1}\}\text{ a POVM on }\mH_{E_1}\big\},\quad
&\Omega_1&=\big\{\sigma_1:\ \sigma_1 \ge \tau_{x_1}^{E_1},\ \forall x_1\in\mathcal{X}_1\big\},\\
P&=\big\{\{M_x\}\,:\ \{M_x\}\text{ a POVM on }\mH_{E}\big\},\quad
&\Omega&=\big\{\sigma:\ \sigma\ge \rho_x^{E},\ \forall x\in\mathcal{X}^{n}\big\}.
\end{aligned}
\]
Define the product-restricted sets
\[
\tilde P := \Big\{\{M_x\}:\ M_x=M_{x_1}\otimes\cdots\otimes M_{x_n},\ 
\{M_{x_i}\}\in P_1\Big\},\qquad
\tilde \Omega := \Big\{\sigma=\sigma_1\otimes\cdots\otimes\sigma_n:\ \sigma_i\in\Omega_1\Big\}.
\]

It is clear that $\tilde P \subseteq P$ and $\tilde \Omega \subseteq \Omega$.

From previous results, we have that
\[
\pguess(X|E)
= \sup_{\{M_x\}\in P}\sum_{x\in\mathcal{X}^{n}}\Tr(M_x\rho_x^E)
\ \ge\ 
\sup_{\{M_x\}\in \tilde P}\sum_{x\in\mathcal{X}^{n}}\Tr(M_x\rho_x^E),
\]
and
\[
\pguess(X|E)
= \inf_{\sigma\in\Omega}\Tr(\sigma)
\ \le\ 
\inf_{\sigma\in\tilde\Omega}\Tr(\sigma).
\]

It is not hard to verify that
\begin{align*}
\sup_{\{M_x\}\in \tilde P}\sum_{x\in\mathcal{X}^{n}}\Tr(M_x\rho_x^E) = (\sup_{\{M_{x_1}\}\in P_1}\sum_{x_1\in\mathcal{X}}\Tr(M_{x_1}\tau_{x_1}^E))^n = (\pguess(X_1|E_1)))^n\enspace;	
\end{align*}
Moreover,
\begin{align*}
	\inf_{\sigma\in\tilde\Omega}\Tr(\sigma) = (\inf_{\sigma_1\in\Omega_1}\Tr(\sigma_1))^n = (\pguess(X_1|E_1)))^n\enspace.
\end{align*}

Therefore, $\pguess(X|E) = (\pguess(X_1|E_1)))^n$.

Thus $\hmin(X|E)_\rho = n \hmin(X_1|E_1)_\tau$.

\end{enumerate}
\end{enumerate}

\end{document}






 
















