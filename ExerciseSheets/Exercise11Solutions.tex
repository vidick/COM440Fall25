\documentclass[12pt]{article}
\usepackage{fullpage}
\usepackage{amssymb,amsmath}
\usepackage{tikz}

\newtheorem{theorem}{Theorem}

 \newcommand{\Header}[1]{\begin{center} {\Large\bf #1} \end{center}}
 \newcommand{\header}[1]{\begin{center} {\large\bf #1} \end{center}}
\setlength{\parindent}{0.0in}
\setlength{\parskip}{1ex}


%\newif\ifnotes\notestrue
\newif\ifnotes\notesfalse


\usepackage{amsmath,amssymb,amsthm,amsfonts,latexsym,bbm,xspace,graphicx,float,mathtools,epigraph}
\usepackage[backref,colorlinks,citecolor=blue,bookmarks=true]{hyperref}
\usepackage{enumitem,manyfoot,fullpage}
\usepackage{subfig,tikz,framed}
\usepackage{endnotes}
\usepackage{braket}


\usepackage{fullpage}
\usepackage{hyperref}
\usepackage{pdfsync}
\usepackage{microtype}
\usepackage{color}
\usepackage{cleveref}

\newtheorem*{namedtheorem}{\theoremname}
\newcommand{\theoremname}{testing}
\newenvironment{named}[1]{ \renewcommand{\theoremname}{#1} \begin{namedtheorem}} {\end{namedtheorem}}
\newtheorem{lemma}[theorem]{Lemma}
\newtheorem{claim}[theorem]{Claim}
\newtheorem{proposition}[theorem]{Proposition}
\newtheorem{fact}[theorem]{Fact}
\newtheorem{corollary}[theorem]{Corollary}

\theoremstyle{definition}
\newtheorem{definition}[theorem]{Definition}
\newtheorem{remark}[theorem]{Remark}
\newtheorem{observation}[theorem]{Observation}
\newtheorem{notation}[theorem]{Notation}
\newtheorem{example}[theorem]{Example}
\newtheorem{examples}[theorem]{Examples}
\newtheorem{exercise}{Exercise}


\newenvironment{quotenote}{
\begin{quote}
  \footnotesize
\noindent{\bf Note:}}
{\end{quote}
}


% probability and other mathops
\renewcommand{\Pr}{\mathop{\bf Pr\/}}
\newcommand{\E}{\mathop{\bf E\/}}
\newcommand{\Ex}{\mathop{\bf E\/}}
\newcommand{\Var}{\mathop{\bf Var\/}}
\newcommand{\Cov}{\mathop{\bf Cov\/}}
\newcommand{\stddev}{\mathop{\bf stddev\/}}
\newcommand{\littlesum}{\mathop{{\textstyle \sum}}}
\newcommand{\apx}{\mathop{\approx}}

\newcommand{\epr}{\textsc{EPR}}

\newcommand{\Zt}{\ensuremath{\Z_t}}
\newcommand{\Zp}{\ensuremath{\Z_p}}
\newcommand{\Zq}{\ensuremath{\Z_q}}
\newcommand{\ZN}{\ensuremath{\Z_N}}
\newcommand{\Zps}{\ensuremath{\Z_p^*}}
\newcommand{\ZNs}{\ensuremath{\Z_N^*}}
\newcommand{\JN}{\ensuremath{\J_N}}
\newcommand{\QR}{\ensuremath{\mathbb{QR}}}
\newcommand{\QRN}{\ensuremath{\QR_{N}}}
\newcommand{\QRp}{\ensuremath{\QR_{p}}}

% mathrm terms
\newcommand{\poly}{\mathrm{poly}}
\newcommand{\negl}{\mathrm{negl}}
\newcommand{\Tr}{\mathrm{Tr}}
\newcommand{\polylog}{\mathrm{polylog}}
\newcommand{\size}{\mathrm{size}}
\newcommand{\avg}{\mathop{\mathrm{avg}}}
\newcommand{\sgn}{\mathrm{sgn}}
\newcommand{\dist}{\mathrm{dist}}
\newcommand{\spn}{\mathrm{span}}
\newcommand{\supp}{\mathrm{supp}}
\newcommand{\Val}{\mathrm{Val}}
\newcommand{\Opt}{\mathrm{Opt}}
\newcommand{\LPOpt}{\mathrm{LPOpt}}
\newcommand{\SDPOpt}{\mathrm{SDPOpt}}
\newcommand{\vol}{\mathrm{vol}}
\newcommand{\Id}{\mathbb{I}}

% number systems
\newcommand{\R}{\mathbbm R}
\newcommand{\C}{\mathbbm C}
\newcommand{\N}{\mathbbm N}
\newcommand{\Z}{\mathbbm Z}
\newcommand{\F}{\mathbbm F}
\newcommand{\Q}{\mathbbm Q}

\newcommand{\mH}{\mathcal{H}}

% complexity classes
\newcommand{\PTIME}{\mathsf{P}}
\newcommand{\NP}{\mathsf{NP}} \newcommand{\np}{\NP}

% short forms
\newcommand{\eps}{\varepsilon}
\newcommand{\lam}{\lambda}
\newcommand{\vphi}{\varphi}
\newcommand{\la}{\langle}
\newcommand{\ra}{\rangle}
\newcommand{\wt}[1]{\widetilde{#1}}
\newcommand{\wh}[1]{\widehat{#1}}
\newcommand{\ul}[1]{\underline{#1}}
\newcommand{\ol}[1]{\overline{#1}}
\newcommand{\ot}{\otimes}
\newcommand{\Ra}{\Rightarrow}
\newcommand{\half}{\tfrac{1}{2}}
\newcommand{\grad}{\nabla}
\newcommand{\sse}{\subseteq}


% calligraphic letters
\newcommand{\calA}{\mathcal{A}}
\newcommand{\calB}{\mathcal{B}}
\newcommand{\calC}{\mathcal{C}}
\newcommand{\calD}{\mathcal{D}}
\newcommand{\calE}{\mathcal{E}}
\newcommand{\calF}{\mathcal{F}}
\newcommand{\calG}{\mathcal{G}}
\newcommand{\calH}{\mathcal{G}}
\newcommand{\calI}{\mathcal{I}}
\newcommand{\calJ}{\mathcal{J}}
\newcommand{\calK}{\mathcal{K}}
\newcommand{\calL}{\mathcal{L}}
\newcommand{\calM}{\mathcal{M}}
\newcommand{\calN}{\mathcal{N}}
\newcommand{\calO}{\mathcal{O}}
\newcommand{\calP}{\mathcal{P}}
\newcommand{\calQ}{\mathcal{Q}}
\newcommand{\calR}{\mathcal{R}}
\newcommand{\calS}{\mathcal{S}}
\newcommand{\calT}{\mathcal{T}}
\newcommand{\calU}{\mathcal{U}}
\newcommand{\calV}{\mathcal{V}}
\newcommand{\calW}{\mathcal{W}}
\newcommand{\calX}{\mathcal{X}}
\newcommand{\calY}{\mathcal{Y}}
\newcommand{\calZ}{\mathcal{Z}}


\newcommand{\myfig}[4]{\begin{figure}[H] \begin{center} \includegraphics[width=#1\textwidth]{#2} \caption{#3} \label{#4} \end{center} \end{figure}} 

\newcommand{\bit}{\ensuremath{\{0,1\}}}

%%% CRYPTO-RELATED NOTATION

% length of a string
\newcommand{\len}[1]{\lvert{#1}\rvert}
\newcommand{\lenfit}[1]{\left\lvert{#1}\right\rvert}
% length of some vector, element
\newcommand{\length}[1]{\lVert{#1}\rVert}
\newcommand{\lengthfit}[1]{\left\lVert{#1}\right\rVert}


% types of indistinguishability
\newcommand{\compind}{\ensuremath{\stackrel{c}{\approx}}}
\newcommand{\statind}{\ensuremath{\stackrel{s}{\approx}}}
\newcommand{\perfind}{\ensuremath{\equiv}}

% font for general-purpose algorithms
\newcommand{\algo}[1]{\ensuremath{\mathsf{#1}}}
% font for general-purpose computational problems
\newcommand{\problem}[1]{\ensuremath{\mathsf{#1}}}
% font for complexity classes
\newcommand{\class}[1]{\ensuremath{\mathsf{#1}}}


% KEYS AND RELATED

\newcommand{\key}[1]{\ensuremath{#1}}

\newcommand{\pk}{\key{pk}}
\newcommand{\vk}{\key{vk}}
\newcommand{\sk}{\key{sk}}
\newcommand{\mpk}{\key{mpk}}
\newcommand{\msk}{\key{msk}}
\newcommand{\fk}{\key{fk}}
\newcommand{\id}{id}
\newcommand{\keyspace}{\ensuremath{\mathcal{K}}}
\newcommand{\msgspace}{\ensuremath{\mathcal{M}}}
\newcommand{\ctspace}{\ensuremath{\mathcal{C}}}
\newcommand{\tagspace}{\ensuremath{\mathcal{T}}}
\newcommand{\idspace}{\ensuremath{\mathcal{ID}}}

\newcommand{\concat}{\ensuremath{\|}}

% GAMES

% advantage
\newcommand{\advan}{\ensuremath{\mathbf{Adv}}}

% different attack models
\newcommand{\attack}[1]{\ensuremath{\text{#1}}}

\newcommand{\atk}{\attack{atk}} % dummy attack
\newcommand{\indcpa}{\attack{ind-cpa}}
\newcommand{\indcca}{\attack{ind-cca}}
\newcommand{\anocpa}{\attack{ano-cpa}} % anonymous
\newcommand{\anocca}{\attack{ano-cca}}
\newcommand{\euacma}{\attack{eu-acma}} % forgery: adaptive chosen-message
\newcommand{\euscma}{\attack{eu-scma}} % forgery: static chosen-message
\newcommand{\suacma}{\attack{su-acma}} % strongly unforgeable

% ADVERSARIES
\newcommand{\attacker}[1]{\ensuremath{\mathcal{#1}}}

\newcommand{\Adv}{\attacker{A}}
\newcommand{\AdvA}{\attacker{A}}
\newcommand{\AdvB}{\attacker{B}}
\newcommand{\Dist}{\attacker{D}}
\newcommand{\Sim}{\attacker{S}}
\newcommand{\Ora}{\attacker{O}}
\newcommand{\Inv}{\attacker{I}}
\newcommand{\For}{\attacker{F}}

% CRYPTO SCHEMES

\newcommand{\scheme}[1]{\ensuremath{\text{#1}}}

% pseudorandom stuff
\newcommand{\prg}{\algo{PRG}}
\newcommand{\prf}{\algo{PRF}}
\newcommand{\prp}{\algo{PRP}}

% symmetric-key cryptosystem
\newcommand{\skc}{\scheme{SKC}}
\newcommand{\skcgen}{\algo{Gen}}
\newcommand{\skcenc}{\algo{Enc}}
\newcommand{\skcdec}{\algo{Dec}}

% public-key cryptosystem
\newcommand{\pkc}{\scheme{PKC}}
\newcommand{\pkcgen}{\algo{Gen}}
\newcommand{\pkcenc}{\algo{Enc}} % can also use \kemenc and \kemdec
\newcommand{\pkcdec}{\algo{Dec}}

% digital signatures
\newcommand{\sig}{\scheme{SIG}}
\newcommand{\siggen}{\algo{Gen}}
\newcommand{\sigsign}{\algo{Sign}}
\newcommand{\sigver}{\algo{Ver}}

% message authentication code
\newcommand{\mac}{\scheme{MAC}}
\newcommand{\macgen}{\algo{Gen}}
\newcommand{\mactag}{\algo{Tag}}
\newcommand{\macver}{\algo{Ver}}

% key-encapsulation mechanism
\newcommand{\kem}{\scheme{KEM}}
\newcommand{\kemgen}{\algo{Gen}}
\newcommand{\kemenc}{\algo{Encaps}}
\newcommand{\kemdec}{\algo{Decaps}}

% identity-based encryption
\newcommand{\ibe}{\scheme{IBE}}
\newcommand{\ibesetup}{\algo{Setup}}
\newcommand{\ibeext}{\algo{Ext}}
\newcommand{\ibeenc}{\algo{Enc}}
\newcommand{\ibedec}{\algo{Dec}}

% hierarchical IBE (as key encapsulation)
\newcommand{\hibe}{\scheme{HIBE}}
\newcommand{\hibesetup}{\algo{Setup}}
\newcommand{\hibeext}{\algo{Extract}}
\newcommand{\hibeenc}{\algo{Encaps}}
\newcommand{\hibedec}{\algo{Decaps}}

% binary tree encryption (as key encapsulation)
\newcommand{\bte}{\scheme{BTE}}
\newcommand{\btesetup}{\algo{Setup}}
\newcommand{\bteext}{\algo{Extract}}
\newcommand{\bteenc}{\algo{Encaps}}
\newcommand{\btedec}{\algo{Decaps}}

% trapdoor functions
\newcommand{\tdf}{\scheme{TDF}}
\newcommand{\tdfgen}{\algo{Gen}}
\newcommand{\tdfeval}{\algo{Eval}}
\newcommand{\tdfinv}{\algo{Invert}}
\newcommand{\tdfver}{\algo{Ver}}

%%% PROTOCOLS

\newcommand{\out}{\text{out}}
\newcommand{\view}{\text{view}}


\newcommand{\proj}[1]{\ket{#1}\!\bra{#1}}




%%%%%%%%%%%% document-writing macros %%%%%%%%%%%%

\ifnotes
\usepackage{color}
\definecolor{mygrey}{gray}{0.50}
\newcommand{\notename}[2]{{\textcolor{mygrey}{\footnotesize{\bf (#1:} {#2}{\bf ) }}}}
\newcommand{\noteswarning}{{\begin{center} {\Large WARNING: NOTES ON}\endnote{Warning: notes on}\end{center}}}

\else

\newcommand{\notename}[2]{{}}
\newcommand{\noteswarning}{{}}
\newcommand{\notesendofpaper}{}

\fi

\newcommand{\tnote}[1]{{\notename{Thomas}{#1}}}
\newcommand{\note}[1]{{\notename{Note}{#1}}}
%\newcommand{\note}[1]{}


\bibliographystyle{alpha}



\begin{document}
\header{COM-440, Introduction to Quantum Cryptography, Fall 2025}
\header{\bf Exercise Solution \# 11}


\begin{enumerate}



\item {\bf Coin flipping from bit commitment}
\begin{enumerate}
\item  Alice and Bob use the following protocol:

\begin{tikzpicture}[>=stealth]

% Titles
\node at (-1,0)  {\bf Alice};
\node at (12,0)  {\bf Bob};
% --------- Row 1 ----------
\node[anchor=west]  at (-2,-1)
  {$a \in_R \{0, 1\}$};

\draw[->] (2,-1) -- (10.5,-1)
  node[midway, above] {Commit to $a$ with the bit commitment scheme};

% --------- Row 2 ----------
\node[anchor=east] at (13,-2)
  {$b \in_R \{0, 1\}$};
\draw[<-] (2,-2) -- (10.5,-2)
  node[midway, above] {$b$};
\node[anchor=west]  at (-2,-2)
  {$c := a \oplus b$};
  
% --------- Row 3 ----------
\draw[->] (2,-3) -- (10.5,-3)
  node[midway, above] {Open to $a$ with the bit commitment scheme};
\node[anchor=east] at (12.9,-3)
  {$c := a \oplus b$};
\end{tikzpicture}
\item We first consider a cheating Bob.

If a cheating Bob can make $c = 0$ with probability $p > 1/2$ in the above coin flipping protocol, he can guess $a$ with probability $p$ after Alice commits, by setting $a$ equal to the message he is going to send in the second phase. By $\eps_h$-hiding of bit commitment, $p \leq \frac{1}{2} + \eps_h$. 

Similarly, if a cheating Bob can make $c+0$ with probability $p < 1/2$ in the above coin flipping protocol, he can guess $a$ with probability $p$ after Alice commits, by setting $a = 1 - b$ (the opposite of the message he is going to send in the second phase). By $\eps_h$-hiding of bit commitment, $1 - p \leq \frac{1}{2} + \eps_h$. 

Therefore, for a dishonest Bob, $\frac{1}{2} - \eps_h \leq \Pr[c = 0] \leq \frac{1}{2} + \eps_h$.

Now let's consider a cheating Alice.

If a cheating Alice can make $c = 0$ with probability $p > 1/2$ in the above coin flipping protocol, she can run the first phase of the protocol to send a (potentially dishonest) commitment. Then the probability that she can open it to 0 is at least $\Pr[c = 0\,|\,b = 0]$, and the probability that she can open it to 1 is at least $\Pr[c = 0\,|\,b = 1]$. By the $\eps_b$-binding of the bit commitment, $\Pr[c = 0\,|\,b = 0] + \Pr[c = 0\,|\,b = 1] \leq 1 + \eps_b$, which means $2p \leq 1 + \eps_b$ since honest Bob chooses $b$ uniformly at random. 

Similarly, if a cheating Alice can make $c = 0$ with probability $p < 1/2$ in the above coin flipping protocol, it means she can make $c = 1$ with probability $1 - p$ in the above coin flipping protocol. Using the same argument, we can show that $2(1 - p) \leq 1 + \eps_b$. 

Therefore, for a dishonest Bob, $\frac{1}{2} - \frac{1}{2}\eps_b \leq \Pr[c = 0] \leq \frac{1}{2} + \frac{1}{2}\eps_b$.

So the maximum bias of the coin flipping protocol is $\max(\eps_h, \frac{1}{2}\eps_b)$.
\end{enumerate}

\item {\bf Different flavors of oblivious transfer}
Below we provide constructions and intuitions on why the constructions are secure. For a formal proof, you need the ideal functionality in the secure function evaluation.
\begin{enumerate}
\item The reduction is straight-forward: the sender sends $(b_0,b_1,0,...,0)$ via 1-out-of-$k$ OT, and the receiver picks $c \in \{0, 1\}$. 
\item Alice and Bob use the following protocol:

\begin{tikzpicture}[>=stealth]

% Titles
\node at (0,0)  {\bf Alice};
\node at (10,0)  {\bf Bob};
% --------- Row 1 ----------
\node[anchor=west]  at (-2,-1)
  {$r_1 \in_R \{0,1\},\ e_1 := b_1$};
\node[anchor=east] at (13,-1)
  {$\text{if } c = 1,\ \text{pick } e_1,\ \text{else } r_1$};

\draw[->] (4,-1) -- (7.6,-1)
  node[midway, above] {$[e_1 \mid r_1]_{1\text{-}2\text{-OT}}$};

% --------- Row 2 ----------
\node[anchor=west]  at (-2,-2)
  {$r_2 \in_R \{0,1\},\ e_2 := b_2 \oplus r_1$};
\node[anchor=east] at (13,-2)
  {$\text{if } c = 2,\ \text{pick } e_2,\ \text{else } r_2$};
\draw[->] (4,-2) -- (7.6,-2)
  node[midway, above] {$[e_2 \mid r_2]_{1\text{-}2\text{-OT}}$};

% --------- Row 3 ----------
\node[anchor=west]  at (-2,-3)
  {$r_3 \in_R \{0,1\},\ e_3 := b_3 \oplus r_1 \oplus r_2$};
\node[anchor=east] at (13,-3)
  {$\text{if } c = 3,\ \text{pick } e_3,\ \text{else } r_3$};
\draw[->] (4,-3) -- (7.6,-3)
  node[midway, above] {$[e_3 \mid r_3]_{1\text{-}2\text{-OT}}$};

% --------- Vertical dots ----------
\node at (0,-4)  {$\vdots$};
\node at (9,-4)  {$\vdots$};
\node at (6,-4)  {$\vdots$};

% --------- Bottom row ----------
\node[anchor=west]  at (-2,-5)
  {$e_k := b_k \oplus r_1 \oplus \dots \oplus r_{k-1}$};
\node[anchor=east] at (13,-5)
  {$b := e_c \oplus r_1 \oplus \dots \oplus r_{c-1}$};
\draw[->] (4,-5) -- (7.6,-5)
  node[midway, above] {$e_k$};
\end{tikzpicture}

Alice trivially does not learn any information about Bob's choice $c \in \{1,...,k\}$ because of the guarantee of each 1-out-of-2 OT. If Bob wishes to learn bit $b_c$, he needs to know all preceding one-time pads $r_1, \ldots ,r_{c-1}$ as well as the value $e_c$. Hence, he cannot choose any of the values $e_1, \ldots, e_{c - 1}$, and he has to choose the bit $e_c$. However, in that case he does not learn $e_i$ for $i < c$, and thus learns no information about $b_i$ for $i<c$, and moreover he does not learn $r_c$, and thus learns no information about $b_i$ for $i > c$. Hence, even when Bob does not follow the protocol, he learns at most one of the $k$ bits.
\item Alice and Bob use the following protocol:

\begin{tikzpicture}[>=stealth]

% Titles
\node at (0,0)  {\bf Alice};
\node at (8,0)  {\bf Bob};

% --------- Row 1 ----------
\node[anchor=west]  at (-2,-1)
  {$i \in_R \{0,1\}$};
 \node[anchor=east]  at (9,-1)
  {$j \in_R \{0,1\}$};
  
% --------- Row 2 ----------
\node[anchor=west]  at (-2,-2)
  {$b_i := b, \ b_{1 - i} = 0$};
\node[anchor=east] at (8.2,-2)
  {pick $b_j$};
\draw[->] (2.5,-2) -- (5.5,-2)
  node[midway, above] {$[b_0 \mid b_1]_{1\text{-}2\text{-OT}}$};

% --------- Row 3 ----------
\draw[->] (2.5,-3) -- (5.5,-3)
  node[midway, above] {$i$};
\node[anchor=east] at (13.2,-3)
  {if $i= j$, set $b:= b_j$, else set $b := \bot$};
\end{tikzpicture}

Alice does not learn any information about whether Bob receives the bit or
not because Alice does not learn anything during the first phase by the guarantee of 1-out-of-2 OT, and Bob does not send anything to Alice in the second phase. Moreover, Bob receives the bit with probability $1/2$ otherwise has no information about it because by the guarantee of 1-out-of-2 OT, Bob can learn only one of $b_0, b_1$.
\item Let $\kappa$ be a security parameter. Alice and Bob use the following protocol:

\begin{tikzpicture}[>=stealth]

% Titles
\node at (0,0)  {\bf Alice};
\node at (10,0)  {\bf Bob};

% --------- Row 1 ----------
\node[anchor=west]  at (-2,-1)
  {$r_1,\ldots,r_\kappa \in_R \{0,1\}$};
\node[anchor=east] at (13,-1)
  {$\forall\, i$: receive $r_i' \in \{r_i, \bot\}$};
\draw[->] (4,-1) -- (7,-1)
  node[midway, above] {$\forall i : [r_i]_{\text{Rabin-OT}}$};
  
% --------- Row 2 ----------
\node[anchor=west]  at (-2,-2)
  {$t_0 := \bigoplus_{i\in T_0} r_i,\,
    t_1 := \bigoplus_{i\in T_1} r_i$};
\node[anchor=east] at (12.2,-1.8)
  {$T_c := \{\, i \mid r'_i \neq \bot \,\}$};
\node[anchor=east] at (12.5,-2.4)
  {$T_{1-c} := \{\, i \mid r'_i = \bot \,\}$};
\draw[<-] (4,-2) -- (7,-2)
  node[midway, above] {$T_0, T_1$};
  
% --------- Row 3 ----------
\node[anchor=west]  at (-2,-3.5)
  {$e_0 := t_0 \oplus b_0,\,
    e_1 := t_1 \oplus b_1$};
\node[anchor=east] at (13.6,-3.5)
  {$t_c := \bigoplus_{i\in T_c} r_i'$, $b_c := e_c \oplus t_c$};
\draw[->] (4,-3.5) -- (7,-3.5)
  node[midway, above] {$e_0,e_1$};
\end{tikzpicture}

Alice does not learn any information about Bob's choice $c \in \{1,\ldots,k\}$ since the sets $T_0$ and $T_1$ do not reveal which instances of the underlying Rabin OT were successful. Furthermore, with probability $1-2^{-\kappa}$ there is at least one bit $r_i$ the receiver does not learn, and, therefore, at least one of the one-time pads $t_0$ and $t_1$ is uniformly random. Therefore, except with probability $2^{-\kappa}$, the receiver learns at most one of the bits $b_0$ and $b_1$.
\end{enumerate}


\end{enumerate}

\end{document}






 
















