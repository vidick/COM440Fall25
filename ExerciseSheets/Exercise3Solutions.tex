\documentclass[12pt]{article}
\usepackage{fullpage}
\usepackage{amssymb,amsmath}

\newtheorem{theorem}{Theorem}

 \newcommand{\Header}[1]{\begin{center} {\Large\bf #1} \end{center}}
 \newcommand{\header}[1]{\begin{center} {\large\bf #1} \end{center}}
\setlength{\parindent}{0.0in}
\setlength{\parskip}{1ex}


%\newif\ifnotes\notestrue
\newif\ifnotes\notesfalse


\usepackage{amsmath,amssymb,amsthm,amsfonts,latexsym,bbm,xspace,graphicx,float,mathtools,epigraph}
\usepackage[backref,colorlinks,citecolor=blue,bookmarks=true]{hyperref}
\usepackage{enumitem,manyfoot,fullpage}
\usepackage{subfig,tikz,framed}
\usepackage{endnotes}
\usepackage{braket}


\usepackage{fullpage}
\usepackage{hyperref}
\usepackage{pdfsync}
\usepackage{microtype}
\usepackage{color}
\usepackage{cleveref}

\newtheorem*{namedtheorem}{\theoremname}
\newcommand{\theoremname}{testing}
\newenvironment{named}[1]{ \renewcommand{\theoremname}{#1} \begin{namedtheorem}} {\end{namedtheorem}}
\newtheorem{lemma}[theorem]{Lemma}
\newtheorem{claim}[theorem]{Claim}
\newtheorem{proposition}[theorem]{Proposition}
\newtheorem{fact}[theorem]{Fact}
\newtheorem{corollary}[theorem]{Corollary}

\theoremstyle{definition}
\newtheorem{definition}[theorem]{Definition}
\newtheorem{remark}[theorem]{Remark}
\newtheorem{observation}[theorem]{Observation}
\newtheorem{notation}[theorem]{Notation}
\newtheorem{example}[theorem]{Example}
\newtheorem{examples}[theorem]{Examples}
\newtheorem{exercise}{Exercise}


\newenvironment{quotenote}{
\begin{quote}
  \footnotesize
\noindent{\bf Note:}}
{\end{quote}
}


% probability and other mathops
\renewcommand{\Pr}{\mathop{\bf Pr\/}}
\newcommand{\E}{\mathop{\bf E\/}}
\newcommand{\Ex}{\mathop{\bf E\/}}
\newcommand{\Var}{\mathop{\bf Var\/}}
\newcommand{\Cov}{\mathop{\bf Cov\/}}
\newcommand{\stddev}{\mathop{\bf stddev\/}}
\newcommand{\littlesum}{\mathop{{\textstyle \sum}}}
\newcommand{\apx}{\mathop{\approx}}

\newcommand{\epr}{\textsc{EPR}}

\newcommand{\Zt}{\ensuremath{\Z_t}}
\newcommand{\Zp}{\ensuremath{\Z_p}}
\newcommand{\Zq}{\ensuremath{\Z_q}}
\newcommand{\ZN}{\ensuremath{\Z_N}}
\newcommand{\Zps}{\ensuremath{\Z_p^*}}
\newcommand{\ZNs}{\ensuremath{\Z_N^*}}
\newcommand{\JN}{\ensuremath{\J_N}}
\newcommand{\QR}{\ensuremath{\mathbb{QR}}}
\newcommand{\QRN}{\ensuremath{\QR_{N}}}
\newcommand{\QRp}{\ensuremath{\QR_{p}}}

% mathrm terms
\newcommand{\poly}{\mathrm{poly}}
\newcommand{\negl}{\mathrm{negl}}
\newcommand{\Tr}{\mathrm{Tr}}
\newcommand{\polylog}{\mathrm{polylog}}
\newcommand{\size}{\mathrm{size}}
\newcommand{\avg}{\mathop{\mathrm{avg}}}
\newcommand{\sgn}{\mathrm{sgn}}
\newcommand{\dist}{\mathrm{dist}}
\newcommand{\spn}{\mathrm{span}}
\newcommand{\supp}{\mathrm{supp}}
\newcommand{\Val}{\mathrm{Val}}
\newcommand{\Opt}{\mathrm{Opt}}
\newcommand{\LPOpt}{\mathrm{LPOpt}}
\newcommand{\SDPOpt}{\mathrm{SDPOpt}}
\newcommand{\vol}{\mathrm{vol}}
\newcommand{\Id}{\mathbb{I}}

\newcommand{\Ext}{\mathrm{Ext}}
\newcommand{\IP}{\mathrm{IP}}

% number systems
\newcommand{\R}{\mathbbm R}
\newcommand{\C}{\mathbbm C}
\newcommand{\N}{\mathbbm N}
\newcommand{\Z}{\mathbbm Z}
\newcommand{\F}{\mathbbm F}
\newcommand{\Q}{\mathbbm Q}

\newcommand{\mH}{\mathcal{H}}

% complexity classes
\newcommand{\PTIME}{\mathsf{P}}
\newcommand{\NP}{\mathsf{NP}} \newcommand{\np}{\NP}

% short forms
\newcommand{\eps}{\varepsilon}
\newcommand{\lam}{\lambda}
\newcommand{\vphi}{\varphi}
\newcommand{\la}{\langle}
\newcommand{\ra}{\rangle}
\newcommand{\wt}[1]{\widetilde{#1}}
\newcommand{\wh}[1]{\widehat{#1}}
\newcommand{\ul}[1]{\underline{#1}}
\newcommand{\ol}[1]{\overline{#1}}
\newcommand{\ot}{\otimes}
\newcommand{\Ra}{\Rightarrow}
\newcommand{\half}{\tfrac{1}{2}}
\newcommand{\grad}{\nabla}
\newcommand{\sse}{\subseteq}


% calligraphic letters
\newcommand{\calA}{\mathcal{A}}
\newcommand{\calB}{\mathcal{B}}
\newcommand{\calC}{\mathcal{C}}
\newcommand{\calD}{\mathcal{D}}
\newcommand{\calE}{\mathcal{E}}
\newcommand{\calF}{\mathcal{F}}
\newcommand{\calG}{\mathcal{G}}
\newcommand{\calH}{\mathcal{G}}
\newcommand{\calI}{\mathcal{I}}
\newcommand{\calJ}{\mathcal{J}}
\newcommand{\calK}{\mathcal{K}}
\newcommand{\calL}{\mathcal{L}}
\newcommand{\calM}{\mathcal{M}}
\newcommand{\calN}{\mathcal{N}}
\newcommand{\calO}{\mathcal{O}}
\newcommand{\calP}{\mathcal{P}}
\newcommand{\calQ}{\mathcal{Q}}
\newcommand{\calR}{\mathcal{R}}
\newcommand{\calS}{\mathcal{S}}
\newcommand{\calT}{\mathcal{T}}
\newcommand{\calU}{\mathcal{U}}
\newcommand{\calV}{\mathcal{V}}
\newcommand{\calW}{\mathcal{W}}
\newcommand{\calX}{\mathcal{X}}
\newcommand{\calY}{\mathcal{Y}}
\newcommand{\calZ}{\mathcal{Z}}

%\newcommand{\ketbra}[2]{|#1\rangle\langle#2|}
\newcommand{\hmin}{H_{\rm min}}
\newcommand{\Hmin}{H_{\rm min}}

\newcommand{\myfig}[4]{\begin{figure}[H] \begin{center} \includegraphics[width=#1\textwidth]{#2} \caption{#3} \label{#4} \end{center} \end{figure}} 

\newcommand{\bit}{\ensuremath{\{0,1\}}}

%%% CRYPTO-RELATED NOTATION

% length of a string
\newcommand{\len}[1]{\lvert{#1}\rvert}
\newcommand{\lenfit}[1]{\left\lvert{#1}\right\rvert}
% length of some vector, element
\newcommand{\length}[1]{\lVert{#1}\rVert}
\newcommand{\lengthfit}[1]{\left\lVert{#1}\right\rVert}


% types of indistinguishability
\newcommand{\compind}{\ensuremath{\stackrel{c}{\approx}}}
\newcommand{\statind}{\ensuremath{\stackrel{s}{\approx}}}
\newcommand{\perfind}{\ensuremath{\equiv}}

% font for general-purpose algorithms
\newcommand{\algo}[1]{\ensuremath{\mathsf{#1}}}
% font for general-purpose computational problems
\newcommand{\problem}[1]{\ensuremath{\mathsf{#1}}}
% font for complexity classes
\newcommand{\class}[1]{\ensuremath{\mathsf{#1}}}


% KEYS AND RELATED

\newcommand{\key}[1]{\ensuremath{#1}}

\newcommand{\pk}{\key{pk}}
\newcommand{\vk}{\key{vk}}
\newcommand{\sk}{\key{sk}}
\newcommand{\mpk}{\key{mpk}}
\newcommand{\msk}{\key{msk}}
\newcommand{\fk}{\key{fk}}
\newcommand{\id}{id}
\newcommand{\keyspace}{\ensuremath{\mathcal{K}}}
\newcommand{\msgspace}{\ensuremath{\mathcal{M}}}
\newcommand{\ctspace}{\ensuremath{\mathcal{C}}}
\newcommand{\tagspace}{\ensuremath{\mathcal{T}}}
\newcommand{\idspace}{\ensuremath{\mathcal{ID}}}

\newcommand{\concat}{\ensuremath{\|}}

% GAMES

% advantage
\newcommand{\advan}{\ensuremath{\mathbf{Adv}}}

% different attack models
\newcommand{\attack}[1]{\ensuremath{\text{#1}}}

\newcommand{\atk}{\attack{atk}} % dummy attack
\newcommand{\indcpa}{\attack{ind-cpa}}
\newcommand{\indcca}{\attack{ind-cca}}
\newcommand{\anocpa}{\attack{ano-cpa}} % anonymous
\newcommand{\anocca}{\attack{ano-cca}}
\newcommand{\euacma}{\attack{eu-acma}} % forgery: adaptive chosen-message
\newcommand{\euscma}{\attack{eu-scma}} % forgery: static chosen-message
\newcommand{\suacma}{\attack{su-acma}} % strongly unforgeable

% ADVERSARIES
\newcommand{\attacker}[1]{\ensuremath{\mathcal{#1}}}

\newcommand{\Adv}{\attacker{A}}
\newcommand{\AdvA}{\attacker{A}}
\newcommand{\AdvB}{\attacker{B}}
\newcommand{\Dist}{\attacker{D}}
\newcommand{\Sim}{\attacker{S}}
\newcommand{\Ora}{\attacker{O}}
\newcommand{\Inv}{\attacker{I}}
\newcommand{\For}{\attacker{F}}

% CRYPTO SCHEMES

\newcommand{\scheme}[1]{\ensuremath{\text{#1}}}

% pseudorandom stuff
\newcommand{\prg}{\algo{PRG}}
\newcommand{\prf}{\algo{PRF}}
\newcommand{\prp}{\algo{PRP}}

% symmetric-key cryptosystem
\newcommand{\skc}{\scheme{SKC}}
\newcommand{\skcgen}{\algo{Gen}}
\newcommand{\skcenc}{\algo{Enc}}
\newcommand{\skcdec}{\algo{Dec}}

% public-key cryptosystem
\newcommand{\pkc}{\scheme{PKC}}
\newcommand{\pkcgen}{\algo{Gen}}
\newcommand{\pkcenc}{\algo{Enc}} % can also use \kemenc and \kemdec
\newcommand{\pkcdec}{\algo{Dec}}

% digital signatures
\newcommand{\sig}{\scheme{SIG}}
\newcommand{\siggen}{\algo{Gen}}
\newcommand{\sigsign}{\algo{Sign}}
\newcommand{\sigver}{\algo{Ver}}

% message authentication code
\newcommand{\mac}{\scheme{MAC}}
\newcommand{\macgen}{\algo{Gen}}
\newcommand{\mactag}{\algo{Tag}}
\newcommand{\macver}{\algo{Ver}}

% key-encapsulation mechanism
\newcommand{\kem}{\scheme{KEM}}
\newcommand{\kemgen}{\algo{Gen}}
\newcommand{\kemenc}{\algo{Encaps}}
\newcommand{\kemdec}{\algo{Decaps}}

% identity-based encryption
\newcommand{\ibe}{\scheme{IBE}}
\newcommand{\ibesetup}{\algo{Setup}}
\newcommand{\ibeext}{\algo{Ext}}
\newcommand{\ibeenc}{\algo{Enc}}
\newcommand{\ibedec}{\algo{Dec}}

% hierarchical IBE (as key encapsulation)
\newcommand{\hibe}{\scheme{HIBE}}
\newcommand{\hibesetup}{\algo{Setup}}
\newcommand{\hibeext}{\algo{Extract}}
\newcommand{\hibeenc}{\algo{Encaps}}
\newcommand{\hibedec}{\algo{Decaps}}

% binary tree encryption (as key encapsulation)
\newcommand{\bte}{\scheme{BTE}}
\newcommand{\btesetup}{\algo{Setup}}
\newcommand{\bteext}{\algo{Extract}}
\newcommand{\bteenc}{\algo{Encaps}}
\newcommand{\btedec}{\algo{Decaps}}

% trapdoor functions
\newcommand{\tdf}{\scheme{TDF}}
\newcommand{\tdfgen}{\algo{Gen}}
\newcommand{\tdfeval}{\algo{Eval}}
\newcommand{\tdfinv}{\algo{Invert}}
\newcommand{\tdfver}{\algo{Ver}}

%%% PROTOCOLS

\newcommand{\out}{\text{out}}
\newcommand{\view}{\text{view}}


\newcommand{\proj}[1]{\ket{#1}\!\!\bra{#1}}
\newcommand{\ketbra}[2]{\ket{#1}\!\bra{#2}}
\renewcommand{\braket}[2]{\langle #1 \vert  #2\rangle}


\newcommand{\inp}[2]{\langle{#1}|{#2}\rangle} 


%%%%%%%%%%%% document-writing macros %%%%%%%%%%%%

\ifnotes
\usepackage{color}
\definecolor{mygrey}{gray}{0.50}
\newcommand{\notename}[2]{{\textcolor{mygrey}{\footnotesize{\bf (#1:} {#2}{\bf ) }}}}
\newcommand{\noteswarning}{{\begin{center} {\Large WARNING: NOTES ON}\endnote{Warning: notes on}\end{center}}}

\else

\newcommand{\notename}[2]{{}}
\newcommand{\noteswarning}{{}}
\newcommand{\notesendofpaper}{}

\fi

\newcommand{\tnote}[1]{{\notename{Thomas}{#1}}}
\newcommand{\note}[1]{{\notename{Note}{#1}}}
%\newcommand{\note}[1]{}


\bibliographystyle{alpha}



\begin{document}
\header{COM-440, Introduction to Quantum Cryptography, Fall 2025}
\header{\bf Exercise Solution \# 3}


\begin{enumerate}

\item {\bf Semidefinite programming}
\begin{enumerate}
\item We first prove a lemma: If $X,Y\in \mathbb{C}^{d \times d}$, $X\ge 0$, $Y\ge 0$, then $\Tr(XY)\ge 0$.

This is because for $X \ge 0$, we can find eigenvalue decomposition of $X$: $X = \sum_{i = 1}^d \lambda_i \ketbra{\phi_i}{\phi_i}$ where the eigenvalues $\lambda_i \ge 0$. 

Then $\Tr(XY) = \Tr(\sum_{i = 1}^d \lambda_i \ketbra{\phi_i}{\phi_i}Y) = \sum_{i = 1}^d\lambda_i \bra{\phi_i} Y \ket{\phi_i} \ge 0$ because since $Y \ge 0$, we have that $\bra{\phi_i} Y \ket{\phi_i} \ge 0$ for each $i$.

Let $\Omega_1=\{X\in \mathbb{C}^{d \times d} : X\ge 0,\ \Phi(X)=B \}$, and $\Omega_2=\{Y\in \mathbb{C}^{d' \times d'} : \Phi^{*}(Y)\ge A,\ Y=Y^\dagger \}$ be the set of $X$ and the set of $Y$ satisfying the constraints in the primal problem and the dual problem.
Now, for any $X\in\Omega_1$ and $Y\in\Omega_2$, we have
\begin{align*}
  \Tr(BY)
  =& \Tr(\Phi(X)Y)\\
  =& \Tr(X \Phi^{*}(Y))\\
  =& \Tr(X (\Phi^{*}(Y) - A)) + \Tr(XA)\\
  \ge& \Tr(X A) \tag{apply the lemma on $X \ge 0, \Phi^{*}(Y) - A \ge 0$}\\
  =& \Tr(AX).
\end{align*}

Therefore, $\beta = \inf_{Y \in \Omega_2}\Tr(BY) \ge \sup_{X \in \Omega_1}\Tr(AX) = \alpha$.

\item For a Hermitian matrix $M \in \mathbb{C}^{d \times d}$, we can find the eigenvalue decomposition of $M$: $M = \sum_{i = 1}^d \mu_i \ketbra{\phi_i}{\phi_i}$.

Since $\{\ket{\phi_i}\}$ forms a basis of $\mathbb{C}^{d}$, we have that $\lambda\Id - M = \sum_{i = 1}^d (\lambda - \mu_i) \ketbra{\phi_i}{\phi_i} $.

Therefore, $M \leq \lambda \Id \Leftrightarrow \lambda\Id - M \ge 0 \Leftrightarrow \forall i, \lambda - \mu_i \ge 0 \Leftrightarrow \forall i, \lambda \ge \mu_i$, which means all eigenvalues of M are less than or equal to $\lambda$.
\item We want $Y$ to act as $\lambda$ in (b), $\Phi^*(\lambda) = \lambda \Id$ and $A = M$ in order to put the condition $\lambda \Id \ge M$ in the dual problem. 

So we choose $d' = 1$, $K_i = (0 \; 0 \; \cdots \; 1 \; 0 \cdots 0) = \bra{i - 1}$ be the $1 \times d$ matrix that has 0 on every entry except the $i\textsuperscript{th}$ entry, and $\nu_i = 1$. Then it is easy to see that $\Phi^*(Y) = \sum_{i = 0}^{d - 1} \ket{i} Y \bra{i} = Y \sum_{i = 0}^{d - 1} \ketbra{i}{i} = Y \Id$ and $\Phi(X) = \sum_{i = 1}^d \bra{i} X \ket{i} = \Tr(X)$.

We choose $A = M$ and $B = 1$, then the dual problem is just
\begin{align}
&\beta\,:=\,  \inf Y \notag\\
s.t.\quad&  Y\Id\geq M,\notag\\
\qquad &Y \in \mathbb{R}.\notag
\end{align}

Using the result of (b), it's easy to see that $\beta = \lambda_1(M)$.

With this choice of $\Phi$, $A$ and $B$, the primal problem is
\begin{align}
&\alpha\,:=\,  \sup \Tr(MX)\notag \\
s.t.\quad& \Tr(X)=1,\notag \\
& X \geq 0.\notag
\end{align}

Recall that from (a), we have that $\alpha \leq \beta$. It remains to prove that $\alpha \geq \beta = \lambda_1(M)$. So we only need to find a feasible $X$ such that $\Tr(MX) = \lambda_1(M)$. 

Let's do the eigenvalue decomposition of $M$: $M = \sum_{i = 1}^d \mu_i \ketbra{\phi_i}{\phi_i}$. Without loss of generality, we assume $\mu_1$ is the largest and thus $\mu_1 = \lambda_1(M)$. 

We set $X = \ketbra{\phi_1}{\phi_1}$. This is a feasible $X$ since it is easy to see that $\Tr(X) = 1$ and $X \ge 0$. Moreover, $\Tr(MX) = \bra{\phi_1}M\ket{\phi_1} = \mu_1 =  \lambda_1(M)$.

Therefore, in this case, its optimum $\alpha = \beta$.
\item We first show that $\|\rho - \sigma\|_{tr} = \max_{0 \leq E \leq \Id}\Tr( (\rho - \sigma) E)$. This is because we can do eigenvalue decomposition of $\rho - \sigma$ to get $\rho - \sigma = \sum_{i = 1}^d \mu_i \ketbra{\phi_i}{\phi_i}$, and then $\max_{0 \leq E \leq \Id}\Tr( (\rho - \sigma) E) = \max_{0 \leq E \leq \Id}\sum_{i = 1}^d \mu_i \bra{\phi_i} E \ket{\phi_i}$ which is at most $\sum_{i: \mu_i > 0}\mu_i$ since $ 0 \leq \bra{\phi_i} E \ket{\phi_i} \leq 1$ for each $i$ (and it is easy to see that there exists $E$ to achieve this). Notice that $\Tr(\rho - \sigma) = 0$ and thus $\sum_i \mu_i = 0$. Therefore, $\max_{0 \leq E \leq \Id}\Tr( (\rho - \sigma) E) = \sum_{i:\mu_i > 0}\mu_i = \frac{1}{2}\sum_i |\mu_i| = \|\rho - \sigma\|_{tr}$.

In order to write it in the primal problem, we need to rewrite the condition that $E \leq \Id$. We let $X_1 = E$ and $X_2 = \Id - E$, then we have that
\begin{align} \label{exercise3: eqn1}
\|\rho - \sigma\|_{tr}
  = \max_{X_1,X_2} \Tr((\rho - \sigma) X_1)\notag\\
\text{s.t. } \quad X_1 \ge 0,\ X_2 \ge 0,\ X_1 + X_2 = \Id.
\end{align}

We group $X_1$ and $X_2$ into the diagonal of a single matrix 
\[
  X = \begin{pmatrix} X_1 & X_3 \\ X_3^\dagger & X_2 \end{pmatrix},
\]
and use the linear map 
\[\Phi\left(\begin{pmatrix} X_1 & X_3 \\ X_3^\dagger & X_2 \end{pmatrix}\right) = X_1 + X_2,\]
which can be also expressed as $\Phi(X) = \nu_1 K_1 X K_1^\dagger + \nu_2 K_2 X K_2^\dagger$ for $K_1 = (\Id \; 0) \in \mathbb{C}^{d \times 2d}$, $K_2 = (0 \; \Id) \in \mathbb{C}^{d \times 2d}$ and $\nu_1 = \nu_2 = 1$.

This allows us to express the conditions $X_1 \ge 0,\ X_2 \ge 0,\ X_1 + X_2 = \Id$ as the standard form condition $X \ge 0$ and $\Phi(X) = B$ where $B = \Id$.

To write the object $\Tr((\rho - \sigma) X_1)$ in terms of $X$, we set 
\[
  A = \begin{pmatrix} \rho - \sigma & 0 \\ 0 & 0 \end{pmatrix},
\]
and it is clear that $\Tr((\rho - \sigma) X_1) = \Tr(AX)$.

We write the primal problem as 
\begin{align}
&\alpha\,:=\,  \sup \Tr(AX)\notag \\
s.t.\quad& \Phi(X)=B,\notag \\
& X \geq 0,\notag
\end{align}
for the above matrices
\[
  A = \begin{pmatrix} \rho - \sigma & 0 \\ 0 & 0 \end{pmatrix},
\]
and $B = \Id$, and the Hermitian preserving linear map
\[\Phi\left(\begin{pmatrix} X_1 & X_3 \\ X_3^\dagger & X_2 \end{pmatrix}\right) = X_1 + X_2.\]

With some effort, we can show $(X_1, X_2)$ satisfying \Cref{exercise3: eqn1} can be mapped to
\[
  X = \begin{pmatrix} X_1 & 0 \\ 0 & X_2 \end{pmatrix},
\] 
which satisfies the condition for the primal problem. Moreover, any matrix $X = \begin{pmatrix} X_1 & X_3 \\ X_3^\dagger & X_2 \end{pmatrix}$ such that $\Phi(X)=B$ and $X \geq 0$ can be mapped to $(X_1, X_2)$ satisfying \Cref{exercise3: eqn1}. Therefore, the above primal problem computes $\|\rho - \sigma\|_{tr}$.

The dual problem is
\begin{align}
&\beta\,:=\,  \inf \Tr(Y) \notag\\
s.t.\quad&  \Phi^*(Y)\geq A,\notag\\
\qquad &Y=Y^\dagger,\notag
\end{align}
where $\Phi^*(Y) = \nu_1 K_1^\dagger Y K_1 + \nu_2 K_2^\dagger Y K_2 = \begin{pmatrix} Y & 0 \\ 0 & Y \end{pmatrix}$ and $A = \begin{pmatrix} \rho - \sigma & 0 \\ 0 & 0 \end{pmatrix}$.

Just as in (c), to prove $\beta = \alpha$, we only need to find a feasible $Y$ such that $\Tr(Y) = \|\rho - \sigma\|_{tr}$.

Recall that $\rho - \sigma$ can be decomposed as $\rho - \sigma = \sum_{i = 1}^d \mu_i \ketbra{\phi_i}{\phi_i}$. We set $Y = \sum_{i: \mu_i \ge 0}\mu_i \ketbra{\phi_i}{\phi_i}$. Then it is easy to see that $\Tr(Y) = \sum_{i: \mu_i \ge 0}\mu_i =  \|\rho - \sigma\|_{tr}$, and $Y = Y^\dagger$. Furthermore, $Y = \sum_{i: \mu_i \ge 0}\mu_i \ketbra{\phi_i}{\phi_i} \geq \sum_{i = 1}^d \mu_i \ketbra{\phi_i}{\phi_i} = \rho - \sigma$ and thus $\Phi^*(Y)\geq A$, which concludes the proof of part (d).
\end{enumerate}
\end{enumerate}


\end{document}






 
















