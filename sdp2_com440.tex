\documentclass[11pt, letterpaper]{article}
%\pdfoutput=1

%%% Basics
\usepackage{fullpage} % letter paper and margins
\usepackage{xspace,xcolor,graphicx,tabularx} %basic enhancements
\usepackage{mathtools,amsthm,amssymb} %basic maths packages
\usepackage{enumitem} %better lists
\setenumerate[0]{label={\normalfont (\roman*)}}


%%% Typesetting and fonts
\usepackage[T1]{fontenc}
\usepackage[UKenglish]{babel}
\usepackage[scaled=0.86]{helvet}
\usepackage{mathptmx}
\DeclareMathAlphabet{\mathcal}{OMS}{ntxm}{m}{n}
\usepackage{dsfont} %double stroke font
\let\mathbb\relax
\let\mathbb\mathds
\usepackage{stmaryrd} %extra symbols
% adjust paragraph spacing slightly
\makeatletter
\renewcommand{\paragraph}{%
  \@startsection{paragraph}{4}%
  {\z@}{2.25ex \@plus 1ex \@minus .2ex}{-1em}%
  {\normalfont\normalsize\bfseries}%
}
\makeatother
\interfootnotelinepenalty=10000


%%% Author list and affiliation
\usepackage{authblk}
\renewcommand*{\Affilfont}{\small}


%%% Links and references
\definecolor{linkblue}{HTML}{001487}
\usepackage[colorlinks=true,allcolors=linkblue]{hyperref}
\usepackage{url}
\usepackage{cleveref}
%\usepackage[nameinlink,capitalize,noabbrev]{cleveref}
%\crefname{enumi}{Step}{Steps}


%%% Theorem environments
\newtheorem{theorem}{Theorem}[section]
\newtheorem*{theorem*}{Theorem}
\newtheorem{proposition}[theorem]{Proposition}
\newtheorem{conjecture}[theorem]{Conjecture}
\newtheorem{lemma}[theorem]{Lemma}
\newtheorem{claim}[theorem]{Claim}
\newtheorem{fact}[theorem]{Fact}
\newtheorem{corollary}[theorem]{Corollary}
\theoremstyle{remark}
\newtheorem{remark}[theorem]{Remark}
\newtheorem{exercise}[theorem]{Exercise}
\theoremstyle{definition}
\newtheorem{definition}[theorem]{Definition}
\newtheorem{example}[theorem]{Example}
\newtheorem{protocol}{Protocol}
\numberwithin{equation}{section}
\newcommand\numberthis{\addtocounter{equation}{1}\tag{\theequation}}


%%% Basic maths abbreviations
%Symbols
\newcommand{\setft}[1]{\textnormal{#1}}
\newcommand{\eps}{\epsilon}
\newcommand{\1}{\mathbb{1}}
\newcommand{\id}{\setft{id}}
\newcommand{\C}{\ensuremath{\mathds{C}}}
\newcommand{\N}{\ensuremath{\mathds{N}}}
\newcommand{\R}{\ensuremath{\mathds{R}}}
\newcommand{\Z}{\ensuremath{\mathds{Z}}}
\newcommand{\F}{\ensuremath{\mathds{F}}}
\newcommand{\bits}{\ensuremath{\{0, 1\}}}

%Operators
\newcommand{\ot}{\ensuremath{\otimes}}
\newcommand{\deq}{\coloneqq}
\newcommand{\Tr}{\mathrm{Tr}}
\newcommand{\tr}{\mathrm{Tr}}
%\newcommand{\tr}[1]{\mathrm{Tr}\!\left[ #1 \right]}
\newcommand{\ptr}[2]{\mathrm{Tr}_{#1}\!\left[ #2 \right]}
\newcommand{\pr}[1]{\mathrm{Pr}\!\left[ #1 \right]}
\newcommand{\prs}[2]{\mathrm{Pr}_{#1}\!\left[ #2 \right]}
\newcommand{\norm}[1]{\left\lVert#1\right\rVert}
\DeclareMathOperator{\pos}{Pos}
\DeclareMathOperator{\poly}{poly}
\DeclareMathOperator{\negl}{negl}
\DeclareMathOperator{\supp}{\setft{supp}}
\DeclareMathOperator{\img}{img}
\DeclareMathOperator{\E}{\mathds{E}}
\def\I{\mathds{1}}
\newcommand{\herm}[1]{\setft{Herm}\left(#1\right)}

%Words
\newcommand{\sth}{{\setft{~s.t.~}}}
\newcommand{\tand}{\;\textnormal{~and~}\;}
\newcommand{\wrt}{\quad {\rm ~~w.r.t.~~}}


%%% Quantum notation
\newcommand{\ket}[1]{|#1\rangle}
\newcommand{\bra}[1]{\langle#1|}
\newcommand{\proj}[1]{\ket{#1}\!\bra{#1}}
\DeclarePairedDelimiterX\braket[2]{\langle}{\rangle}{#1 \delimsize\vert #2}
\newcommand{\cp}{\setft{CP}}
\newcommand{\cptp}{\setft{CPTP}}

\newcommand{\density}[1]{\setft{D}\left(#1\right)}
\newcommand{\lin}[1]{\setft{L}\left(#1\right)}
\newcommand{\reg}[1]{\mathsf{#1}}

\def\X{\mathcal{X}}
\def\Y{\mathcal{Y}}

\bibliographystyle{alpha}

 \newcommand{\Header}[1]{\begin{center} {\Large\bf #1} \end{center}}


\begin{document}


\Header{COM-440, Introduction to Quantum Cryptography, Fall 2025 \\[3mm] \large Security of Wiesner's scheme}


\section{An SDP formulation of simple counterfeiting
  attacks \label{sec:counterfeit-sdp}} 

We describe how the optimal success probability of a counterfeiting
strategy may be represented by a
semidefinite program.

\subsection{A mathematical expression for the optimal counterfeiting strategy}

Mathematically speaking, a simple counterfeiting attack is described
by a quantum channel $\Phi$  with input from $\reg{X}$, a $2$-dimensional register containing the challenger's single-qubit money state,
and output in $(\reg{Y}_1,\reg{Y}_2)$ two $2$-dimensional registers that should contain the two copies. This channel takes a state
$\rho\in\density{\X}$ to a state
$\Phi(\rho)\in\density{\Y_1\otimes\Y_2}$, where $\density{\cdot}$ denotes the set of density matrices on some register.

In order to be physically realizable, at least in an idealized sense,
the channel $\Phi$ must correspond to a completely positive and trace
preserving linear mapping of the form 
$\Phi: \lin{\X}\rightarrow\lin{\Y_1\otimes\Y_2}$.
Conditioned on the bank having chosen the key $k$ and prepared the money state $\ket{\psi_k}$, the probability of
success for an attack described by $\Phi$ is given by
$\bra{\psi_k \otimes \psi_k} \Phi(\ket{\psi_k}\bra{\psi_k})
\ket{\psi_k\otimes\psi_k}$.
Averaging over the possible choices of $k$, 
%upon which $\Phi$ may not depend because the counterfeiter has no
%knowledge of the note's key, 
 the overall success probability of a counterfeiting
attack is
\begin{equation} 
  \label{eq:probability-to-counterfeit}
  \sum_{k = 1}^N p_k
  \bra{\psi_k \otimes \psi_k} \Phi(\ket{\psi_k}\bra{\psi_k})
  \ket{\psi_k\otimes\psi_k}.
\end{equation}
The optimal success probability of a counterfeiting
strategy is then the supremum of the probability
\eqref{eq:probability-to-counterfeit}, taken over all valid channels $\Phi:
\lin{\X}\rightarrow\lin{\Y_1\otimes\Y_2}$.

\subsection{Reformulation using the Choi-Jamiolkowski representation}

Let us see how to model this as a semidefinite program. 
We first choose a standard basis $\{\ket{1},\ldots,\ket{d}\}$,
shared by the spaces $\X$, $\Y_1$, and $\Y_2$.
With respect to this basis, the 
\emph{Choi-Jamio{\l}kowski representation} of a linear mapping
$\Phi:\lin{\X}\rightarrow\lin{\Y_1\otimes\Y_2}$ is defined as
\[
J(\Phi) = \sum_{1\leq i,j\leq d} \Phi(\ket{i}\bra{j}) \otimes
\ket{i}\bra{j}\,\in\,\lin{\Y_1\otimes\Y_2\otimes\X}.
\]
The operator $J(\Phi)$ uniquely determines $\Phi$, and a well-known
necessary and sufficient condition for $J(\Phi)$ to represent a valid 
channel (i.e., a completely positive and trace-preserving map) is that
(i) $J(\Phi)$ is positive semidefinite, and 
(ii) $\tr_{\Y_1\otimes\Y_2}(J(\Phi)) = \I_{\X}$.
It is also  easy to verify that
\[
\bra{\phi} \Phi(\ket{\psi}\bra{\psi}) \ket{\phi}
= \bra{\phi \otimes \overline{\psi}} J(\Phi)\ket{\phi \otimes
  \overline{\psi}}
\]
for any choice of vectors $\ket{\psi}\in\X$ and
$\ket{\phi}\in\Y_1\otimes\Y_2$, with
complex conjugation taken with respect to the standard basis.

\subsection{The semidefinite program}

Combining the observations that have just been made with the
expression \eqref{eq:probability-to-counterfeit}, one finds that the
optimal success probability of any simple counterfeiting strategy is
given by the following semidefinite program:\vspace{2mm}
\begin{center}
\begin{minipage}{0.5\textwidth}
  \centerline{\underline{Primal problem}}\vspace{-6mm}
  \begin{align*}
    \text{maximize:}\;\; & \tr(Q X)\\
    \text{subject to:}\;\; & \tr_{\Y_1\otimes\Y_2}(X) = \I_{\X}\\
    & X\in\pos{\Y_1\otimes\Y_2\otimes\X}
  \end{align*}
\end{minipage}\hspace*{5mm}
\begin{minipage}{0.4\textwidth}
  \centerline{\underline{Dual problem}}\vspace{-6mm}
  \begin{align*}
    \text{minimize:}\;\; & \tr(Y)\\
    \text{subject to:}\;\; & \I_{\Y_1\otimes\Y_2}\otimes Y \geq Q\\
    & Y \in \herm{\X}
  \end{align*}
\end{minipage}
\end{center}
where
\[
Q = \sum_{k = 1}^N p_k
\ket{\psi_k \otimes \psi_k \otimes \overline{\psi_k}}
\bra{\psi_k \otimes \psi_k \otimes \overline{\psi_k}}.
\]

The dual problem is obtained from the primal problem in a routine way,
as described in the notes from last week. 
It may be verified that for this particular semidefinite program, the
optimal values for the primal and dual problems are always equal, and
are both achieved by feasible choices for $X$ and $Y$.

\subsection{Analysis of Wiesner's original scheme}

We can now determine the optimal success probability of a
counterfeiting attack against Wiesner's original scheme, by
considering just the single-qubit scheme given by the ensemble 
$\mathcal{E} = 
\left\{
\left(\frac{1}{4}, \, \ket{0}\right),
\left(\frac{1}{4}, \, \ket{1}\right),f
\left(\frac{1}{4}, \, \ket{+}\right),
\left(\frac{1}{4}, \, \ket{-}\right)
\right\}$.
This ensemble yields the operator 
\[
Q =  
\frac{1}{4} \left( \ket{000} \bra{000} + \ket{111}\bra{111} +  
\ket{+++} \bra{+++}  + \ket{---} \bra{---}   \right)
\]
in the semidefinite programming formulation.
We claim that the optimal value of the semidefinite program in this
case is equal to 3/4. To
show this we exhibit explicit primal and dual feasible
solutions achieving the value $3/4$. For the primal problem
 the value 3/4 is obtained by the solution
$X = J(\Phi)$ for $\Phi$ being the channel
\setlength{\arraycolsep}{4pt}
%\[
% X = 
%\frac{1}{12}
% \begin{pmatrix}
%    9 & 0 & 0 & 3 & 0 & 3 & 3 & 0 \\[1mm]
%    0 & 1 & 1 & 0 & 1 & 0 & 0 & 3 \\[1mm]
%    0 & 1 & 1 & 0 & 1 & 0 & 0 & 3 \\[1mm]
%    3 & 0 & 0 & 1 & 0 & 1 & 1 & 0 \\[1mm]
%    0 & 1 & 1 & 0 & 1 & 0 & 0 & 3 \\[1mm]
%    3 & 0 & 0 & 1 & 0 & 1 & 1 & 0 \\[1mm]
%    3 & 0 & 0 & 1 & 0 & 1 & 1 & 0 \\[1mm]
%    0 & 3 & 3 & 0 & 3 & 0 & 0 & 9
%   \end{pmatrix},
%\]
%The channel $\Phi$ represented by this solution is given by
$\Phi(\rho) = A_0 \rho A_0^{\ast} + A_1 \rho A_1^{\ast}$, where
\[
A_0 = \frac{1}{\sqrt{12}}
\begin{pmatrix} 3 & 0 \\ 0 & 1 \\  0 & 1 \\ 1 & 0 \end{pmatrix}
\qquad\text{and}\qquad
A_1 =  \frac{1}{\sqrt{12}}
\begin{pmatrix} 0 & 1 \\ 1 & 0 \\  1 & 0 \\ 0 & 3 \end{pmatrix}.
\]
For the dual problem, the value 3/4 is obtained by the solution
$Y = \frac{3}{8}\I$, whose feasibility may be verified by computing
$\norm{Q} = 3/8$.



\end{document}