\documentclass[12pt]{article}
\usepackage{fullpage}
\usepackage{amssymb,amsmath}

\newtheorem{theorem}{Theorem}

 \newcommand{\Header}[1]{\begin{center} {\Large\bf #1} \end{center}}
 \newcommand{\header}[1]{\begin{center} {\large\bf #1} \end{center}}
\setlength{\parindent}{0.0in}
\setlength{\parskip}{1ex}


%\newif\ifnotes\notestrue
\newif\ifnotes\notesfalse


\usepackage{amsmath,amssymb,amsthm,amsfonts,latexsym,bbm,xspace,graphicx,float,mathtools,epigraph}
\usepackage[backref,colorlinks,citecolor=blue,bookmarks=true]{hyperref}
\usepackage{enumitem,manyfoot,fullpage}
\usepackage{subfig,tikz,framed}
\usepackage{endnotes}
\usepackage{braket}


\usepackage{fullpage}
\usepackage{hyperref}
\usepackage{pdfsync}
\usepackage{microtype}
\usepackage{color}
\usepackage{cleveref}

\newtheorem*{namedtheorem}{\theoremname}
\newcommand{\theoremname}{testing}
\newenvironment{named}[1]{ \renewcommand{\theoremname}{#1} \begin{namedtheorem}} {\end{namedtheorem}}
\newtheorem{lemma}[theorem]{Lemma}
\newtheorem{claim}[theorem]{Claim}
\newtheorem{proposition}[theorem]{Proposition}
\newtheorem{fact}[theorem]{Fact}
\newtheorem{corollary}[theorem]{Corollary}

\theoremstyle{definition}
\newtheorem{definition}[theorem]{Definition}
\newtheorem{remark}[theorem]{Remark}
\newtheorem{observation}[theorem]{Observation}
\newtheorem{notation}[theorem]{Notation}
\newtheorem{example}[theorem]{Example}
\newtheorem{examples}[theorem]{Examples}
\newtheorem{exercise}{Exercise}


\newenvironment{quotenote}{
\begin{quote}
  \footnotesize
\noindent{\bf Note:}}
{\end{quote}
}


% probability and other mathops
\renewcommand{\Pr}{\mathop{\bf Pr\/}}
\newcommand{\E}{\mathop{\bf E\/}}
\newcommand{\Ex}{\mathop{\bf E\/}}
\newcommand{\Var}{\mathop{\bf Var\/}}
\newcommand{\Cov}{\mathop{\bf Cov\/}}
\newcommand{\stddev}{\mathop{\bf stddev\/}}
\newcommand{\littlesum}{\mathop{{\textstyle \sum}}}
\newcommand{\apx}{\mathop{\approx}}

\newcommand{\epr}{\textsc{EPR}}

\newcommand{\Zt}{\ensuremath{\Z_t}}
\newcommand{\Zp}{\ensuremath{\Z_p}}
\newcommand{\Zq}{\ensuremath{\Z_q}}
\newcommand{\ZN}{\ensuremath{\Z_N}}
\newcommand{\Zps}{\ensuremath{\Z_p^*}}
\newcommand{\ZNs}{\ensuremath{\Z_N^*}}
\newcommand{\JN}{\ensuremath{\J_N}}
\newcommand{\QR}{\ensuremath{\mathbb{QR}}}
\newcommand{\QRN}{\ensuremath{\QR_{N}}}
\newcommand{\QRp}{\ensuremath{\QR_{p}}}

% mathrm terms
\newcommand{\poly}{\mathrm{poly}}
\newcommand{\negl}{\mathrm{negl}}
\newcommand{\Tr}{\mathrm{Tr}}
\newcommand{\polylog}{\mathrm{polylog}}
\newcommand{\size}{\mathrm{size}}
\newcommand{\avg}{\mathop{\mathrm{avg}}}
\newcommand{\sgn}{\mathrm{sgn}}
\newcommand{\dist}{\mathrm{dist}}
\newcommand{\spn}{\mathrm{span}}
\newcommand{\supp}{\mathrm{supp}}
\newcommand{\Val}{\mathrm{Val}}
\newcommand{\Opt}{\mathrm{Opt}}
\newcommand{\LPOpt}{\mathrm{LPOpt}}
\newcommand{\SDPOpt}{\mathrm{SDPOpt}}
\newcommand{\vol}{\mathrm{vol}}
\newcommand{\Id}{\mathbb{I}}

% number systems
\newcommand{\R}{\mathbbm R}
\newcommand{\C}{\mathbbm C}
\newcommand{\N}{\mathbbm N}
\newcommand{\Z}{\mathbbm Z}
\newcommand{\F}{\mathbbm F}
\newcommand{\Q}{\mathbbm Q}

\newcommand{\mH}{\mathcal{H}}

% complexity classes
\newcommand{\PTIME}{\mathsf{P}}
\newcommand{\NP}{\mathsf{NP}} \newcommand{\np}{\NP}

% short forms
\newcommand{\eps}{\varepsilon}
\newcommand{\lam}{\lambda}
\newcommand{\vphi}{\varphi}
\newcommand{\la}{\langle}
\newcommand{\ra}{\rangle}
\newcommand{\wt}[1]{\widetilde{#1}}
\newcommand{\wh}[1]{\widehat{#1}}
\newcommand{\ul}[1]{\underline{#1}}
\newcommand{\ol}[1]{\overline{#1}}
\newcommand{\ot}{\otimes}
\newcommand{\Ra}{\Rightarrow}
\newcommand{\half}{\tfrac{1}{2}}
\newcommand{\grad}{\nabla}
\newcommand{\sse}{\subseteq}


% calligraphic letters
\newcommand{\calA}{\mathcal{A}}
\newcommand{\calB}{\mathcal{B}}
\newcommand{\calC}{\mathcal{C}}
\newcommand{\calD}{\mathcal{D}}
\newcommand{\calE}{\mathcal{E}}
\newcommand{\calF}{\mathcal{F}}
\newcommand{\calG}{\mathcal{G}}
\newcommand{\calH}{\mathcal{G}}
\newcommand{\calI}{\mathcal{I}}
\newcommand{\calJ}{\mathcal{J}}
\newcommand{\calK}{\mathcal{K}}
\newcommand{\calL}{\mathcal{L}}
\newcommand{\calM}{\mathcal{M}}
\newcommand{\calN}{\mathcal{N}}
\newcommand{\calO}{\mathcal{O}}
\newcommand{\calP}{\mathcal{P}}
\newcommand{\calQ}{\mathcal{Q}}
\newcommand{\calR}{\mathcal{R}}
\newcommand{\calS}{\mathcal{S}}
\newcommand{\calT}{\mathcal{T}}
\newcommand{\calU}{\mathcal{U}}
\newcommand{\calV}{\mathcal{V}}
\newcommand{\calW}{\mathcal{W}}
\newcommand{\calX}{\mathcal{X}}
\newcommand{\calY}{\mathcal{Y}}
\newcommand{\calZ}{\mathcal{Z}}


\newcommand{\myfig}[4]{\begin{figure}[H] \begin{center} \includegraphics[width=#1\textwidth]{#2} \caption{#3} \label{#4} \end{center} \end{figure}} 

\newcommand{\bit}{\ensuremath{\{0,1\}}}

%%% CRYPTO-RELATED NOTATION

% length of a string
\newcommand{\len}[1]{\lvert{#1}\rvert}
\newcommand{\lenfit}[1]{\left\lvert{#1}\right\rvert}
% length of some vector, element
\newcommand{\length}[1]{\lVert{#1}\rVert}
\newcommand{\lengthfit}[1]{\left\lVert{#1}\right\rVert}


% types of indistinguishability
\newcommand{\compind}{\ensuremath{\stackrel{c}{\approx}}}
\newcommand{\statind}{\ensuremath{\stackrel{s}{\approx}}}
\newcommand{\perfind}{\ensuremath{\equiv}}

% font for general-purpose algorithms
\newcommand{\algo}[1]{\ensuremath{\mathsf{#1}}}
% font for general-purpose computational problems
\newcommand{\problem}[1]{\ensuremath{\mathsf{#1}}}
% font for complexity classes
\newcommand{\class}[1]{\ensuremath{\mathsf{#1}}}


% KEYS AND RELATED

\newcommand{\key}[1]{\ensuremath{#1}}

\newcommand{\pk}{\key{pk}}
\newcommand{\vk}{\key{vk}}
\newcommand{\sk}{\key{sk}}
\newcommand{\mpk}{\key{mpk}}
\newcommand{\msk}{\key{msk}}
\newcommand{\fk}{\key{fk}}
\newcommand{\id}{id}
\newcommand{\keyspace}{\ensuremath{\mathcal{K}}}
\newcommand{\msgspace}{\ensuremath{\mathcal{M}}}
\newcommand{\ctspace}{\ensuremath{\mathcal{C}}}
\newcommand{\tagspace}{\ensuremath{\mathcal{T}}}
\newcommand{\idspace}{\ensuremath{\mathcal{ID}}}

\newcommand{\concat}{\ensuremath{\|}}

% GAMES

% advantage
\newcommand{\advan}{\ensuremath{\mathbf{Adv}}}

% different attack models
\newcommand{\attack}[1]{\ensuremath{\text{#1}}}

\newcommand{\atk}{\attack{atk}} % dummy attack
\newcommand{\indcpa}{\attack{ind-cpa}}
\newcommand{\indcca}{\attack{ind-cca}}
\newcommand{\anocpa}{\attack{ano-cpa}} % anonymous
\newcommand{\anocca}{\attack{ano-cca}}
\newcommand{\euacma}{\attack{eu-acma}} % forgery: adaptive chosen-message
\newcommand{\euscma}{\attack{eu-scma}} % forgery: static chosen-message
\newcommand{\suacma}{\attack{su-acma}} % strongly unforgeable

% ADVERSARIES
\newcommand{\attacker}[1]{\ensuremath{\mathcal{#1}}}

\newcommand{\Adv}{\attacker{A}}
\newcommand{\AdvA}{\attacker{A}}
\newcommand{\AdvB}{\attacker{B}}
\newcommand{\Dist}{\attacker{D}}
\newcommand{\Sim}{\attacker{S}}
\newcommand{\Ora}{\attacker{O}}
\newcommand{\Inv}{\attacker{I}}
\newcommand{\For}{\attacker{F}}

% CRYPTO SCHEMES

\newcommand{\scheme}[1]{\ensuremath{\text{#1}}}

% pseudorandom stuff
\newcommand{\prg}{\algo{PRG}}
\newcommand{\prf}{\algo{PRF}}
\newcommand{\prp}{\algo{PRP}}

% symmetric-key cryptosystem
\newcommand{\skc}{\scheme{SKC}}
\newcommand{\skcgen}{\algo{Gen}}
\newcommand{\skcenc}{\algo{Enc}}
\newcommand{\skcdec}{\algo{Dec}}

% public-key cryptosystem
\newcommand{\pkc}{\scheme{PKC}}
\newcommand{\pkcgen}{\algo{Gen}}
\newcommand{\pkcenc}{\algo{Enc}} % can also use \kemenc and \kemdec
\newcommand{\pkcdec}{\algo{Dec}}

% digital signatures
\newcommand{\sig}{\scheme{SIG}}
\newcommand{\siggen}{\algo{Gen}}
\newcommand{\sigsign}{\algo{Sign}}
\newcommand{\sigver}{\algo{Ver}}

% message authentication code
\newcommand{\mac}{\scheme{MAC}}
\newcommand{\macgen}{\algo{Gen}}
\newcommand{\mactag}{\algo{Tag}}
\newcommand{\macver}{\algo{Ver}}

% key-encapsulation mechanism
\newcommand{\kem}{\scheme{KEM}}
\newcommand{\kemgen}{\algo{Gen}}
\newcommand{\kemenc}{\algo{Encaps}}
\newcommand{\kemdec}{\algo{Decaps}}

% identity-based encryption
\newcommand{\ibe}{\scheme{IBE}}
\newcommand{\ibesetup}{\algo{Setup}}
\newcommand{\ibeext}{\algo{Ext}}
\newcommand{\ibeenc}{\algo{Enc}}
\newcommand{\ibedec}{\algo{Dec}}

% hierarchical IBE (as key encapsulation)
\newcommand{\hibe}{\scheme{HIBE}}
\newcommand{\hibesetup}{\algo{Setup}}
\newcommand{\hibeext}{\algo{Extract}}
\newcommand{\hibeenc}{\algo{Encaps}}
\newcommand{\hibedec}{\algo{Decaps}}

% binary tree encryption (as key encapsulation)
\newcommand{\bte}{\scheme{BTE}}
\newcommand{\btesetup}{\algo{Setup}}
\newcommand{\bteext}{\algo{Extract}}
\newcommand{\bteenc}{\algo{Encaps}}
\newcommand{\btedec}{\algo{Decaps}}

% trapdoor functions
\newcommand{\tdf}{\scheme{TDF}}
\newcommand{\tdfgen}{\algo{Gen}}
\newcommand{\tdfeval}{\algo{Eval}}
\newcommand{\tdfinv}{\algo{Invert}}
\newcommand{\tdfver}{\algo{Ver}}

%%% PROTOCOLS

\newcommand{\out}{\text{out}}
\newcommand{\view}{\text{view}}


\newcommand{\proj}[1]{\ket{#1}\!\bra{#1}}




%%%%%%%%%%%% document-writing macros %%%%%%%%%%%%

\ifnotes
\usepackage{color}
\definecolor{mygrey}{gray}{0.50}
\newcommand{\notename}[2]{{\textcolor{mygrey}{\footnotesize{\bf (#1:} {#2}{\bf ) }}}}
\newcommand{\noteswarning}{{\begin{center} {\Large WARNING: NOTES ON}\endnote{Warning: notes on}\end{center}}}

\else

\newcommand{\notename}[2]{{}}
\newcommand{\noteswarning}{{}}
\newcommand{\notesendofpaper}{}

\fi

\newcommand{\tnote}[1]{{\notename{Thomas}{#1}}}
\newcommand{\note}[1]{{\notename{Note}{#1}}}
%\newcommand{\note}[1]{}


\bibliographystyle{alpha}



\begin{document}
\header{COM-440, Introduction to Quantum Cryptography, Fall 2025}
{\bf Homework \# 2 Solutions} 


\medskip

\hrule

 
\medskip
{\bf Problems:}

\begin{enumerate}
		
\item {\bf A three-player game}
\begin{enumerate}
\item First we notice that if each player always answers $0$ then they succeed with probability $\frac34$. Next we verify that this is optimal. It suffices to consider deterministic strategies, since for every random strategy there is a deterministic one whose winning probability is at least as high (fix the best choice of random bits for everyone). If $f_A:\{0,1\}\to\{0,1\}$ is Alice's strategy (i.e.\ Alice answers $f_A(x)$ to $x$) and similarly $f_B,f_C$ for Bob and Charlie then the chance that they answer correctly the question $(0,0,0)$ is $\frac{1}{2}+\frac12 (-1)^{f_A(0)}(-1)^{f_B(0)}(-1)^{f_C(0)}$, as can be verified by direct calculation. A similar expression is obtained for the other questions. Letting $a_x=(-1)^{f_A(x)}$ and similarly $b_y,c_z$ we deduce that the winning probability is 
\[\frac12+\frac18\big( a_0b_0c_0 -a_0b_1c_1 - a_1b_0c_1 - a_1 b_1 c_0\big)\;.\]
Since the product of the four quantities $a_0b_0c_0$, etc., is $1$ (each variable appears exactly twice), exactly $0$, $2$ or $4$ of them must equal $-1$. In all cases the entire expression is at most $\frac34$, as claimed.  
\item One needed to verify that 
\[ \bra{\psi} Z\otimes Z \otimes Z \ket{\psi}=\bra{\psi} Z\otimes X \otimes X \ket{\psi}=\bra{\psi} X\otimes Z \otimes X \ket{\psi}=\bra{\psi} X\otimes X \otimes Z \ket{\psi}=1\;.\]
Everyone did this calculation correctly. 
\item To study entanglement between Alice and Bob, we first compute the reduced density matrix between them. We obtain 
\[ \rho_{AB} = \frac14\begin{pmatrix} 1 & 0 & 0 & -1 \\ 0 & 1 & 1 & 0 \\ 0 & 1 & 1 & 0 \\ -1 & 0 & 0 & 1 \end{pmatrix}\;.\]
At this point there are different ways to proceed. The shortest solution is to compute the rank of $\rho_{AB}$ and verify that it is $2$, not $1$. Therefore the state is not pure, and cannot be maximally entangled. One could go further by showing that  $\rho_{AB} $ is in fact separable. This can be shown by using the PPT criterion, as the partial transpose of $\rho_{AB}$ is positive and this criterion is complete in the $2 \times 2$ qubit case. (Since we hadn't reviewed this criterion in class, I didn't expect you to use it.) Another option was to find an explicit separable decomposition by observing that  $\rho_{AB} = \frac{1}{2}(\Id \otimes\Id + Y\otimes Y)$. From there, one obtains  
\[ \rho_{AB} = \frac12\big(\proj{+i}\otimes \proj{+i}+\proj{-i}\otimes\proj{-i}\big)\;,\]
with $\ket{+i}$, $\ket{-i}$ the eigenstates of the Pauli $Y$ matrix.  

A different solution was to observe that, with respect to the bipartition AB:C, the state is locally unitarily equivalent to an EPR pair, i.e. there is maximal entanglement between (AB) and C. Because A and B are a qubit each, there could not be maximal entanglement between A and B; as then the entanglement with C would have to be zero. 

Finally, since the state $\ket{\psi}$ is invariant under permutation of any two registers, the previous reasoning applies to any pair of players. 
\end{enumerate}



\item {\bf Robustness of GHZ and $W$ States}\label{ex:robust-GHZ}
\begin{enumerate}
\item
	First note that if $\rho = \proj{\psi}$ is pure, then
	\[\Tr\left(\rho \sigma\right) = \Tr\left(\proj{\psi} \sigma\right) = \Tr\left(\bra{\psi} \sigma\ket{\psi}\right) = \bra{\psi} \sigma\ket{\psi}\]
	We can view tracing out part of a state as a consequence of an unknown party, say Bob, measuring that part of the state. In this case, measuring the third qubit in the computational basis (remember that the partial trace is independent of the basis we choose) gives Bob either 0 or 1 with probability a half. The corresponding states on the first two qubits are $\proj{00}$ and $\proj{11}$, respectively. Thus, the remaining state is $\frac{1}{2}\proj{00}+\frac{1}{2}\proj{11}$.
	We have that $\proj{GHZ_2} = \frac{1}{2}\left(\proj{00}+\proj{11}+\ketbra{00}{11}+\ketbra{11}{00} \right)$, which is a pure state. Using the fact that this is a pure state, we find that
	\begin{align*}
		\Tr\big(\ketbra{GHZ_2}{GHZ_2} & \Tr_3\ketbra{GHZ_3}{GHZ_3}\big) \\
		& = \bra{GHZ_2} \Tr_3 \left(\proj{GHZ_3}\right)\ket{GHZ_2}\\
		& = \frac{1}{4}\left(\braket{00}{00}+\braket{11}{11}\right)=\frac{1}{2}
	\end{align*}
\item
	Measuring the third qubit gives a 0 with probability $2/3$, so that the state on the other two qubits is equal to $\proj{W_2}$. If a measurement on the third qubit gives a 1 (with probability $1/3$), the corresponding state on the first two qubits is $\proj{00}$. The state on the first two qubits after tracing out the third qubit is $\Tr_3\left(\proj{W_3}\right) = \frac{2}{3}\proj{W_2}+\frac{1}{3}\proj{00}$.
	Since $\inp{W_2}{00} = 0$, we have that $$\Tr\left(\proj{W_2} \Tr_3\left(\proj{W_3}\right)\right) = \frac{2}{3}$$
\item
	We have that
	\[\Tr_N \left(\proj{GHZ_N}\right) = \frac{1}{2}|\underbrace{00\ldots0}_{N-1}\rangle\langle\underbrace{00\ldots0}_{N-1}|+\frac{1}{2}|\underbrace{11\ldots1}_{N-1}\rangle\langle\underbrace{11\ldots1}_{N-1}|\]
	Using similar reasoning as in the first problem, we find that the overlap is equal to $1/2$. Taking the limit we also get $1/2$.
\item	
	Measuring the last qubit gives a $0$ with probability $\frac{N-1}{N}$, so that the state on the remaining qubits is equal to $\ketbra{W_{N-1}}{W_{N-1}}$. If a measurement on the last qubit gives a 1 (with probability $\frac{1}{N}$), the corresponding state on the remaining qubits is $|\underbrace{00\ldots0}_{N-1}\rangle \langle\underbrace{00\ldots0}_{N-1} |$. The state on the first $N-1$ qubits after tracing out the last qubit is then $\Tr_N\left(\left| W_N \right\rangle \left\langle W_N \right|\right) = \frac{N-1}{N}\ketbra{W_{N-1}}{W_{N-1}}+\frac{1}{N}|\underbrace{00\ldots0}_{N-1}\rangle \langle \underbrace{00\ldots0}_{N-1}|$.
	Since $\langle W_{N-1}|\underbrace{00\ldots}_{N-1}\rangle = 0$, we have that $\Tr\Big(\left| W_{N-1} \right\rangle \left\langle W_{N-1} \right| \Tr_N\left(\left| W_N \right\rangle \left\langle W_N \right|\right)\Big) = \frac{N-1}{N}$. Taking the limit we get $\lim_{N\rightarrow\infty}\frac{N-1}{N} = 1$.
\end{enumerate}

\item {\bf Nonlocal boxes}
\begin{enumerate}
\item Using the equality that $\sum_{a}A_x^{(i), a} = \Id$, we can get that for every set of indices $S \subseteq \{1, 2, \cdots, n\}$,
\begin{align*}
\sum_{a_i: i \in S}p(a_1, \cdots, a_n | x_1, \cdots, x_n) &= \sum_{a_i: i \in S}	\Tr\big( (A_{x_1}^{(1),a_1} \otimes \cdots\otimes A_{x_n}^{(n),a_n}) \rho_{A_1\cdots A_n}\big)\\
&= \Tr\big( (\sum_{a_i: i \in S}A_{x_1}^{(1),a_1} \otimes \cdots\otimes A_{x_n}^{(n),a_n}) \rho_{A_1\cdots A_n}\big)\\
&= \Tr\big( \bigotimes_{i \in S} A_{x_i}^{(i),a_i} \rho_{A_1\cdots A_n} \big)\enspace,
\end{align*}
which does not depend on $x_i$ for $i \notin S$. Therefore, the non-local box is non-signaling.

\textit{A common mistake is only checking marginals on one location.}
\item Everyone did this verification correctly.
\item (U), (PR) and (CH) are non-signaling. (SIG) is signaling.
\item By the winning condition of CHSH game, 
\[\Pr\left[\text{Win}\right] = \frac{1}{4}\sum_{a, b, x, y \text{ s.t. }a \oplus b = x \cdot y}p(a, b| x, y)\enspace.\]

Therefore, the success probabilities of (U), (PR), (CH) and (SIG) in the CHSH game are $\frac{1}{2}$, $1$, $\cos^2 \pi / 8 \approx 0.85$, and $1/4$, respectively.

\item (U) can be implemented using $A_x^a = B_y^b = \Id/2$ for every $x, y, a, b$, and a state $\rho_{AB} = \ketbra{00}{00}$.

(PR) wins the CHSH game with certainty, but the optimum success probability in the CHSH game for a quantum strategy is $\cos^2 \pi/8 < 1$. Thus (PR) cannot be implemented using quantum mechanics.

(CH) can be implemented using the optimal quantum strategy for the CHSH game.

(SIG) is signaling, but by part (a), any quantum non-local box is non-signaling. So (SIG) cannot be implemented using quantum mechanics.

\item By the non-signaling property, $\sum_{a, b} q(a, b, c | x, y, z)$ does not depend on $x, y$. So we can set $p'(c|z) = \sum_{a, b} q(a, b, c | x, y, z)$. We set $q(a, b | x, y) = p(a, b | x, y)$. Now let's show that $q(a, b, c|x, y, z) = p(a, b| x, y) p'(c|z)$.

\textbf{Step 1:} when $a \oplus b \ne x \cdot y$, $\sum_c q(a, b, c|x, y, z) = p(a, b | x, y) = 0$. As each $q(a, b, c|x, y, z) \ge 0$, we can get that when $a \oplus b \ne x \cdot y$, $\forall c, z$, $q(a, b, c|x, y, z) = 0$.

\textbf{Step 2:} by the non-signaling property, $\sum_{b} q(a, b, c | x, y, z) = \sum_{b} q(a, b, c | x, 1 - y, z)$ and $\sum_{a} q(a, b, c | x, y, z) = \sum_{a} q(a, b, c | 1- x, y, z)$. 

As $q(a, b, c|x, y, z) = 0$ whenever $a \oplus b \ne x \cdot y$, the above two equations can be simplified into $q(a, a \oplus (x \cdot y), c | x, y, z) = q(a, a \oplus (x \cdot (1 - y)), c | x, 1 - y, z)$ and $q(b \oplus (x \cdot y), b, c | x, y, z) = q(b \oplus ((1 - x) \cdot y), b, c | 1- x, y, z)$.

Enumerate each $a, b, x, y$ and we can get that $q(a, b, c | x, y, z) = q(a', b', c | x', y', z)$ for every $a, b, c, x, y, z, a', b', x', y'$ satisfying $a \oplus b = x \cdot y$ and $a' \oplus b' = x' \cdot y'$.

\textbf{Step 3:} now let's verify that $q(a, b, c|x, y, z) = p(a, b| x, y) p'(c|z)$. For $a \oplus b \ne x \cdot y$, both sides are 0 and thus are equal. For $a \oplus b = x \cdot y$, $q(a, b, c|x, y, z) = \frac{1}{2}(q(a, b, c|x, y, z) + q(1 - a, 1 - b, c|x, y, z))$ (by step 2), which is exactly $p(a, b| x, y) p'(c|z)$.

Hence any non-signaling tripartite extension of (PR) must have a product form.

\item The optimum success probability achieved by any non-signaling tripartite box in the CHSH game is $3/4$. 

On one hand, the non-signaling tripartite box that always outputs 0 ($p(0, 0, 0 | x, y, z) = 1$ for every $x, y, z$) wins the game with probability $3/4$. 

On the other hand, similar as (d), we can write the success probability as
\begin{align*}
	\Pr\left[\text{Win}\right] =& \frac{1}{24}\left(\sum_{a, b, c, x, y, z \text{ s.t. }a \oplus b = x \cdot y}p(a, b, c| x, y, z) + \sum_{a, b, c, x, y, z \text{ s.t. }b \oplus c = y \cdot z}p(a, b, c| x, y, z)\right.\\
	 &\left.+ \sum_{a, b, c, x, y, z \text{ s.t. }c \oplus a = z \cdot x}p(a, b, c| x, y, z)\right)\enspace,
\end{align*}
which is linear in each $p(a, b, c| x, y, z)$.

Besides, all the constraints for a family of distributions to be a non-local box are linear in $p(a, b, c| x, y, z)$.

Therefore, we can use the linear programming to solve the problem, and the result is $3/4$.

\end{enumerate}


\end{enumerate}

\end{document}






 
















