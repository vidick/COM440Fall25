\documentclass[12pt]{article}
\usepackage{fullpage}
\usepackage{amssymb,amsmath}

\newtheorem{theorem}{Theorem}

 \newcommand{\Header}[1]{\begin{center} {\Large\bf #1} \end{center}}
 \newcommand{\header}[1]{\begin{center} {\large\bf #1} \end{center}}
\setlength{\parindent}{0.0in}
\setlength{\parskip}{1ex}


%\newif\ifnotes\notestrue
\newif\ifnotes\notesfalse


\usepackage{amsmath,amssymb,amsthm,amsfonts,latexsym,bbm,xspace,graphicx,float,mathtools,epigraph}
\usepackage[backref,colorlinks,citecolor=blue,bookmarks=true]{hyperref}
\usepackage{enumitem,manyfoot,fullpage}
\usepackage{subfig,tikz,framed}
\usepackage{endnotes}
\usepackage{braket}


\usepackage{fullpage}
\usepackage{hyperref}
\usepackage{pdfsync}
\usepackage{microtype}
\usepackage{color}
\usepackage{cleveref}

\newtheorem*{namedtheorem}{\theoremname}
\newcommand{\theoremname}{testing}
\newenvironment{named}[1]{ \renewcommand{\theoremname}{#1} \begin{namedtheorem}} {\end{namedtheorem}}
\newtheorem{lemma}[theorem]{Lemma}
\newtheorem{claim}[theorem]{Claim}
\newtheorem{proposition}[theorem]{Proposition}
\newtheorem{fact}[theorem]{Fact}
\newtheorem{corollary}[theorem]{Corollary}

\theoremstyle{definition}
\newtheorem{definition}[theorem]{Definition}
\newtheorem{remark}[theorem]{Remark}
\newtheorem{observation}[theorem]{Observation}
\newtheorem{notation}[theorem]{Notation}
\newtheorem{example}[theorem]{Example}
\newtheorem{examples}[theorem]{Examples}
\newtheorem{exercise}{Exercise}


\newenvironment{quotenote}{
\begin{quote}
  \footnotesize
\noindent{\bf Note:}}
{\end{quote}
}


% probability and other mathops
\renewcommand{\Pr}{\mathop{\bf Pr\/}}
\newcommand{\E}{\mathop{\bf E\/}}
\newcommand{\Ex}{\mathop{\bf E\/}}
\newcommand{\Var}{\mathop{\bf Var\/}}
\newcommand{\Cov}{\mathop{\bf Cov\/}}
\newcommand{\stddev}{\mathop{\bf stddev\/}}
\newcommand{\littlesum}{\mathop{{\textstyle \sum}}}
\newcommand{\apx}{\mathop{\approx}}

\newcommand{\epr}{\textsc{EPR}}

\newcommand{\Zt}{\ensuremath{\Z_t}}
\newcommand{\Zp}{\ensuremath{\Z_p}}
\newcommand{\Zq}{\ensuremath{\Z_q}}
\newcommand{\ZN}{\ensuremath{\Z_N}}
\newcommand{\Zps}{\ensuremath{\Z_p^*}}
\newcommand{\ZNs}{\ensuremath{\Z_N^*}}
\newcommand{\JN}{\ensuremath{\J_N}}
\newcommand{\QR}{\ensuremath{\mathbb{QR}}}
\newcommand{\QRN}{\ensuremath{\QR_{N}}}
\newcommand{\QRp}{\ensuremath{\QR_{p}}}

% mathrm terms
\newcommand{\poly}{\mathrm{poly}}
\newcommand{\negl}{\mathrm{negl}}
\newcommand{\Tr}{\mathrm{Tr}}
\newcommand{\polylog}{\mathrm{polylog}}
\newcommand{\size}{\mathrm{size}}
\newcommand{\avg}{\mathop{\mathrm{avg}}}
\newcommand{\sgn}{\mathrm{sgn}}
\newcommand{\dist}{\mathrm{dist}}
\newcommand{\spn}{\mathrm{span}}
\newcommand{\supp}{\mathrm{supp}}
\newcommand{\Val}{\mathrm{Val}}
\newcommand{\Opt}{\mathrm{Opt}}
\newcommand{\LPOpt}{\mathrm{LPOpt}}
\newcommand{\SDPOpt}{\mathrm{SDPOpt}}
\newcommand{\vol}{\mathrm{vol}}
\newcommand{\Id}{\mathbb{I}}

% number systems
\newcommand{\R}{\mathbbm R}
\newcommand{\C}{\mathbbm C}
\newcommand{\N}{\mathbbm N}
\newcommand{\Z}{\mathbbm Z}
\newcommand{\F}{\mathbbm F}
\newcommand{\Q}{\mathbbm Q}

\newcommand{\mH}{\mathcal{H}}

% complexity classes
\newcommand{\PTIME}{\mathsf{P}}
\newcommand{\NP}{\mathsf{NP}} \newcommand{\np}{\NP}

% short forms
\newcommand{\eps}{\varepsilon}
\newcommand{\lam}{\lambda}
\newcommand{\vphi}{\varphi}
\newcommand{\la}{\langle}
\newcommand{\ra}{\rangle}
\newcommand{\wt}[1]{\widetilde{#1}}
\newcommand{\wh}[1]{\widehat{#1}}
\newcommand{\ul}[1]{\underline{#1}}
\newcommand{\ol}[1]{\overline{#1}}
\newcommand{\ot}{\otimes}
\newcommand{\Ra}{\Rightarrow}
\newcommand{\half}{\tfrac{1}{2}}
\newcommand{\grad}{\nabla}
\newcommand{\sse}{\subseteq}


% calligraphic letters
\newcommand{\calA}{\mathcal{A}}
\newcommand{\calB}{\mathcal{B}}
\newcommand{\calC}{\mathcal{C}}
\newcommand{\calD}{\mathcal{D}}
\newcommand{\calE}{\mathcal{E}}
\newcommand{\calF}{\mathcal{F}}
\newcommand{\calG}{\mathcal{G}}
\newcommand{\calH}{\mathcal{G}}
\newcommand{\calI}{\mathcal{I}}
\newcommand{\calJ}{\mathcal{J}}
\newcommand{\calK}{\mathcal{K}}
\newcommand{\calL}{\mathcal{L}}
\newcommand{\calM}{\mathcal{M}}
\newcommand{\calN}{\mathcal{N}}
\newcommand{\calO}{\mathcal{O}}
\newcommand{\calP}{\mathcal{P}}
\newcommand{\calQ}{\mathcal{Q}}
\newcommand{\calR}{\mathcal{R}}
\newcommand{\calS}{\mathcal{S}}
\newcommand{\calT}{\mathcal{T}}
\newcommand{\calU}{\mathcal{U}}
\newcommand{\calV}{\mathcal{V}}
\newcommand{\calW}{\mathcal{W}}
\newcommand{\calX}{\mathcal{X}}
\newcommand{\calY}{\mathcal{Y}}
\newcommand{\calZ}{\mathcal{Z}}


\newcommand{\myfig}[4]{\begin{figure}[H] \begin{center} \includegraphics[width=#1\textwidth]{#2} \caption{#3} \label{#4} \end{center} \end{figure}} 

\newcommand{\bit}{\ensuremath{\{0,1\}}}

%%% CRYPTO-RELATED NOTATION

% length of a string
\newcommand{\len}[1]{\lvert{#1}\rvert}
\newcommand{\lenfit}[1]{\left\lvert{#1}\right\rvert}
% length of some vector, element
\newcommand{\length}[1]{\lVert{#1}\rVert}
\newcommand{\lengthfit}[1]{\left\lVert{#1}\right\rVert}


% types of indistinguishability
\newcommand{\compind}{\ensuremath{\stackrel{c}{\approx}}}
\newcommand{\statind}{\ensuremath{\stackrel{s}{\approx}}}
\newcommand{\perfind}{\ensuremath{\equiv}}

% font for general-purpose algorithms
\newcommand{\algo}[1]{\ensuremath{\mathsf{#1}}}
% font for general-purpose computational problems
\newcommand{\problem}[1]{\ensuremath{\mathsf{#1}}}
% font for complexity classes
\newcommand{\class}[1]{\ensuremath{\mathsf{#1}}}


% KEYS AND RELATED

\newcommand{\key}[1]{\ensuremath{#1}}

\newcommand{\pk}{\key{pk}}
\newcommand{\vk}{\key{vk}}
\newcommand{\sk}{\key{sk}}
\newcommand{\mpk}{\key{mpk}}
\newcommand{\msk}{\key{msk}}
\newcommand{\fk}{\key{fk}}
\newcommand{\id}{id}
\newcommand{\keyspace}{\ensuremath{\mathcal{K}}}
\newcommand{\msgspace}{\ensuremath{\mathcal{M}}}
\newcommand{\ctspace}{\ensuremath{\mathcal{C}}}
\newcommand{\tagspace}{\ensuremath{\mathcal{T}}}
\newcommand{\idspace}{\ensuremath{\mathcal{ID}}}

\newcommand{\concat}{\ensuremath{\|}}

% GAMES

% advantage
\newcommand{\advan}{\ensuremath{\mathbf{Adv}}}

% different attack models
\newcommand{\attack}[1]{\ensuremath{\text{#1}}}

\newcommand{\atk}{\attack{atk}} % dummy attack
\newcommand{\indcpa}{\attack{ind-cpa}}
\newcommand{\indcca}{\attack{ind-cca}}
\newcommand{\anocpa}{\attack{ano-cpa}} % anonymous
\newcommand{\anocca}{\attack{ano-cca}}
\newcommand{\euacma}{\attack{eu-acma}} % forgery: adaptive chosen-message
\newcommand{\euscma}{\attack{eu-scma}} % forgery: static chosen-message
\newcommand{\suacma}{\attack{su-acma}} % strongly unforgeable

% ADVERSARIES
\newcommand{\attacker}[1]{\ensuremath{\mathcal{#1}}}

\newcommand{\Adv}{\attacker{A}}
\newcommand{\AdvA}{\attacker{A}}
\newcommand{\AdvB}{\attacker{B}}
\newcommand{\Dist}{\attacker{D}}
\newcommand{\Sim}{\attacker{S}}
\newcommand{\Ora}{\attacker{O}}
\newcommand{\Inv}{\attacker{I}}
\newcommand{\For}{\attacker{F}}

% CRYPTO SCHEMES

\newcommand{\scheme}[1]{\ensuremath{\text{#1}}}

% pseudorandom stuff
\newcommand{\prg}{\algo{PRG}}
\newcommand{\prf}{\algo{PRF}}
\newcommand{\prp}{\algo{PRP}}

% symmetric-key cryptosystem
\newcommand{\skc}{\scheme{SKC}}
\newcommand{\skcgen}{\algo{Gen}}
\newcommand{\skcenc}{\algo{Enc}}
\newcommand{\skcdec}{\algo{Dec}}

% public-key cryptosystem
\newcommand{\pkc}{\scheme{PKC}}
\newcommand{\pkcgen}{\algo{Gen}}
\newcommand{\pkcenc}{\algo{Enc}} % can also use \kemenc and \kemdec
\newcommand{\pkcdec}{\algo{Dec}}

% digital signatures
\newcommand{\sig}{\scheme{SIG}}
\newcommand{\siggen}{\algo{Gen}}
\newcommand{\sigsign}{\algo{Sign}}
\newcommand{\sigver}{\algo{Ver}}

% message authentication code
\newcommand{\mac}{\scheme{MAC}}
\newcommand{\macgen}{\algo{Gen}}
\newcommand{\mactag}{\algo{Tag}}
\newcommand{\macver}{\algo{Ver}}

% key-encapsulation mechanism
\newcommand{\kem}{\scheme{KEM}}
\newcommand{\kemgen}{\algo{Gen}}
\newcommand{\kemenc}{\algo{Encaps}}
\newcommand{\kemdec}{\algo{Decaps}}

% identity-based encryption
\newcommand{\ibe}{\scheme{IBE}}
\newcommand{\ibesetup}{\algo{Setup}}
\newcommand{\ibeext}{\algo{Ext}}
\newcommand{\ibeenc}{\algo{Enc}}
\newcommand{\ibedec}{\algo{Dec}}

% hierarchical IBE (as key encapsulation)
\newcommand{\hibe}{\scheme{HIBE}}
\newcommand{\hibesetup}{\algo{Setup}}
\newcommand{\hibeext}{\algo{Extract}}
\newcommand{\hibeenc}{\algo{Encaps}}
\newcommand{\hibedec}{\algo{Decaps}}

% binary tree encryption (as key encapsulation)
\newcommand{\bte}{\scheme{BTE}}
\newcommand{\btesetup}{\algo{Setup}}
\newcommand{\bteext}{\algo{Extract}}
\newcommand{\bteenc}{\algo{Encaps}}
\newcommand{\btedec}{\algo{Decaps}}

% trapdoor functions
\newcommand{\tdf}{\scheme{TDF}}
\newcommand{\tdfgen}{\algo{Gen}}
\newcommand{\tdfeval}{\algo{Eval}}
\newcommand{\tdfinv}{\algo{Invert}}
\newcommand{\tdfver}{\algo{Ver}}

%%% PROTOCOLS

\newcommand{\out}{\text{out}}
\newcommand{\view}{\text{view}}


\newcommand{\proj}[1]{\ket{#1}\!\bra{#1}}




%%%%%%%%%%%% document-writing macros %%%%%%%%%%%%

\ifnotes
\usepackage{color}
\definecolor{mygrey}{gray}{0.50}
\newcommand{\notename}[2]{{\textcolor{mygrey}{\footnotesize{\bf (#1:} {#2}{\bf ) }}}}
\newcommand{\noteswarning}{{\begin{center} {\Large WARNING: NOTES ON}\endnote{Warning: notes on}\end{center}}}

\else

\newcommand{\notename}[2]{{}}
\newcommand{\noteswarning}{{}}
\newcommand{\notesendofpaper}{}

\fi

\newcommand{\tnote}[1]{{\notename{Thomas}{#1}}}
\newcommand{\note}[1]{{\notename{Note}{#1}}}
%\newcommand{\note}[1]{}


\bibliographystyle{alpha}



\begin{document}
\header{COM-440, Introduction to Quantum Cryptography, Fall 2025}
{\bf Homework \# 3} \hfill {\bf due: 12:59PM, {November 18th}, 2025}


\medskip

\hrule

\medskip 

Ground rules: 

{\bf Please
  format your solutions so that each problem begins on a new page, and
  so that your name appears at the top of each page.}

You are encouraged to collaborate with your classmates on
homework problems, but each person must write up the final solutions
individually. You should note on your homework specifically which
problems were a collaborative effort and with whom. You may not search
online for solutions, but if you do use research papers or other
sources in your solutions, you must cite them. If you use an AI tool to 
partially solve a question, or improve your write-up of a solution, then 
you should also mention it. All questions on the homework, including requests for clarifications or typos, should be directed to the Ed forum.

Late homework will not be accepted or graded. Extensions will not be granted, except on the recommendation of an academic committee. 

Each of the four problem set counts for 20\% of the total homework grade. (Each of the two readings will count for 10\% of the final homework grade.)


\medskip

\hrule

\medskip



Revisions since the first posting are in \textcolor{blue}{blue}.

{\bf Problems:}
\begin{enumerate}



\item \emph{(6 points)} {\bf Information reconciliation via linear codes.}
Suppose Alice and Bob have access to the the binary symmetric channel with
error $p$: Bob receives each bit that Alice sends correctly with probability $(1-p)$, and incorrectly with probability $p$. Suppose that Alice selects $7$ bits uniformly at random, $X_A \in \{0,1\}^7$, and sends them to Bob. Then, Bob has $X_B \in \{0,1\}^7$ such that each bit of $X_B$ equals the same bit of $X_A$ with probability $(1-p)$. 
\begin{enumerate}
\item 
Consider the linear code generated by the parity check matrix
\[H = \begin{pmatrix} 0 & 0 & 0 & 1 & 1 & 1 & 1 \\
0 & 1 & 1 & 0 & 0 & 1 & 1 \\
1 & 0 & 1 & 0 & 1 & 0 & 1
\end{pmatrix}\;.\]
Show that this code corrects a single error: for any codeword $c\in \ker H$ and any error  $e\in \{0,1\}^7$ of Hamming weight $1$, $c$ can be uniquely recovered from $c+e$ (in other words, there are no $c,c'$ and errors $e,e'$ such that $c+e=c'+e'$.
\item
Using the information reconciliation scheme from class (described in the Chapter 6 notes), with what probability (as a function of $p$) do Alice and Bob succeed in performing reconciliation on their strings (thus obtaining an identical $7$-bit key)?
\item What is the probability that a $7$-bit message is transmitted correctly with no
reconciliation? Compare this to the success probability of the previous part and
to the success probability of the $3$-bit scheme generated by the parity check matrix
\[H = \begin{pmatrix} 1 & 1 & 0 \\ 0 & 1 & 1\end{pmatrix}\;.\]
Which scheme has success probability with the best leading order
behavior? For $p\in \big(0,\frac{1}{2}\big)$, which scheme is best?
\end{enumerate}

\item \emph{(6 points)} {\bf Establishing keys in the presence of a limited eavesdropper.}
In all settings below, we assume that Alice and Bob are connected by a classical authenticated channel. 
Your goal is to devise ways in which Alice and Bob can obtain a key in any of the situations below. \emph{[Hint: In both cases, start by evaluating $\Hmin(X|E)$ where $X$ is the string shared by Alice and Bob after transmission and $E$ is Eve's (classical) information about it.]}
\begin{enumerate}
\item[(a)] Suppose that Alice and Bob are connected by a classical channel such that Eve learns each bit with probability exactly $q$, where we only know that
$1/3 \leq q \leq 1/2$. (Bob always receives the correct bit.)
Give a protocol that allows Alice and Bob to create an $\epsilon$-secure key, where $\epsilon = 10^{-5}$. Explain why your protocol is secure. How many uses of the channel are required per bit of key produced? 
\item[(b)] Suppose now that Alice and Bob are connected by a classical channel on which Eve can intercept bits arbitrarily. However, Eve's memory 
is limited to $k = 1024$ bits. (Bob always receives the correct bit.) Give a protocol that allows Alice and Bob to create an $\epsilon$-secure key where $\epsilon = 10^{-10}$. 
Explain why your protocol is secure.
\end{enumerate}


\item \emph{(8 points)} {\bf Min-entropy from the matching outcomes bound}\label{ex:matching-min}\\
This problem investigates a direct method to lower bound Alice and Bob's key extraction rate based on the probability that the matching-outcomes test succeeds. If we assume that the adversary Eve prepares $n$ identical and uncorrelated copies of the tripartite state $\ket{\Psi_{ABE}}$ and sends the qubits $A$ to Alice and $B$ to Bob, then as shown in Chapter 4 the key extraction rate can be asymptotically lower-bounded by the min-entropy $\Hmin(X|E)$ per round, where $X$ is the outcome of Alice's measurement on her qubit. The goal of this problem is to prove a lower bound on this quantity.

Recall that if Alice measures her qubit in the standard basis, and the resulting post-measurement state on her qubit and Eve's system $E$ is a classical-quantum (cq) state
$$\rho_{XE} = \frac{1}{2}\ketbra{0}{0} \otimes \rho_E^{Z,0}+\frac{1}{2}\ketbra{1}{1} \otimes \rho_E^{Z,1},$$
then since $X$ consists of a single bit as we saw in class the optimal guessing probability $P_{guess}(X|E)$ such that
$$\Hmin(X|E) = - \log P_{guess}(X|E)$$
is given by the optimal distinguishing measurement, for which
$P_{guess}(X|E) = \frac{1}{2}+\frac{1}{4}\|\rho_E^{Z,0}-\rho_E^{Z,1}\|_1.$

The same reasoning holds for any other choice of Alice's basis, notably the Hadamard basis $\{|+\rangle,|-\rangle\}$. In the BB'84 protocol Alice chooses with probability 1/2 one of the two bases in which to measure her qubit. If we denote by $P_{guess}(X|E,\Theta=X)$ and $P_{guess}(X|E,\Theta=1)$ the optimal guessing probabilities for Alice measuring in the standard ($\Theta=0$) and Hadamard ($\Theta=1$) bases respectively, the desired lower bound is given by
$$\Hmin(X|E\Theta) = -\log \left[\frac{1}{2}P_{guess}(X|E,\Theta=0)+\frac{1}{2}P_{guess}(X|E,\Theta=1)\right].$$
\begin{enumerate}
\item Suppose Alice and Bob share a pure EPR pair $\ket{\epr}$, uncorrelated with Eve's system: $\rho_{ABE} = \proj{\epr}_{AB} \otimes \rho_E$. What is $\Hmin(X|E)$?
%\solopen{Recall that $X$ is a single bit, the output of Alice's measurement. Therefore its min-entropy is upper-bounded by $1$. Since Eve is totally uncorrelated and Alice's local density matrix is $\frac12\id$, $X$ is a uniform random bit with respect to Eve's point of view. That is, $\Hmin(X\mid E) = 1$. }
%
% \begin{enumerate}[a)]
%     \item $\frac{1}{2}$
%     \item $\frac{1}{\sqrt{2}}$
%     \item $1$
%     \item $2$
% \end{enumerate}
%
\item Now consider the general case, where $\ket{\Psi_{ABE}}$ is an arbitrary state prepared by Eve. Let $p$ be the probability that this state succeeds in the matching outcomes test, when Alice and Bob both measure in the same basis $\Theta$ chosen at random. Give coefficients $a,b,c$ such that
$$p = a \bra{\Psi_{ABE}} X_A \otimes X_B \otimes \id_E \ket{\Psi_{ABE}} + b\bra{\Psi_{ABE}} Z_A \otimes Z_B \otimes \id_E \ket{\Psi_{ABE}} + c,$$
where $X,Z$ are the Pauli observables $X = \ketbra{0}{1} + \ketbra{1}{0}$ and $Z = \ketbra{+}{-} + \ketbra{-}{+}$.
%\solopen{We want to compute the probability that Alice and Bob obtain the same outcome. Suppose $\Theta = 0$, so that Alice and Bob both measure in the standard basis.
%The expected value of the product of their measurement outcomes is $\bra{\Psi_{ABE}} Z\otimes Z \otimes I \ket{\Psi_{ABE}}$. If they make the same measurement, then the product of their measurements is $+1$. If they make different measurements, this product is $-1$.
%Therefore the expectation is equal to $p_{\text{same}\mid\Theta = 0} - p_{\text{different}\mid\Theta = 0}$. These two probabilities sum to $1$, so we conclude that
%\[p_{\text{same}\mid\Theta = 0} = \frac12 + \frac 12\bra{\Psi_{ABE}} Z\otimes Z \otimes I \ket{\Psi_{ABE}}.\]
%Similarly,
%\[p_{\text{same}\mid\Theta = 1} = \frac12 + \frac 12\bra{\Psi_{ABE}} X\otimes X \otimes I \ket{\Psi_{ABE}}.\]
%Finally,
%\[
%p_{\text{same}} = \frac12 p_{\text{same}\mid\Theta = 0} + \frac 12 p_{\text{same}\mid\Theta = 0}
%= \frac12 + \frac 14\bra{\Psi_{ABE}} X\otimes X \otimes I \ket{\Psi_{ABE}}+ \frac 14\bra{\Psi_{ABE}} Z\otimes Z \otimes I \ket{\Psi_{ABE}}.
%\]}
\item Let $p_X$ (resp. $p_Z$) be the probability that the state $\ket{\Psi_{ABE}}$ passes the matching outcomes test in the Hadamard (resp. computational) basis, so that $p=\frac{1}{2}(p_X+p_Z)$.
By expanding the qubit $A$ in the computational basis, the state $\ket{\Psi_{ABE}}$ can be expressed as $\ket{\Psi_{ABE}} = \ket{0}\otimes\ket{u_0}_{BE} + \ket{1}\otimes\ket{u_1}_{BE}$, with $\|\ket{u_0}_{BE}\|^2+\|\ket{u_1}_{BE}\|^2=1$. Give coefficients $a',b'$ such that
$ \bra{\Psi_{ABE}} X_A \otimes X_B \otimes \id_E \ket{\Psi_{ABE}} = a'\, \Re(\bra{u_0} X_B \otimes \id_E \ket{u_1}) + b'.$
%\solopen{
%Applying the standard basis expansion suggested in the problem, along with the identity $X = \ketbra 01 + \ketbra 10$, we can rewrite $\bra{\Psi_{ABE}} X_A\otimes X_B \otimes I_E \ket{\Psi_{ABE}}$ as
%\[
%\left(\bra 0_A \otimes \bra{u_0}_{BE} +\bra 1_A \otimes \bra{u_1}_{BE}\right) \left[(\ketbra01 + \ketbra10)_A\otimes X_B\otimes I_E \right]\left(\ket 0_A\otimes \ket{u_0}_{BE} +\ket 1_A \otimes \ket{u_1}_{BE}\right).
%\]
%Evaluating inner products simplifies this to
%\[
%\bra{u_0}_{BE}X_B\otimes I_E\ket{u_1}_{BE} +
%\bra{u_1}_{BE}X_B\otimes I_E\ket{u_0}_{BE}.
%\]
%Since the two terms in the above some are complex conjugates of each other, their sum is equal to twice their real part. That is,
%\[
%\bra{\Psi_{ABE}} X_A\otimes X_B \otimes I_E \ket{\Psi_{ABE}}
%=2\Re\bra{u_0}_{BE}X_B\otimes I_E\ket{u_1}_{BE}
%=2\Re\bra{u_1}_{BE}X_B\otimes I_E\ket{u_0}_{BE}.
%\]
%}
\end{enumerate}
A similar bound can be obtained for $p_Z$.

Suppose Alice measures her qubit in the computational basis: the post-measurement state on $A$ and $E$ (tracing out $B$) can be written as $\rho_{AE}^Z = \ketbra{0}{0}_A\otimes \sigma_E^{Z,0} + \ketbra{1}{1}_A\otimes \sigma_E^{Z,1}$. Similarly, if Alice measures in the Hadamard basis we may write the post-measurement state as $\rho_{AE}^X = \ketbra{+}{+}_A\otimes \sigma_E^{X,+} + \ketbra{-}{-}_A\otimes \sigma_E^{X,-}$.
\begin{enumerate}
\item[(d)] Use the previous two questions to determine coefficients $\alpha,\beta$ such that
$$ 2p-1 \leq \alpha F(\sigma_E^{X,0},\sigma_E^{X,1}) + \beta F(\sigma_E^{Z,+},\sigma_E^{Z,-})$$
where $F$ denotes the fidelity. [Hint: observe that $\ket{u_0}_{BE}$ and $\ket{u_1}_{BE}$ considered in the previous question are purifications of $\sigma_E^{Z,0}$ and $\sigma_E^{Z,1}$ respectively, and use Uhlmann's theorem]
%\solopen{First, observe that by part b),
%\[
% 2p-1
% = \frac 12\bra{\Psi_{ABE}} X\otimes X \otimes I \ket{\Psi_{ABE}}+ \frac 12\bra{\Psi_{ABE}} Z\otimes Z \otimes I \ket{\Psi_{ABE}}.
% \]
%By part c),
%\[
% 2p-1
% =\Re\bra{u_0}X_B\otimes I_E\ket{u_1}
% +\Re\bra{u_0}Z_B\otimes I_E\ket{u_1}.
% \]
% By Ulhmann's theorem,
% \[
%\abs{\bra{u_0}X_B\otimes I_E\ket{u_1}} \leq F(\sigma_E^{X,0},\sigma_E^{X,1}) \text{ and }
%\abs{\bra{u_0}Z_B\otimes I_E\ket{u_1}} \leq F(\sigma_E^{Z,+},\sigma_E^{Z,-}).
% \]
% Finally, notice that for any complex $z$, $\Re z \leq \abs z$. Chaining and adding inequalities gives the desired inequality with $\alpha = \beta = 1$.
% \[
%|2p-1| \leq  F(\sigma_E^{X,0},\sigma_E^{X,1}) +  F(\sigma_E^{Z,+},\sigma_E^{Z,-}).
% \]
%}
\item[(e)] Recall the inequality $\|\rho-\sigma\|_{tr} \leq \sqrt{1-F(\rho,\sigma)^2}$. Using also the definition of $\Hmin (X|E)$, what is the best lower bound on $\Hmin(X|E)$ as a function of $p$ that you can get?
%\solopen{Recall that from part d) we have
%\begin{equation}
%2p-1 \leq \alpha F(\sigma_E^{X,0},\sigma_E^{X,1}) + \beta F(\sigma_E^{Z,+},\sigma_E^{Z,-})
%\end{equation}
%Now, $D(\rho,\sigma) \leq \sqrt{1-F(\rho,\sigma)^2} \Rightarrow F(\rho,\sigma) \leq \sqrt{1-D(\rho,\sigma)^2}$. Hence, we have
%\begin{equation}
%2p-1 \leq \sqrt{1-D(\sigma_E^{X,0},\sigma_E^{X,1})^2} + \sqrt{1-D(\sigma_E^{Z,+},\sigma_E^{Z,-})^2}
%\end{equation}
%Now, from the discussion at the start of the problem, we have the following lower bound on $\Hmin (X|E)$.
%\begin{equation}
%\Hmin (X|E) \geq -\log \big(\frac12(\frac12+\frac12 D(\sigma_E^{X,0},\sigma_E^{X,1}) +\frac12(\frac12+\frac12 D(\sigma_E^{Z,+},\sigma_E^{Z,-}))\big) \\
%\end{equation}
%Simplifying the right-hand side results in
%\begin{equation}
%\Hmin (X|E) \geq 1 - \log \big(1 + \frac{D(\sigma_E^{X,0},\sigma_E^{X,1}) + D(\sigma_E^{Z,+},\sigma_E^{Z,-})}{2}\big)
%\end{equation}
%We want to find the maximum value of $D(\sigma_E^{X,0},\sigma_E^{X,1}) + D(\sigma_E^{Z,+},\sigma_E^{Z,-})$ subject to $2p-1 \leq \sqrt{1-D(\sigma_E^{X,0},\sigma_E^{X,1})^2} + \sqrt{1-D(\sigma_E^{Z,+},\sigma_E^{Z,-})^2}$.
%We look at this as an optimization problem. The maximum value of $x+y$ subject to the constraint $\sqrt{1-x^2} + \sqrt{1-y^2} \geq \alpha$ is achieved when $\sqrt{1-x^2} + \sqrt{1-y^2} = \alpha$. Hence, we can use the Lagrange multiplier method to find that the maximum value of $x+y$ is $\sqrt{4-\alpha^2}$. Plugging in $\alpha = 2p-1$ gives that the maximum value of $x+y$ is $2\sqrt{p(1-p)+\frac34}$. Plugging this into the inequality for the min-entropy above gives the desired bound
%\begin{equation}
%\Hmin (X|E) \geq 1 - \log\big(1+\sqrt{p(1-p)+ \frac34}\,\,\big)
%\end{equation}
%}
\end{enumerate}


\end{enumerate}
\end{document}






 
















