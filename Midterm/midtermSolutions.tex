\documentclass[12pt]{article}
\usepackage{fullpage}
\usepackage{amssymb,amsmath}

\newtheorem{theorem}{Theorem}

 \newcommand{\Header}[1]{\begin{center} {\Large\bf #1} \end{center}}
 \newcommand{\header}[1]{\begin{center} {\large\bf #1} \end{center}}
\setlength{\parindent}{0.0in}
\setlength{\parskip}{1ex}


%\newif\ifnotes\notestrue
\newif\ifnotes\notesfalse


\usepackage{amsmath,amssymb,amsthm,amsfonts,latexsym,bbm,xspace,graphicx,float,mathtools,epigraph}
\usepackage[backref,colorlinks,citecolor=blue,bookmarks=true]{hyperref}
\usepackage{enumitem,manyfoot,fullpage}
\usepackage{subfig,tikz,framed}
\usepackage{endnotes}
\usepackage{braket}


\usepackage{fullpage}
\usepackage{hyperref}
\usepackage{pdfsync}
\usepackage{microtype}
\usepackage{color}
\usepackage{cleveref}

\newtheorem*{namedtheorem}{\theoremname}
\newcommand{\theoremname}{testing}
\newenvironment{named}[1]{ \renewcommand{\theoremname}{#1} \begin{namedtheorem}} {\end{namedtheorem}}
\newtheorem{lemma}[theorem]{Lemma}
\newtheorem{claim}[theorem]{Claim}
\newtheorem{proposition}[theorem]{Proposition}
\newtheorem{fact}[theorem]{Fact}
\newtheorem{corollary}[theorem]{Corollary}

\theoremstyle{definition}
\newtheorem{definition}[theorem]{Definition}
\newtheorem{remark}[theorem]{Remark}
\newtheorem{observation}[theorem]{Observation}
\newtheorem{notation}[theorem]{Notation}
\newtheorem{example}[theorem]{Example}
\newtheorem{examples}[theorem]{Examples}
\newtheorem{exercise}{Exercise}


\newenvironment{quotenote}{
\begin{quote}
  \footnotesize
\noindent{\bf Note:}}
{\end{quote}
}


% probability and other mathops
\renewcommand{\Pr}{\mathop{\bf Pr\/}}
\newcommand{\E}{\mathop{\bf E\/}}
\newcommand{\Ex}{\mathop{\bf E\/}}
\newcommand{\Var}{\mathop{\bf Var\/}}
\newcommand{\Cov}{\mathop{\bf Cov\/}}
\newcommand{\stddev}{\mathop{\bf stddev\/}}
\newcommand{\littlesum}{\mathop{{\textstyle \sum}}}
\newcommand{\apx}{\mathop{\approx}}

\newcommand{\epr}{\textsc{EPR}}

\newcommand{\Zt}{\ensuremath{\Z_t}}
\newcommand{\Zp}{\ensuremath{\Z_p}}
\newcommand{\Zq}{\ensuremath{\Z_q}}
\newcommand{\ZN}{\ensuremath{\Z_N}}
\newcommand{\Zps}{\ensuremath{\Z_p^*}}
\newcommand{\ZNs}{\ensuremath{\Z_N^*}}
\newcommand{\JN}{\ensuremath{\J_N}}
\newcommand{\QR}{\ensuremath{\mathbb{QR}}}
\newcommand{\QRN}{\ensuremath{\QR_{N}}}
\newcommand{\QRp}{\ensuremath{\QR_{p}}}

% mathrm terms
\newcommand{\poly}{\mathrm{poly}}
\newcommand{\negl}{\mathrm{negl}}
\newcommand{\Tr}{\mathrm{Tr}}
\newcommand{\polylog}{\mathrm{polylog}}
\newcommand{\size}{\mathrm{size}}
\newcommand{\avg}{\mathop{\mathrm{avg}}}
\newcommand{\sgn}{\mathrm{sgn}}
\newcommand{\dist}{\mathrm{dist}}
\newcommand{\spn}{\mathrm{span}}
\newcommand{\supp}{\mathrm{supp}}
\newcommand{\Val}{\mathrm{Val}}
\newcommand{\Opt}{\mathrm{Opt}}
\newcommand{\LPOpt}{\mathrm{LPOpt}}
\newcommand{\SDPOpt}{\mathrm{SDPOpt}}
\newcommand{\vol}{\mathrm{vol}}
\newcommand{\Id}{\mathbb{I}}

\newcommand{\Ext}{\mathrm{Ext}}
\newcommand{\IP}{\mathrm{IP}}

% number systems
\newcommand{\R}{\mathbbm R}
\newcommand{\C}{\mathbbm C}
\newcommand{\N}{\mathbbm N}
\newcommand{\Z}{\mathbbm Z}
\newcommand{\F}{\mathbbm F}
\newcommand{\Q}{\mathbbm Q}

\newcommand{\mH}{\mathcal{H}}

% complexity classes
\newcommand{\PTIME}{\mathsf{P}}
\newcommand{\NP}{\mathsf{NP}} \newcommand{\np}{\NP}

% short forms
\newcommand{\eps}{\varepsilon}
\newcommand{\lam}{\lambda}
\newcommand{\vphi}{\varphi}
\newcommand{\la}{\langle}
\newcommand{\ra}{\rangle}
\newcommand{\wt}[1]{\widetilde{#1}}
\newcommand{\wh}[1]{\widehat{#1}}
\newcommand{\ul}[1]{\underline{#1}}
\newcommand{\ol}[1]{\overline{#1}}
\newcommand{\ot}{\otimes}
\newcommand{\Ra}{\Rightarrow}
\newcommand{\half}{\tfrac{1}{2}}
\newcommand{\grad}{\nabla}
\newcommand{\sse}{\subseteq}


% calligraphic letters
\newcommand{\calA}{\mathcal{A}}
\newcommand{\calB}{\mathcal{B}}
\newcommand{\calC}{\mathcal{C}}
\newcommand{\calD}{\mathcal{D}}
\newcommand{\calE}{\mathcal{E}}
\newcommand{\calF}{\mathcal{F}}
\newcommand{\calG}{\mathcal{G}}
\newcommand{\calH}{\mathcal{G}}
\newcommand{\calI}{\mathcal{I}}
\newcommand{\calJ}{\mathcal{J}}
\newcommand{\calK}{\mathcal{K}}
\newcommand{\calL}{\mathcal{L}}
\newcommand{\calM}{\mathcal{M}}
\newcommand{\calN}{\mathcal{N}}
\newcommand{\calO}{\mathcal{O}}
\newcommand{\calP}{\mathcal{P}}
\newcommand{\calQ}{\mathcal{Q}}
\newcommand{\calR}{\mathcal{R}}
\newcommand{\calS}{\mathcal{S}}
\newcommand{\calT}{\mathcal{T}}
\newcommand{\calU}{\mathcal{U}}
\newcommand{\calV}{\mathcal{V}}
\newcommand{\calW}{\mathcal{W}}
\newcommand{\calX}{\mathcal{X}}
\newcommand{\calY}{\mathcal{Y}}
\newcommand{\calZ}{\mathcal{Z}}

%\newcommand{\ketbra}[2]{|#1\rangle\langle#2|}
\newcommand{\hmin}{H_{\rm min}}
\newcommand{\Hmin}{H_{\rm min}}

\newcommand{\myfig}[4]{\begin{figure}[H] \begin{center} \includegraphics[width=#1\textwidth]{#2} \caption{#3} \label{#4} \end{center} \end{figure}} 

\newcommand{\bit}{\ensuremath{\{0,1\}}}

%%% CRYPTO-RELATED NOTATION

% length of a string
\newcommand{\len}[1]{\lvert{#1}\rvert}
\newcommand{\lenfit}[1]{\left\lvert{#1}\right\rvert}
% length of some vector, element
\newcommand{\length}[1]{\lVert{#1}\rVert}
\newcommand{\lengthfit}[1]{\left\lVert{#1}\right\rVert}


% types of indistinguishability
\newcommand{\compind}{\ensuremath{\stackrel{c}{\approx}}}
\newcommand{\statind}{\ensuremath{\stackrel{s}{\approx}}}
\newcommand{\perfind}{\ensuremath{\equiv}}

% font for general-purpose algorithms
\newcommand{\algo}[1]{\ensuremath{\mathsf{#1}}}
% font for general-purpose computational problems
\newcommand{\problem}[1]{\ensuremath{\mathsf{#1}}}
% font for complexity classes
\newcommand{\class}[1]{\ensuremath{\mathsf{#1}}}


% KEYS AND RELATED

\newcommand{\key}[1]{\ensuremath{#1}}

\newcommand{\pk}{\key{pk}}
\newcommand{\vk}{\key{vk}}
\newcommand{\sk}{\key{sk}}
\newcommand{\mpk}{\key{mpk}}
\newcommand{\msk}{\key{msk}}
\newcommand{\fk}{\key{fk}}
\newcommand{\id}{id}
\newcommand{\keyspace}{\ensuremath{\mathcal{K}}}
\newcommand{\msgspace}{\ensuremath{\mathcal{M}}}
\newcommand{\ctspace}{\ensuremath{\mathcal{C}}}
\newcommand{\tagspace}{\ensuremath{\mathcal{T}}}
\newcommand{\idspace}{\ensuremath{\mathcal{ID}}}

\newcommand{\concat}{\ensuremath{\|}}

% GAMES

% advantage
\newcommand{\advan}{\ensuremath{\mathbf{Adv}}}

% different attack models
\newcommand{\attack}[1]{\ensuremath{\text{#1}}}

\newcommand{\atk}{\attack{atk}} % dummy attack
\newcommand{\indcpa}{\attack{ind-cpa}}
\newcommand{\indcca}{\attack{ind-cca}}
\newcommand{\anocpa}{\attack{ano-cpa}} % anonymous
\newcommand{\anocca}{\attack{ano-cca}}
\newcommand{\euacma}{\attack{eu-acma}} % forgery: adaptive chosen-message
\newcommand{\euscma}{\attack{eu-scma}} % forgery: static chosen-message
\newcommand{\suacma}{\attack{su-acma}} % strongly unforgeable

% ADVERSARIES
\newcommand{\attacker}[1]{\ensuremath{\mathcal{#1}}}

\newcommand{\Adv}{\attacker{A}}
\newcommand{\AdvA}{\attacker{A}}
\newcommand{\AdvB}{\attacker{B}}
\newcommand{\Dist}{\attacker{D}}
\newcommand{\Sim}{\attacker{S}}
\newcommand{\Ora}{\attacker{O}}
\newcommand{\Inv}{\attacker{I}}
\newcommand{\For}{\attacker{F}}

% CRYPTO SCHEMES

\newcommand{\scheme}[1]{\ensuremath{\text{#1}}}

% pseudorandom stuff
\newcommand{\prg}{\algo{PRG}}
\newcommand{\prf}{\algo{PRF}}
\newcommand{\prp}{\algo{PRP}}

% symmetric-key cryptosystem
\newcommand{\skc}{\scheme{SKC}}
\newcommand{\skcgen}{\algo{Gen}}
\newcommand{\skcenc}{\algo{Enc}}
\newcommand{\skcdec}{\algo{Dec}}

% public-key cryptosystem
\newcommand{\pkc}{\scheme{PKC}}
\newcommand{\pkcgen}{\algo{Gen}}
\newcommand{\pkcenc}{\algo{Enc}} % can also use \kemenc and \kemdec
\newcommand{\pkcdec}{\algo{Dec}}

% digital signatures
\newcommand{\sig}{\scheme{SIG}}
\newcommand{\siggen}{\algo{Gen}}
\newcommand{\sigsign}{\algo{Sign}}
\newcommand{\sigver}{\algo{Ver}}

% message authentication code
\newcommand{\mac}{\scheme{MAC}}
\newcommand{\macgen}{\algo{Gen}}
\newcommand{\mactag}{\algo{Tag}}
\newcommand{\macver}{\algo{Ver}}

% key-encapsulation mechanism
\newcommand{\kem}{\scheme{KEM}}
\newcommand{\kemgen}{\algo{Gen}}
\newcommand{\kemenc}{\algo{Encaps}}
\newcommand{\kemdec}{\algo{Decaps}}

% identity-based encryption
\newcommand{\ibe}{\scheme{IBE}}
\newcommand{\ibesetup}{\algo{Setup}}
\newcommand{\ibeext}{\algo{Ext}}
\newcommand{\ibeenc}{\algo{Enc}}
\newcommand{\ibedec}{\algo{Dec}}

% hierarchical IBE (as key encapsulation)
\newcommand{\hibe}{\scheme{HIBE}}
\newcommand{\hibesetup}{\algo{Setup}}
\newcommand{\hibeext}{\algo{Extract}}
\newcommand{\hibeenc}{\algo{Encaps}}
\newcommand{\hibedec}{\algo{Decaps}}

% binary tree encryption (as key encapsulation)
\newcommand{\bte}{\scheme{BTE}}
\newcommand{\btesetup}{\algo{Setup}}
\newcommand{\bteext}{\algo{Extract}}
\newcommand{\bteenc}{\algo{Encaps}}
\newcommand{\btedec}{\algo{Decaps}}

% trapdoor functions
\newcommand{\tdf}{\scheme{TDF}}
\newcommand{\tdfgen}{\algo{Gen}}
\newcommand{\tdfeval}{\algo{Eval}}
\newcommand{\tdfinv}{\algo{Invert}}
\newcommand{\tdfver}{\algo{Ver}}

%%% PROTOCOLS

\newcommand{\out}{\text{out}}
\newcommand{\view}{\text{view}}


\newcommand{\proj}[1]{\ket{#1}\!\!\bra{#1}}
\newcommand{\ketbra}[2]{\ket{#1}\!\bra{#2}}
\renewcommand{\braket}[2]{\langle #1 \vert  #2\rangle}


\newcommand{\inp}[2]{\langle{#1}|{#2}\rangle} 


%%%%%%%%%%%% document-writing macros %%%%%%%%%%%%

\ifnotes
\usepackage{color}
\definecolor{mygrey}{gray}{0.50}
\newcommand{\notename}[2]{{\textcolor{mygrey}{\footnotesize{\bf (#1:} {#2}{\bf ) }}}}
\newcommand{\noteswarning}{{\begin{center} {\Large WARNING: NOTES ON}\endnote{Warning: notes on}\end{center}}}

\else

\newcommand{\notename}[2]{{}}
\newcommand{\noteswarning}{{}}
\newcommand{\notesendofpaper}{}

\fi

\newcommand{\tnote}[1]{{\notename{Thomas}{#1}}}
\newcommand{\note}[1]{{\notename{Note}{#1}}}
%\newcommand{\note}[1]{}


\bibliographystyle{alpha}



\begin{document}
\header{COM-440, Introduction to Quantum Cryptography, Fall 2025}
{\bf Homework \# 1} \hfill {\bf due: 12:59PM, October 8th, 2019}


\medskip

\hrule

\medskip 

Ground rules: 

{\bf Please
  format your solutions so that each problem begins on a new page, and
  so that your name appears at the top of each page.}

You are encouraged to collaborate with your classmates on
homework problems, but each person must write up the final solutions
individually. You should note on your homework specifically which
problems were a collaborative effort and with whom. You may not search
online for solutions, but if you do use research papers or other
sources in your solutions, you must cite them.

Late homework will not be accepted or graded. Extensions will not be granted, except on the recommendation of a dean. We will grade as many problems as possible, but sometimes one or two problems will not be graded. Your lowest homework grade of the semester will be dropped from your final grade.


\medskip

\hrule

 
\medskip
{\bf Problems:}

\begin{enumerate}
		
\item {\bf An optimal attack}\label{ex:opt-wiesner}
 
%\[ N_1 = \frac{1}{\sqrt{12}} \begin{pmatrix} 3 & 0 \\ 0 & 1 \\ 0 & 1 \\ 1 & 0 \end{pmatrix} \quad \text{and}\quad N_2 = \frac{1}{\sqrt{12}} \begin{pmatrix} 0 & 1 \\ 1 & 0 \\ 1 & 0 \\ 0 & 3\end{pmatrix}\;.\]
\begin{enumerate}
\item We check that 
\begin{align*}
N_1^\dagger N_1 + N_2^\dagger N_2  &= \frac{1}{12}\begin{pmatrix} 10 & 0 \\ 0 & 2 \end{pmatrix}+\frac{1}{12} \begin{pmatrix} 2 & 0 \\ 0 & 10 \end{pmatrix} = \begin{pmatrix} 1 & 0 \\ 0 & 1\end{pmatrix}\;,
\end{align*}
as should be the case. 
\item For example, we can compute the image of $\ket{+}$ as 
\begin{align*} 
\mathcal{N}(\proj{+}) &= N_1 \proj{+} N_1^\dagger + N_2 \proj{+}N_2^\dagger \\
&= \frac{1}{12} \frac{1}{2}\big( 3\ket{00} + \ket{01} + \ket{10} + \ket{11} \big)\big( 3\bra{00} + \bra{01} + \bra{10} + \bra{11} \big)\\
&\qquad +\frac{1}{12} \frac{1}{2}\big( \ket{00} + \ket{01} + \ket{10} + 3\ket{11} \big)\big( \bra{00} + \bra{01} + \bra{10} + 3\ket{11} \big)\;.
\end{align*}
We can then verify that 
\begin{align*}
\bra{+}\bra{+}\mathcal{N}(\proj{+})\ket{+}\ket{+} &= \frac{1}{4}\frac{1}{24}\big( (3+1+1+1)^2 + (1+1+1+3)^2 \big)\\
&= \frac{1}{96}( 36+36) = \frac{3}{4}\;.
\end{align*}
A similar calculation gives the same for the three other BB'84 states!
\end{enumerate}






\item  \textbf{Secret sharing among three people.}
\begin{enumerate}[(a)]
\item We directly calculate one matrix and use the symmetry among $A,B,C$ to extrapolate the others. $$\begin{aligned}\rho_{A} & = \text{Tr}_{BC}(\ketbra{\Psi}\Psi) \\  & = \frac{1}2(\ketbra{0}0_A\otimes\text{tr}(\ketbra{0}0_B)\otimes \text{tr}(\ketbra{0}0_C) + (-1)^{2b} \ketbra{1}1_A\otimes\text{tr}(\ketbra{1}1_B)\otimes \text{tr}(\ketbra{1}1_C)) \\ & = \frac{1}2(\ketbra{0}0_A + \ketbra{1}1_A) \\ & = \frac{\mathbb{I}}2\end{aligned}$$ We similarly have $\rho_B =\frac{1}2(\ketbra{0}0_B + \ketbra{1}1_B) = \frac{\mathbb{I}}2$ and $\rho_C = \frac{1}2(\ketbra{0}0_C + \ketbra{1}1_C) = \frac{\mathbb{I}}2$. Notice that this state is independent of $b$ and so any measurement these people take alone will be independent of $b$, and the secret cannot be retrieved on their own. \\ 

\item Again calculate one matrix and use the symmetry among $A,B,C$ to extrapolate the others. $$\begin{aligned}\rho_{AB} & = \text{Tr}_{C}(\ketbra{\Psi}\Psi) \\  & = \frac{1}2(\ketbra{0}0_A\otimes\ketbra{0}0_B\otimes \text{tr}(\ketbra{0}0_C) + (-1)^{2b} \ketbra{1}1_A\otimes\ketbra{1}1_B\otimes \text{tr}(\ketbra{1}1_C)) \\ & = \frac{1}2(\ketbra{0}0_A\otimes \ketbra{0}0_B + \ketbra{1}1_A\otimes \ketbra{1}1_B) \\ & = \frac{1}2(\ketbra{00}{00}_{AB} + \ketbra{11}{11}_{AB})\end{aligned}$$ and similarly $\rho_{BC} = \frac{1}2(\ketbra{00}{00}_{BC} + \ketbra{11}{11}_{BC})$ and $\rho_{AC} = \frac{1}2(\ketbra{00}{00}_{AC} + \ketbra{11}{11}_{AC})$. Again, we see that these densities are independent of $B$, so any measurement made by any pair of people will not give information about $b$.

\item Have Alice, Bob, and Charlie apply the Hadamard operation on their qubit to get the new state $$\ket{\Psi_0} = \frac{1}4((\ket{0}+\ket{1})^3 + (\ket{0}-\ket{1})^3) = \frac{1}4((\ket{000}+\ket{011}+\ket{110}+\ket{101})$$ if $b=0$ and the new state $$ \ket{\Psi_1} = \frac{1}4((\ket{0}+\ket{1})^3 - (\ket{0} - \ket{1})^3) = \frac{1}4 (\ket{001}+\ket{010}+\ket{100}+\ket{111})$$ if $b=1$. Next, we have Alice, Bob, and Charlie measure their qubits. Note by inspection, the tensor product of all three observations must either be in the set $$S_0 = \{\ket{0}_A\ket{0}_B\ket{0}_C, \ket{0}_A\ket{1}_B\ket{1}_C,  \ket{1}_A\ket{1}_B\ket{0}_C,\ket{1}_A\ket{0}_B\ket{1}_C\}$$ if $b=0$, and must be in the set $$S_1 = \{\ket{0}_A\ket{0}_B\ket{1}_C, \ket{0}_A\ket{1}_B\ket{0}_C,  \ket{1}_A\ket{0}_B\ket{0}_C,\ket{1}_A\ket{1}_B\ket{1}_C\}$$ if $b=1$. \\ 

 Hence we can have Alice and Bob send their measurement to Charlie, and Charlie will see whether their collective measurements are in $S_0$ or $S_1$ to recover $b$.
\end{enumerate} 





\item \textbf{A Guessing Game.}
\begin{enumerate}[(a)]
\item If we assume $U_A(\ket{0}) = \alpha_0\ket{0}+\beta_0\ket{1}$ and $U_A(\ket{1}) = \alpha_1\ket{0}+\beta_1\ket{1}$we can note that $$\begin{aligned}(U_A\otimes \mathbb{I}_B)\frac{1}{\sqrt{2}}(\ket{00}+\ket{11}) & = \frac{1}{\sqrt{2}}(U_A(\ket{0})\otimes \ket{0} + U_A(\ket{1})\otimes \ket{1}) \\ & = \frac{1}{\sqrt{2}}(\alpha_0\ket{00}+\beta_0\ket{10} + \alpha_1\ket{01}+\beta_1\ket{11}) \\ & = \frac{1}{\sqrt{2}} (\ket{0}\otimes (\alpha_0\ket{0}+\alpha_1\ket{1}) + \ket{1}\otimes (\beta_0\ket{0} + \beta_1\ket{1})) \\ & = \frac{1}{\sqrt{2}}(\ket{0}\otimes U_B^T(\ket{0}) + \ket{1}\otimes U_B^T(\ket{1})) \\ & = (\mathbb{I}_A\otimes U_B^T)\frac{1}{\sqrt{2}}(\ket{00}+\ket{11})\end{aligned}$$ Hence, if Eve applies the unitary transformation $(U^T)^{-1} = (U^\dagger)^T = \conj{U}$, we will have undone this unitary transformation. Then we can replicate the argument given in the problem statement, and so using $\theta$, can measure in the same basis that Alice did and perfectly recover Alice's bit with probability 1. \\

\item We claim that by using only two matrices unitary matrices we can get a success probability of 1/2, which is equivalent to simply guessing. Simply have Alice choose one of $\mathbb{I}$ or $H$ uniformly at random irregardless of $\theta$. Then for either value of $\theta$, the density matrix for $\rho_E$ would be the mixture $$ \rho_E = \frac{1}4 (\ketbra{0}0 + \ketbra{1}1 + \ketbra{+}+ +\ketbra{-}-) = \frac{\mathbb{I}}2$$ which is independent of $\theta$. Hence Eve's best bet is just guessing independently at random.

\item See part (b)
\end{enumerate}

\item {\bf Robustness of GHZ and W States, Part 2}

\begin{enumerate}
	\item By direct calculation we have $$\Tr_N \proj{GHZ_N} = \frac{1}{2}\proj{0}^{\otimes {N-1}}+\frac{1}{2}\proj{1}^{\otimes {N-1}}$$ and $$\Tr_ N \proj{W_N} = \frac{N-1}{N}\proj{W_{N-1}}+\frac{1}{N}\proj{0}^{\otimes {N-1}}\;.$$ Both of these are diagonal (in some basis) and have rank 2. Note that this is also the \emph{highest} rank one can get when tracing out a single qubit, as $\rho_A=\rho_B$.
	
	\item For $\rho =\proj{0}$ we have $\rho^2=\rho$ and thus $\Tr(\rho^2)=1$. On the other hand, for $\rho =\frac{1}{d} \id_d$ we have $\rho^2=\frac{1}{d^2} \id_ d$ from which it follows that $\Tr(\rho^2)=\frac{1}{d}$
	
	\item The extremes (pure and maximally mixed) that you considered in Problem 2.2 certainly suggest this. Informally, the more entangled $A$ and $B$ are, the more classical uncertainty you have --- the more information you lose --- in the state $\rho _ A$ of $A$ alone after tracing out $B$. This expresses itself as a lower purity as defined above.
	
	\item Again we have by direct calculation $$\rho = \Tr_ N \proj{GHZ_N} = \frac{1}{2}\proj{0}^{\otimes {N-1}}+\frac{1}{2}\proj{1}^{\otimes {N-1}}\;,$$ from which it follows that $$\rho^2 = \frac{1}{4}\proj{0}^{\otimes {N-1}}+\frac{1}{4}\proj{1}^{\otimes {N-1}}$$ and $\Tr(\rho^2)=\frac{1}{2}$ for all $N$.
	
	\item We have again by direct calculation
	$$\rho = \Tr_N \proj{W_N} = \frac{N-1}{N}\proj{W_{N-1}}+\frac{1}{N}\proj{0}^{\otimes {N-1}}\;,$$ from which it follows that $$\rho ^2 = \frac{(N-1)^2}{N^2}\proj{W_{N-1}}+\frac{1}{N^2}\proj{0}^{\otimes {N-1}}$$ and $\Tr( \rho ^2) = \frac{N^2-2N+2}{N^2} \rightarrow 1$ as $N \rightarrow \infty$.
	
	As $N$ grows, the $\ket{GHZ_N}$ states to which one qubit has been discarded have a lower purity than the $\ket{W_N}$ states. According to the preceding discussion, this means that there is more entanglement between $(N-1)$ and $1$ qubits of a $\ket{GHZ_N}$ state, than there is in a $\ket{W_N}$ state. Conversely, if we consider the qubit to be ``lost'' then there is less entanglement remaining in the $\ket{GHZ_N}$ state. If we remove two qubits, then continuing the calculations made in 4. and 5. we see that the result is essentially unchanged.
\end{enumerate}

\end{enumerate}
\end{document}






 
















