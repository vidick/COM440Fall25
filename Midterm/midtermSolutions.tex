\documentclass[12pt]{article}
\usepackage{fullpage}
\usepackage{amssymb,amsmath}

\newtheorem{theorem}{Theorem}

 \newcommand{\Header}[1]{\begin{center} {\Large\bf #1} \end{center}}
 \newcommand{\header}[1]{\begin{center} {\large\bf #1} \end{center}}
\setlength{\parindent}{0.0in}
\setlength{\parskip}{1ex}


%\newif\ifnotes\notestrue
\newif\ifnotes\notesfalse


\usepackage{amsmath,amssymb,amsthm,amsfonts,latexsym,bbm,xspace,graphicx,float,mathtools,epigraph}
\usepackage[backref,colorlinks,citecolor=blue,bookmarks=true]{hyperref}
\usepackage{enumitem,manyfoot,fullpage}
\usepackage{subfig,tikz,framed}
\usepackage{endnotes}
\usepackage{braket}


\usepackage{fullpage}
\usepackage{hyperref}
\usepackage{pdfsync}
\usepackage{microtype}
\usepackage{color}
\usepackage{cleveref}

\newtheorem*{namedtheorem}{\theoremname}
\newcommand{\theoremname}{testing}
\newenvironment{named}[1]{ \renewcommand{\theoremname}{#1} \begin{namedtheorem}} {\end{namedtheorem}}
\newtheorem{lemma}[theorem]{Lemma}
\newtheorem{claim}[theorem]{Claim}
\newtheorem{proposition}[theorem]{Proposition}
\newtheorem{fact}[theorem]{Fact}
\newtheorem{corollary}[theorem]{Corollary}

\theoremstyle{definition}
\newtheorem{definition}[theorem]{Definition}
\newtheorem{remark}[theorem]{Remark}
\newtheorem{observation}[theorem]{Observation}
\newtheorem{notation}[theorem]{Notation}
\newtheorem{example}[theorem]{Example}
\newtheorem{examples}[theorem]{Examples}
\newtheorem{exercise}{Exercise}


\newenvironment{quotenote}{
\begin{quote}
  \footnotesize
\noindent{\bf Note:}}
{\end{quote}
}


% probability and other mathops
\renewcommand{\Pr}{\mathop{\bf Pr\/}}
\newcommand{\E}{\mathop{\bf E\/}}
\newcommand{\Ex}{\mathop{\bf E\/}}
\newcommand{\Var}{\mathop{\bf Var\/}}
\newcommand{\Cov}{\mathop{\bf Cov\/}}
\newcommand{\stddev}{\mathop{\bf stddev\/}}
\newcommand{\littlesum}{\mathop{{\textstyle \sum}}}
\newcommand{\apx}{\mathop{\approx}}

\newcommand{\epr}{\textsc{EPR}}

\newcommand{\Zt}{\ensuremath{\Z_t}}
\newcommand{\Zp}{\ensuremath{\Z_p}}
\newcommand{\Zq}{\ensuremath{\Z_q}}
\newcommand{\ZN}{\ensuremath{\Z_N}}
\newcommand{\Zps}{\ensuremath{\Z_p^*}}
\newcommand{\ZNs}{\ensuremath{\Z_N^*}}
\newcommand{\JN}{\ensuremath{\J_N}}
\newcommand{\QR}{\ensuremath{\mathbb{QR}}}
\newcommand{\QRN}{\ensuremath{\QR_{N}}}
\newcommand{\QRp}{\ensuremath{\QR_{p}}}

% mathrm terms
\newcommand{\poly}{\mathrm{poly}}
\newcommand{\negl}{\mathrm{negl}}
\newcommand{\Tr}{\mathrm{Tr}}
\newcommand{\polylog}{\mathrm{polylog}}
\newcommand{\size}{\mathrm{size}}
\newcommand{\avg}{\mathop{\mathrm{avg}}}
\newcommand{\sgn}{\mathrm{sgn}}
\newcommand{\dist}{\mathrm{dist}}
\newcommand{\spn}{\mathrm{span}}
\newcommand{\supp}{\mathrm{supp}}
\newcommand{\Val}{\mathrm{Val}}
\newcommand{\Opt}{\mathrm{Opt}}
\newcommand{\LPOpt}{\mathrm{LPOpt}}
\newcommand{\SDPOpt}{\mathrm{SDPOpt}}
\newcommand{\vol}{\mathrm{vol}}
\newcommand{\Id}{\mathbb{I}}

% number systems
\newcommand{\R}{\mathbbm R}
\newcommand{\C}{\mathbbm C}
\newcommand{\N}{\mathbbm N}
\newcommand{\Z}{\mathbbm Z}
\newcommand{\F}{\mathbbm F}
\newcommand{\Q}{\mathbbm Q}

\newcommand{\mH}{\mathcal{H}}

% complexity classes
\newcommand{\PTIME}{\mathsf{P}}
\newcommand{\NP}{\mathsf{NP}} \newcommand{\np}{\NP}

% short forms
\newcommand{\eps}{\varepsilon}
\newcommand{\lam}{\lambda}
\newcommand{\vphi}{\varphi}
\newcommand{\la}{\langle}
\newcommand{\ra}{\rangle}
\newcommand{\wt}[1]{\widetilde{#1}}
\newcommand{\wh}[1]{\widehat{#1}}
\newcommand{\ul}[1]{\underline{#1}}
\newcommand{\ol}[1]{\overline{#1}}
\newcommand{\ot}{\otimes}
\newcommand{\Ra}{\Rightarrow}
\newcommand{\half}{\tfrac{1}{2}}
\newcommand{\grad}{\nabla}
\newcommand{\sse}{\subseteq}


% calligraphic letters
\newcommand{\calA}{\mathcal{A}}
\newcommand{\calB}{\mathcal{B}}
\newcommand{\calC}{\mathcal{C}}
\newcommand{\calD}{\mathcal{D}}
\newcommand{\calE}{\mathcal{E}}
\newcommand{\calF}{\mathcal{F}}
\newcommand{\calG}{\mathcal{G}}
\newcommand{\calH}{\mathcal{G}}
\newcommand{\calI}{\mathcal{I}}
\newcommand{\calJ}{\mathcal{J}}
\newcommand{\calK}{\mathcal{K}}
\newcommand{\calL}{\mathcal{L}}
\newcommand{\calM}{\mathcal{M}}
\newcommand{\calN}{\mathcal{N}}
\newcommand{\calO}{\mathcal{O}}
\newcommand{\calP}{\mathcal{P}}
\newcommand{\calQ}{\mathcal{Q}}
\newcommand{\calR}{\mathcal{R}}
\newcommand{\calS}{\mathcal{S}}
\newcommand{\calT}{\mathcal{T}}
\newcommand{\calU}{\mathcal{U}}
\newcommand{\calV}{\mathcal{V}}
\newcommand{\calW}{\mathcal{W}}
\newcommand{\calX}{\mathcal{X}}
\newcommand{\calY}{\mathcal{Y}}
\newcommand{\calZ}{\mathcal{Z}}


\newcommand{\myfig}[4]{\begin{figure}[H] \begin{center} \includegraphics[width=#1\textwidth]{#2} \caption{#3} \label{#4} \end{center} \end{figure}} 

\newcommand{\bit}{\ensuremath{\{0,1\}}}

%%% CRYPTO-RELATED NOTATION

% length of a string
\newcommand{\len}[1]{\lvert{#1}\rvert}
\newcommand{\lenfit}[1]{\left\lvert{#1}\right\rvert}
% length of some vector, element
\newcommand{\length}[1]{\lVert{#1}\rVert}
\newcommand{\lengthfit}[1]{\left\lVert{#1}\right\rVert}


% types of indistinguishability
\newcommand{\compind}{\ensuremath{\stackrel{c}{\approx}}}
\newcommand{\statind}{\ensuremath{\stackrel{s}{\approx}}}
\newcommand{\perfind}{\ensuremath{\equiv}}

% font for general-purpose algorithms
\newcommand{\algo}[1]{\ensuremath{\mathsf{#1}}}
% font for general-purpose computational problems
\newcommand{\problem}[1]{\ensuremath{\mathsf{#1}}}
% font for complexity classes
\newcommand{\class}[1]{\ensuremath{\mathsf{#1}}}


% KEYS AND RELATED

\newcommand{\key}[1]{\ensuremath{#1}}

\newcommand{\pk}{\key{pk}}
\newcommand{\vk}{\key{vk}}
\newcommand{\sk}{\key{sk}}
\newcommand{\mpk}{\key{mpk}}
\newcommand{\msk}{\key{msk}}
\newcommand{\fk}{\key{fk}}
\newcommand{\id}{id}
\newcommand{\keyspace}{\ensuremath{\mathcal{K}}}
\newcommand{\msgspace}{\ensuremath{\mathcal{M}}}
\newcommand{\ctspace}{\ensuremath{\mathcal{C}}}
\newcommand{\tagspace}{\ensuremath{\mathcal{T}}}
\newcommand{\idspace}{\ensuremath{\mathcal{ID}}}

\newcommand{\concat}{\ensuremath{\|}}

% GAMES

% advantage
\newcommand{\advan}{\ensuremath{\mathbf{Adv}}}

% different attack models
\newcommand{\attack}[1]{\ensuremath{\text{#1}}}

\newcommand{\atk}{\attack{atk}} % dummy attack
\newcommand{\indcpa}{\attack{ind-cpa}}
\newcommand{\indcca}{\attack{ind-cca}}
\newcommand{\anocpa}{\attack{ano-cpa}} % anonymous
\newcommand{\anocca}{\attack{ano-cca}}
\newcommand{\euacma}{\attack{eu-acma}} % forgery: adaptive chosen-message
\newcommand{\euscma}{\attack{eu-scma}} % forgery: static chosen-message
\newcommand{\suacma}{\attack{su-acma}} % strongly unforgeable

% ADVERSARIES
\newcommand{\attacker}[1]{\ensuremath{\mathcal{#1}}}

\newcommand{\Adv}{\attacker{A}}
\newcommand{\AdvA}{\attacker{A}}
\newcommand{\AdvB}{\attacker{B}}
\newcommand{\Dist}{\attacker{D}}
\newcommand{\Sim}{\attacker{S}}
\newcommand{\Ora}{\attacker{O}}
\newcommand{\Inv}{\attacker{I}}
\newcommand{\For}{\attacker{F}}

% CRYPTO SCHEMES

\newcommand{\scheme}[1]{\ensuremath{\text{#1}}}

% pseudorandom stuff
\newcommand{\prg}{\algo{PRG}}
\newcommand{\prf}{\algo{PRF}}
\newcommand{\prp}{\algo{PRP}}

% symmetric-key cryptosystem
\newcommand{\skc}{\scheme{SKC}}
\newcommand{\skcgen}{\algo{Gen}}
\newcommand{\skcenc}{\algo{Enc}}
\newcommand{\skcdec}{\algo{Dec}}

% public-key cryptosystem
\newcommand{\pkc}{\scheme{PKC}}
\newcommand{\pkcgen}{\algo{Gen}}
\newcommand{\pkcenc}{\algo{Enc}} % can also use \kemenc and \kemdec
\newcommand{\pkcdec}{\algo{Dec}}

% digital signatures
\newcommand{\sig}{\scheme{SIG}}
\newcommand{\siggen}{\algo{Gen}}
\newcommand{\sigsign}{\algo{Sign}}
\newcommand{\sigver}{\algo{Ver}}

% message authentication code
\newcommand{\mac}{\scheme{MAC}}
\newcommand{\macgen}{\algo{Gen}}
\newcommand{\mactag}{\algo{Tag}}
\newcommand{\macver}{\algo{Ver}}

% key-encapsulation mechanism
\newcommand{\kem}{\scheme{KEM}}
\newcommand{\kemgen}{\algo{Gen}}
\newcommand{\kemenc}{\algo{Encaps}}
\newcommand{\kemdec}{\algo{Decaps}}

% identity-based encryption
\newcommand{\ibe}{\scheme{IBE}}
\newcommand{\ibesetup}{\algo{Setup}}
\newcommand{\ibeext}{\algo{Ext}}
\newcommand{\ibeenc}{\algo{Enc}}
\newcommand{\ibedec}{\algo{Dec}}

% hierarchical IBE (as key encapsulation)
\newcommand{\hibe}{\scheme{HIBE}}
\newcommand{\hibesetup}{\algo{Setup}}
\newcommand{\hibeext}{\algo{Extract}}
\newcommand{\hibeenc}{\algo{Encaps}}
\newcommand{\hibedec}{\algo{Decaps}}

% binary tree encryption (as key encapsulation)
\newcommand{\bte}{\scheme{BTE}}
\newcommand{\btesetup}{\algo{Setup}}
\newcommand{\bteext}{\algo{Extract}}
\newcommand{\bteenc}{\algo{Encaps}}
\newcommand{\btedec}{\algo{Decaps}}

% trapdoor functions
\newcommand{\tdf}{\scheme{TDF}}
\newcommand{\tdfgen}{\algo{Gen}}
\newcommand{\tdfeval}{\algo{Eval}}
\newcommand{\tdfinv}{\algo{Invert}}
\newcommand{\tdfver}{\algo{Ver}}

%%% PROTOCOLS

\newcommand{\out}{\text{out}}
\newcommand{\view}{\text{view}}


\newcommand{\proj}[1]{\ket{#1}\!\bra{#1}}




%%%%%%%%%%%% document-writing macros %%%%%%%%%%%%

\ifnotes
\usepackage{color}
\definecolor{mygrey}{gray}{0.50}
\newcommand{\notename}[2]{{\textcolor{mygrey}{\footnotesize{\bf (#1:} {#2}{\bf ) }}}}
\newcommand{\noteswarning}{{\begin{center} {\Large WARNING: NOTES ON}\endnote{Warning: notes on}\end{center}}}

\else

\newcommand{\notename}[2]{{}}
\newcommand{\noteswarning}{{}}
\newcommand{\notesendofpaper}{}

\fi

\newcommand{\tnote}[1]{{\notename{Thomas}{#1}}}
\newcommand{\note}[1]{{\notename{Note}{#1}}}
%\newcommand{\note}[1]{}


\bibliographystyle{alpha}



\begin{document}
\header{COM-440, Introduction to Quantum Cryptography, Fall 2025}
{\bf Partial Midterm Solutions} \hfill %{\bf due: 12:59PM, October 8th, 2019}


\medskip

\hrule


 
\medskip

\begin{enumerate}
		
\item \textbf{Superdense Coding.}
\begin{enumerate}
\item Here one has to be careful about the task. Clearly you are being asked to evaluate a guessing probability. On what state? If we label Alice's two bits as $x,\theta\in\{0,1\}$ then we can write the qubit that encodes them as $\ket{x}_\theta$. The joint state of Alice's bits and the encoded qubit is a cq state $\rho_{X\Theta E} = \frac{1}{4} \sum_{x,\theta} \proj{x}\otimes\proj{\theta} \otimes \proj{x}_\theta$. We know that $P_{guess}(X\Theta|E)=2^{-H_{min}(X\Theta|E)}$. To evaluate this last term, we note that by the chain rule $H_{min}(X\Theta|E)\geq H_{min}(X\Theta)-\log|E|$. The first term is the min-entropy of two uniformly random bits, so it equals $2$; and the second term is the size of $E$ in qubits, which is $1$. Therefore, we have established $P_{guess}(X\Theta|E)\leq \frac{1}{2}$. To show that there is equality we design a guessing strategy: Eve measures $E$ in the standard basis, obtains an outcome $y\in\{0,1\}$ and returns $(y,0)$ as her two bits. This is correct with probability $1$ in case the basis was indeed the standard basis, i.e.\ $\theta=0$, and it is correct with probability $0$ in case $\theta=1$. Overall, the success probability is $\frac{1}{2}$.  
 \end{enumerate}
The other questions were solved well and we do not detail the solution. 


\item  \textbf{Secret sharing among three people.}
\begin{enumerate}
\item We directly calculate one matrix and use the symmetry among $A,B,C$ to extrapolate the others. $$\begin{aligned}\rho_{A} & = \text{Tr}_{BC}(\ketbra{\Psi}\Psi) \\  & = \frac{1}2(\ketbra{0}0_A\otimes\text{tr}(\ketbra{0}0_B)\otimes \text{tr}(\ketbra{0}0_C) + (-1)^{2b} \ketbra{1}1_A\otimes\text{tr}(\ketbra{1}1_B)\otimes \text{tr}(\ketbra{1}1_C)) \\ & = \frac{1}2(\ketbra{0}0_A + \ketbra{1}1_A) \\ & = \frac{\mathbb{I}}2\end{aligned}$$ We similarly have $\rho_B =\frac{1}2(\ketbra{0}0_B + \ketbra{1}1_B) = \frac{\mathbb{I}}2$ and $\rho_C = \frac{1}2(\ketbra{0}0_C + \ketbra{1}1_C) = \frac{\mathbb{I}}2$. This state is independent of $b$ and so any measurement on a single system will lead to an outcome distribution that is independent of $b$: the secret cannot be retrieved by one party only. \\ 

\item Again calculate one matrix and use the symmetry among $A,B,C$ to extrapolate the others. $$\begin{aligned}\rho_{AB} & = \text{Tr}_{C}(\ketbra{\Psi}\Psi) \\  & = \frac{1}2(\ketbra{0}0_A\otimes\ketbra{0}0_B\otimes \text{tr}(\ketbra{0}0_C) + (-1)^{2b} \ketbra{1}1_A\otimes\ketbra{1}1_B\otimes \text{tr}(\ketbra{1}1_C)) \\ & = \frac{1}2(\ketbra{0}0_A\otimes \ketbra{0}0_B + \ketbra{1}1_A\otimes \ketbra{1}1_B) \\ & = \frac{1}2(\ketbra{00}{00}_{AB} + \ketbra{11}{11}_{AB})\end{aligned}$$ and similarly $\rho_{BC} = \frac{1}2(\ketbra{00}{00}_{BC} + \ketbra{11}{11}_{BC})$ and $\rho_{AC} = \frac{1}2(\ketbra{00}{00}_{AC} + \ketbra{11}{11}_{AC})$.  These densities are independent of $b$, so any measurement made by any pair will not give information about $b$.

\item Have Alice, Bob, and Charlie apply the Hadamard operation on their qubit to get the new state $$\ket{\Psi_0} = \frac{1}4((\ket{0}+\ket{1})^3 + (\ket{0}-\ket{1})^3) = \frac{1}4((\ket{000}+\ket{011}+\ket{110}+\ket{101})$$ if $b=0$ and the new state $$ \ket{\Psi_1} = \frac{1}4((\ket{0}+\ket{1})^3 - (\ket{0} - \ket{1})^3) = \frac{1}4 (\ket{001}+\ket{010}+\ket{100}+\ket{111})$$ if $b=1$. Next, we have Alice, Bob, and Charlie measure their qubits. Note by inspection, the tensor product of all three observations must either be in the set $$S_0 = \{\ket{0}_A\ket{0}_B\ket{0}_C, \ket{0}_A\ket{1}_B\ket{1}_C,  \ket{1}_A\ket{1}_B\ket{0}_C,\ket{1}_A\ket{0}_B\ket{1}_C\}$$ if $b=0$, and must be in the set $$S_1 = \{\ket{0}_A\ket{0}_B\ket{1}_C, \ket{0}_A\ket{1}_B\ket{0}_C,  \ket{1}_A\ket{0}_B\ket{0}_C,\ket{1}_A\ket{1}_B\ket{1}_C\}$$ if $b=1$. \\ 

 Hence we can have Alice and Bob send their measurement to Charlie, and Charlie will see whether their collective measurements are in $S_0$ or $S_1$ to recover $b$.
\end{enumerate} 

\item {\bf Quantum money security variant.}
The scheme is \emph{not} secure under the new security definition. Consider the following adversary, which runs in exponential time. The adversary ignores the quantum bill sent by the challenger. Instead, for every possible pair $x,\theta\in \{0,1\}^n$ the adversary does the following: prepare the $n$-qubit state $\ket{x}_\theta$ and send it for verification. If verification fails, proceed to the next pair $x,\theta$. If it succeeds, then prepare two fresh copies of $\ket{x}_\theta$ (which is possible since $x,\theta$ are known) and submit them as response to the challenger. This adversary succeeds with probability $1$, after at most $2^n\times 2^n$ attempts at verification. 


\item {\bf Inner product extractor.}
\begin{enumerate}
\item The family is not $2$-universal (in fact, it is not even $1$-universal) because for the case where $x=0^n$, $x'\neq x$, $z=1$ and $z'$ arbitrary we calculate $\Pr_y[f_y(x)=z \wedge f_y(x')=z')=0$ (because $f_y(0^n)=0$ for all $y$), and not $\frac{1}{4}$ as would be required for a $2$-universal family (note that here the parameter $m=1$).
\begin{enumerate}
\item This question could be approached in different ways. One possibility is to use the characterization of the trace norm as $\|\sigma_{0,y}-\sigma_{1,y}\|_{tr}=\sup_{0\leq M \leq \Id} \Tr(M(\sigma_{0,y}-\sigma_{1,y}))$. Because $0\leq M\leq \Id$ the right-hand side is always at most $1$. For there to be equality, for the optimal $M$ we must have $\Tr(M\sigma_{0,y})=1$ and $\Tr(M\sigma_{0,y})=0$. Now we can see that if we replace $M$ by the projection $P$ on its support (i.e.\ round the singular values to $1$, unless they are $0$) the condition $\Tr(P\sigma_{0,y})=0$ still holds but $\Tr(P\sigma_{1,y})\geq \Tr(M\sigma_{1,y})=1$. Since of course $\Tr(P\sigma_{1,y})\leq \Tr(\sigma_{1,y})=1$, we are done. 
\item Let $P_y$ be the projection obtained in the previous question. We verify that the map 
\[ U:\ket{\psi}_E\ket{y}_Y\ket{0}_A \mapsto P_y\ket{\psi}_E\ket{y}_Y\ket{0}_A +(\Id-P_y)\ket{\psi}_E\ket{y}_Y\ket{1}_A \;,\]
defined as such on every state $\ket{\psi}_E$ and $\ket{y}$, can be extended into a valid unitary. This is verified by checking that inner products are preserved, i.e. for any $\ket{\psi}_E$, $\ket{\psi'}_E$ and $y,y'$, 
\begin{align*}
\big(P_y\ket{\psi}_E\ket{y}_Y\ket{0}_A &+(\Id-P_y)\ket{\psi}_E\ket{y}_Y\ket{1}_A\big)^\dagger\cdot\big(P_{y'}\ket{\psi'}_E\ket{y'}_Y\ket{0}_A \\
&\qquad\qquad+(\Id-P_{y'})\ket{\psi'}_E\ket{y'}_Y\ket{1}_A\big)\\
&= \delta_{yy'} \bra{\psi}_E P_y \ket{\psi'}_E + \bra{\psi}_E (\Id-P_y) \ket{\psi'}_E \\
&= \delta_{yy'} \langle \psi \ket{\psi'}_E\;,
\end{align*}
where for the last two lines we used that $P_y$ is an orthogonal projection (and in particular $P_y(\Id-P_y)=0$). 
\item This is a direct calculation and we find 
\[U'\ket{\psi_x}_E\ket{y}_Y\ket{0}_A\ket{-}_B=(-1)^{x\cdot y}\ket{\psi_x}_E\ket{y}_Y\ket{0}_A\ket{-}_B\;.\]
\item Given $\ket{\psi_x}$, $V$ first applies some Hadamards on the $Y$ register to create $\frac{1}{\sqrt{2^n}}\sum_y \ket{y}$. It also applies an $X$ followed by a Hadamard on $B$ to create a $\ket{-}$. It then applies $U'$ from the previous question, and undoes all the Hadamards and the $X$. This amounts to performing a QFT and the result is exactly the one desired. 
\item We see that assumption (2) implies that it is possible to exactly recover $X$ from $E$, and thus this is only possible in case $H_{min}(X|E)=0$.
\end{enumerate}
\end{enumerate}


\end{enumerate}
\end{document}






 
















